% Input your problem and solution below.
% Three dashes on a newline indicate the breaking points.
% vim: tw=72

---

Let $ABC$ be an acute triangle with circumcircle $\Gamma$, and let $\Omega$ be a circle centered at $P$ passing through $B$ and $C$. Suppose that $\Omega$ intersects $\overline{CA}$ and $\overline{AB}$ again at $E$ and $F$, respectively. Let $\overline{BE}$ and $\overline{CF}$ intersect at $H$, and line $AH$ intersect $\overline{BC}$ at $D$.
\begin{itemize}
    \item[(a)] Let line $HP$ intersect $\overline{EF}$ at $S$ and the tangents to $\Gamma$ at $B$ and $C$ intersect at $X$. Prove that points $X,D,S$ are collinear.
        \vspace{-0.5em}
    \item[(b)] Let $X$ be the midpoint of $\overline{EF}$ and let line $HX$ intersect $\overline{BC}$ at $Y$. Prove that if $Z$ denotes the foot of the perpendicular from $A$ to line $HX$, then the circumcircle of $\triangle MYZ$ is tangent to $\Gamma$.
\end{itemize}

---

\begin{customsol}{to part (a)}
    Let $T=\overline{BC}\cap\overline{EF}$. We want to prove that $S$ lies on line $XD$, the polar of $T$ with respect to $\Gamma$.
    \begin{center}
        \begin{asy}
            size(11.5cm);
            defaultpen(fontsize(9pt));

            pen pri=deepblue;
            pen sec=deepgreen;
            pen tri=red;
            pen qua=orange;
            pen qui=purple;
            pen fil=Cyan+opacity(0.05);
            pen sfil=springgreen+opacity(0.05);
            pen tfil=red+opacity(0.05);
            pen qfil=orange+opacity(0.05);
            pen qifil=purple+opacity(0.05);

            pair A, B, C, M, K, EE, F, H, D, T, J, G, Lp, NN, P, N2, S1, S2;
            A=dir(130);
            B=dir(215);
            C=dir(325);
            M=(B+C)/2;
            K=M+(0, 1/8);
            EE=2*foot(K, C, A)-C;
            F=2*foot(K, A, B)-B;
            H=extension(B, EE, C, F);
            D=extension(A, H, B, C);
            T=extension(B, C, EE, F);
            J=extension(A, T, K, H);
            G=extension(EE, F, K, H);
            Lp=2*circumcenter(B, C, circumcenter(B, C, H))-circumcenter(B, C, H);
            NN=extension(H, Lp, B, C);
            P=foot(A, NN, H);
            N2=foot(A, EE, F);
            S1=intersectionpoint(circumcircle(A, B, C), T -- F);
            S2=2*N2-S1;

            filldraw(circumcircle(A, B, C), fil, pri);
            filldraw(circumcircle(B, C, F), sfil, sec);
            filldraw(circumcircle(A, EE, F), tfil, tri);
            filldraw(circumcircle(A, P, H), qfil, qua);
            filldraw(circumcircle(M, NN, P), qifil, qui);
            draw(A -- B -- C -- A, pri);
            draw(B -- T, pri);
            draw(A -- T -- S2, sec);
            draw(B -- EE, sec); draw(C -- F, sec); draw(A -- D, sec);
            draw(K -- J, qua);
            draw(P -- NN, qui);

            dot("$A$", A, dir(100));
            dot("$B$", B, SW);
            dot("$C$", C, SE);
            dot("$M$", M, SE);
            dot("$P$", K, SE);
            dot("$E$", EE, dir(80));
            dot("$F$", F, dir(110));
            dot("$H$", H, dir(30));
            dot("$D$", D, S);
            dot("$T$", T, SW);
            dot("$Q$", J, dir(190));
            dot("$S$", G, dir(10));
            dot("$Y$", NN, SW);
            dot("$Z$", P, N);
            dot("$N$", N2, dir(195));
            dot("$S_1$", S1, dir(150));
            dot("$S_2$", S2, NE);
        \end{asy}
    \end{center}
    Let $Q$ be the Miquel Point of $BCEF$, so that $Q$ lies on $\Gamma$, $(AEF)$, and $(AH)$, and also $Q=\overline{AT}\cap\overline{HP}$. Let $O$ denote the circumcenter of $\triangle ABC$, and let line $EF$ intersect $\Gamma$ at $S_1$ and $S_2$. Since $\overline{BC}$ and $\overline{EF}$ are antiparallel with respect to $\angle A$, $\overline{AO}\perp\overline{EF}$, so $N=\overline{AO}\cap\overline{EF}$ is the midpoint of $\overline{S_1S_2}$. Moreover, $ANSQ$ is cyclic, whence $TS\cdot TN=TQ\cdot TA=TS_1\cdot TS_2$, so by the Midpoint of Harmonic Bundles Lemma, $-1=(TS;S_1S_2)$. Hence, $S$ lies on the polar of $T$ with respect to $\Gamma$, and we are done.
\end{customsol}
\begin{customsol}{to part (b)}
    Let $M$ be the midpoint of $\overline{BC}$. By Ceva-Menelaus $(BC;DT)$ is harmonic, so by the Midpoint of Harmonic Bundles Lemma, $TA\cdot TQ=TB\cdot TC=TD\cdot TM$, whence $AQDM$ is cyclic. Then, \[\measuredangle QMY=\measuredangle QMD=\measuredangle QAD=\measuredangle QAH=\measuredangle QZH=\measuredangle QZY,\]
    so $MYQZ$ is cyclic. Note that $\triangle QFB\sim\triangle QEC$ and $\triangle HFB\sim\triangle HEC$, so \[\frac{HB}{HC}=\frac{FB}{EC}=\frac{QB}{QC}.\]
    Let the tangent to $\Gamma$ at $Q$ intersect $\overline{BC}$ at $R$. Since $\overline{HY}$ is a symmedian of $\triangle HBC$, $\overline{QY}$ is a symmedian of $\triangle QBC$, so $-1=(BC;YR)$. It follows that $RQ^2=RB\cdot RC=RY\cdot RM$, and we win.
\end{customsol}

