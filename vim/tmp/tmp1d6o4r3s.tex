% Input your problem and solution below.
% Three dashes on a newline indicate the breaking points.

---

Let $\triangle ABC$ be an isosceles triangle with $AB=AC=1$, and let $E$ and $F$ be the feet of the altitudes from $B$ and $C$ to sides $AC$ and $AB$, respectively. If line $EF$ is tangent to the incircle of $\triangle ABC$, compute the perimeter of $\triangle ABC$.

---

Let $x=\sin\tfrac A2$, and denote by $h_A$ and $r$ the length of the $A$-altitude and the inradius, respectively. Furthermore let $I$ denote the incenter and $\triangle DPQ$ the contact triangle of $\triangle ABC$.
\begin{center}
\begin{asy}
size(6cm);
defaultpen(fontsize(10pt));
real theta=2*aSin((sqrt(5)-1)/2);
pair A, B, C, D, I, EE, F, E1, F1;
A=dir(90); B=dir(270-theta); C=dir(270+theta);
D=foot(A, B, C); I=incenter(A, B, C); EE=foot(I, A, C); F=foot(I, A, B);
E1=foot(B, A, C); F1=foot(C, A, B);
draw(A -- B -- C -- A);
draw(incircle(A, B, C));
draw(A -- D); draw(EE -- I -- F, dotted);
draw(B -- E1, dashed); draw(C -- F1, dashed); draw(E1 -- F1);

dot("$A$", A, N);
dot("$B$", B, SW);
dot("$C$", C, SE);
dot("$D$", D, S);
dot("$E$", E1, NE);
dot("$F$", F1, NW);
dot("$I$", I, SW);
dot("$P$", EE, NE);
dot("$Q$", F, NW);
\end{asy}
\end{center}
Notice that since $\triangle AFE\sim\triangle ABC$ with scale factor $\cos A:1$, we have that $1-\cos A=\tfrac{2r}{h_A}$. Clearly $BD=DC=x$, so the semiperimeter of $\triangle ABC$ is $1+x$, and $AP=AQ=1-x$. Then, $r=AP\tan\tfrac A2=(1-x)\tan\tfrac A2$. Since $AD=\cos\tfrac A2$, \[2x^2=1-\cos A=\frac{2(1-x)\tan\tfrac A2}{\cos\tfrac A2}=\frac{2x(1-x)}{\cos^2\tfrac A2}=\frac{2x(1-x)}{1-x^2}.\]
Since $x\ne 0$, we can simplify this to $x=\tfrac1{1+x}$, so $x^2+x-1=0$. As $-1\le x\le 1$, $x=\tfrac{\sqrt5-1}2$, and the perimeter of $\triangle ABC$ is $2x+2=1+\sqrt5$, the answer.

---

$1+\sqrt5$

