% Input your problem and solution below.
% Three dashes on a newline indicate the breaking points.

---

Let $ABC$ be a triangle with $AB=AC$. The angle bisectors of $\angle CAB$ and $\angle ABC$ meet sides $BC$ and $CA$ at $D$ and $E$, respectively. Let $K$ be the incenter of triangle $ADC$. Suppose that $\angle BEK=45^\circ$. Find all possible values of $\angle CAB$.

---

The answer is $\angle A=60\dg$ and $\angle A=90\dg$.
\begin{center}
\begin{asy}
    size(10cm); defaultpen(fontsize(10pt));
    picture pic1, pic2;
    pair A,B,C,I,D,EE,K,F;
    A=(0,sqrt(3));
    B=(-1,0);
    C=(1,0);
    I=incenter(A,B,C);
    D=(B+C)/2;
    EE=extension(B,I,A,C);
    K=incenter(A,D,C);
    draw(pic1,A--D,gray);
    draw(pic1,B--EE,gray);
    draw(pic1,A--B--C--cycle);
    draw(pic1,incircle(A,D,C));
    draw(pic1,EE--K,Dotted);
    dot(pic1,"$A$",A,N);
    dot(pic1,"$B$",B,SW);
    dot(pic1,"$C$",C,S);
    dot(pic1,"$I$",I,dir(150));
    dot(pic1,"$D$",D,S);
    dot(pic1,"$E$",EE,dir(30));
    dot(pic1,"$K$",K,S);
    add(shift( (-1.5,0))*pic1);

    A=(0,sqrt(3));
    B=(-sqrt(3),0);
    C=(sqrt(3),0);
    I=incenter(A,B,C);
    D=(B+C)/2;
    EE=extension(B,I,A,C);
    K=incenter(A,D,C);
    F=foot(I,A,C);
    draw(pic2,A--D,gray);
    draw(pic2,B--EE,gray);
    draw(pic2,I--F,gray);
    draw(pic2,A--B--C--cycle);
    draw(pic2,circumcircle(I,F,K),dashed);
    draw(pic2,incircle(A,D,C));
    draw(pic2,EE--K,Dotted);
    dot(pic2,"$A$",A,N);
    dot(pic2,"$B$",B,S);
    dot(pic2,"$C$",C,SE);
    dot(pic2,"$I$",I,dir(146.25));
    dot(pic2,"$D$",D,S);
    dot(pic2,"$E$",EE,NE);
    dot(pic2,"$K$",K,S);
    dot(pic2,"$F$",F,NE);
    add(shift( (sqrt(3),0))*pic2);
\end{asy}
\end{center}
Let $I$ be the incenter of $\triangle ABC$ and let the incircle of $\triangle ABC$ touch $\seg{AC}$ at $F$. Then $CDIF$ is a kite, so it has an incircle; that is, $\seg{IF}$ is tangent to the incircle of $\triangle ADC$.

It is evident $\angle A=60\dg$ works; now assume $E\ne F$. Since $\seg{FK}$ bisects $\angle IFC$, we have $\da IFK=45\dg=\da IEK$, so $IFEK$ is cyclic. Then $\da BIC=\da EIK=\da EFK=135\dg$, so $\da BAC=90\dg$. Conversely if $\angle A=90\dg$, then $\angle BEK=45\dg$ analogously, so we are done.

