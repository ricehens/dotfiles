% Input your problem and solution below.
% Three dashes on a newline indicate the breaking points.

---

In $\triangle ABC$, the sides have integer lengths and $AB=AC$. Circle $\omega$ has its center at the incenter of $\triangle ABC$. An \emph{excircle} of $\triangle ABC$ is a circle in the exterior of $\triangle ABC$ that is tangent to one side of the triangle and tangent to the extensions of the other two sides. Suppose that the excircle tangent to $\overline{BC}$ is internally tangent to $\omega$, and the other two excircles are both externally tangent to $\omega$. Find the minimum possible value of the perimeter of $\triangle ABC$.

---

\begin{center}
    \begin{asy}
        size(8cm);
        defaultpen(fontsize(8pt));
        pair A, B, C, I, IA, IB, IC;
        A=(0, 4sqrt(5));
        B=(-1, 0);
        C=(1, 0);
        I=incenter(A, B, C);
        IA=2*circumcenter(I,B,C)-I;
        IB=2*circumcenter(I,C,A)-I;
        IC=2*circumcenter(I,A,B)-I;

        draw(B -- A -- C); draw(IB -- IC); draw(incircle(A, B, C));
        draw(foot(IB, B, C) -- foot(IC, B, C));
        draw(circle(IA, length(IA-foot(IA, B, C))));
        draw(arc(IB, IB-(4sqrt(5), 0), IB-(0, 4sqrt(5))));
        draw(arc(IC, IC-(0, 4sqrt(5)), IC+(4sqrt(5), 0)));
        draw(circle(I, 2/sqrt(5)+sqrt(5)));

        dot("$A$", A, N);
        dot("$B$", B, SW);
        dot("$C$", C, SE);
        dot("$I$", I, N);
        dot("$I_A$", IA, S);
        dot("$I_B$", IB, NE);
        dot("$I_C$", IC, NW);
    \end{asy}
\end{center}
First assume that $BC=2$ and $AB=AC=x$, and scale up later. Notice that $\overline{I_BAI_C}\parallel\overline{BC}$. Then, the height from $A$ is $\sqrt{x^2-1}$, so if $K=[ABC]$, we know $K=\sqrt{x^2-1}$. Then, if $r_D$ denotes the $D$-exradius for $D\in\{A,B,C\}$ and $s=x+1$ denotes the semiperimeter, \[r_A=\frac{K}{s-2}=\frac{K}{x-1},\;r_b=r_C=\frac{K}{s-x}=K,\text{ and }r=\frac{K}{s}=\frac{K}{x+1}.\]
Then, if $X$ denotes the tangency point between the $B$-excircle and $\overline{BC}$, it is known that $BX=s$, so $AI_B=s-1=x$. Furthermore, $AI=\sqrt{(s-2)^2+r^2}=\sqrt{(x-1)^2+(K/(x+1))^2}$. Then, \[r+2r_A=II_A=II_B-r_B.\]
It follows that
\begin{align*}
    II_B&=r+2r_A+r_B\\
    \sqrt{AI^2+AI_B^2}&=\frac{K}{x+1}+\frac{2K}{x-1}+K\\
    \sqrt{x^2+(x-1)^2+\left(\frac{K}{x+1}\right)^2}&=K\left(\frac1{x+1}+\frac2{x-1}+1\right)\\
    \frac{\sqrt{(x^2+(x-1)^2)(x+1)^2+x^2-1}}{x+1}&=K\left(\frac{x^2+3x}{x^2-1}\right)\\
    \frac{\sqrt{2x^3(x+1)}}{x+1}&=\frac{x(x+3)}{\sqrt{x^2-1}}\\
    2x(x-1)&=x^2+6x^2+9\\
    0&=x^2-8x-9\\
    &=(x+1)(x-9),
\end{align*}
whence $x=9$. Then, since $\gcd(2,9,9)=1$, the smallest possible perimeter is $2+9+9=20$.

---

020
