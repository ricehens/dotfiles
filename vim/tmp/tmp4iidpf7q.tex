% Input your problem and solution below.
% Three dashes on a newline indicate the breaking points.
% vim: tw=72

---

Let $H$ be the orthocenter and $G$ be the centroid of acute-angled triangle $ABC$ with $AB\ne AC$. The line $AG$ intersects the circumcircle of $ABC$ at $A$ and $P$. Let $P'$ be the reflection of $P$ in the line $BC$. Prove that $\angle CAB=60^\circ$ if and only if $HG=GP'$.

---

\begin{center}
    \begin{asy}
        size(8cm);
        defaultpen(fontsize(10pt));

        pen pri=deepgreen;
        pen sec=chartreuse;
        pen tri=springgreen;
        pen fil=pri+opacity(0.05);
        pen sfil=sec+opacity(0.05);
        pen tfil=tri+opacity(0.05);

        pair A=dir(130);
        pair B=dir(210);
        pair C=dir(330);
        pair O=(0, 0);
        pair Ap=-A;
        pair H=orthocenter(A, B, C);
        pair G=(A+B+C)/3;
        pair Hp=2*foot(A, B, C)-H;
        pair L=dir(270);
        pair M=(B+C)/2;
        pair P=intersectionpoint(M -- (M+100*(M-A)), circle(O, 1));
        pair D=2*foot(P, O, L)-P;
        pair Pp=2*foot(P, B, C)-P;

        draw(A -- B -- C -- A, pri);
        filldraw(circumcircle(A, B, C), fil, pri);
        filldraw(circle(L, 1), sfil, sec);
        draw(Hp -- A -- L, sec);
        draw(A -- D, pri);
        draw(A -- P -- Pp, sec);
        draw(G -- L, pri);
        draw(H -- Ap, tri);
        draw(H -- Pp, pri);
        draw(H -- O, tri);

        dot("$A$", A, A);
        dot("$B$", B, W);
        dot("$C$", C, E);
        dot("$H$", H, NW);
        dot("$G$", G, 2*dir(80));
        dot("$P'$", Pp, NE);
        dot("$H'$", Hp, SW);
        dot("$D$", D, D);
        dot("$L$", L, S);
        dot("$P$", P, P);
        dot("$A'$", Ap, SE);
        dot("$M$", M, NE);
        dot("$O$", O, N);

        clip((100, -1-Sin(10)) -- (-100, -1-Sin(10)) -- (-100, 100) -- (100, 100) -- cycle);
    \end{asy}
\end{center}
Let $M$ be the midpoint of $\overline{BC}$, $L$ be the reflection of $O$ over $\overline{BC}$, and $A'$ the antipode of $A$, so that $M$ is the midpoint of $\overline{HA'}$. Let the $A$-symmedian meet $(ABC)$ again at $D$. Since $\overline{BC}$ bisects $\angle AMD$, $D$ and $P$ are reflections across $\overline{OL}$. Note the following series of equivalences, in which all steps are reversible, whence our conclusion goes both ways.
\begin{itemize}
    \item $\angle CAB=60^\circ$, one side of the condition.
        \vspace{-0.75em}
    \item $GA=GL$, since $\angle A=60^\circ$ implies that $\overline{AL}$ is the perpendicular bisector of $\overline{OH}$.\footnote{This is well-known, but here is the proof. Let $K$ be the midpoint of arc $BAC$. Since $\angle BOC=2A=120^\circ$, $L$ is the midpoint of arc $BC$. Then, $\angle A=60^\circ$ yields $AH=2R\cos A=AO$, and since $H$ and $O$ are isogonal conjugates, $\overline{AL}$ must be the perpendicular bisector of $\overline{OH}$, on which $G$ lies.}
        \vspace{-0.75em}
    \item $\measuredangle DAL=\measuredangle LAG=\measuredangle GLA$, in which the first equality comes from $\overline{AD}$ and $\overline{AG}$ being isogonal.
        \vspace{-0.75em}
    \item $\overline{AD}\parallel\overline{GL}$, which is immediate.
        \vspace{-0.75em}
    \item $\overline{DA'}\perp\overline{GL}$, since $\angle ADA'=90^\circ$.
        \vspace{-0.75em}
    \item $\overline{HP'}\perp\overline{GL}$, by homothety at $M$ with scale factor $-1$.
        \vspace{-0.75em}
    \item $GH=GP'$, since $OH'=OP$ implies $LH=LP'$.
\end{itemize}
Hence, we are done. 

