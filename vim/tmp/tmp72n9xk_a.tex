% Input your problem and solution below.
% Three dashes on a newline indicate the breaking points.

---

Let $ABC$ be a triangle with circumcenter $O$, $A$-excenter $I_A$, $B$-excenter $I_B$, $C$-excenter $I_C$. The incircle of $\triangle ABC$ it tangent to sides $\seg{BC}$, $\seg{CA}$, $\seg{AB}$ at $D$, $E$, $F$ respectively. Lines $I_BE$ and $I_CF$ intersect at $P$. If the line through $O$ perpendicular to $\seg{OP}$ passes through $I_A$, prove that $\angle A=60\dg$.

---

This is a silly problem. Let $I$ be the incenter. In fact, $P$, $O$, $I$ are always collinear.
\begin{center}
\begin{asy}
    size(8cm); defaultpen(fontsize(10pt));

    pair O,A,B,C,I,D,EE,F,IA,IB,IC,P,V;
    O=(0,0);
    A=dir(130);
    B=dir(210);
    C=dir(330);
    I=incenter(A,B,C);
    D=foot(I,B,C);
    EE=foot(I,C,A);
    F=foot(I,A,B);
    IA=2*(2*foot(O,A,I)-A)-I;
    IB=2*(2*foot(O,B,I)-B)-I;
    IC=2*(2*foot(O,C,I)-C)-I;
    P=extension(IB,EE,IC,F);
    V=2O-I;

    draw(IB--P--IC,gray);
    draw(IA--P,gray);
    draw(circumcircle(B,I,C),gray);
    draw( (3P-2O)--(4O-3P),dashed);
    draw(incircle(A,B,C));
    draw(A--B--C--A);
    draw(IA--IB--IC--IA);
    draw(D--EE--F--D);

    dot("$A$",A,dir(110));
    dot("$B$",B,dir(190));
    dot("$C$",C,E);
    dot("$D$",D,dir(240));
    dot("$E$",EE,N);
    dot("$F$",F,dir(120));
    dot("$I_A$",IA,dir(285));
    dot("$I_B$",IB,NE);
    dot("$I_C$",IC,W);

    dot("$P$",P,N);
    dot("$O$",O,dir(100));
    dot("$I$",I,N);
    dot("$V$",V,N);
\end{asy}
\end{center}
To see this, note that $\seg{I_BI_C}$ and $\seg{EF}$ are both perpendicular to $\seg{AI}$; symmetrically applying this argument, $\triangle I_AI_BI_C$ and $\triangle DEF$ are homothetic at $P$, thus the Bevan point $V$ lies on line $IP$.

But $\triangle ABC$ is the orthic triangle of $\triangle I_AI_BI_C$, $I$ is the orthocenter, and $P$ is the circumcenter, so $O$ is the midpoint of $\seg{IV}$, and $P$, $I$, $O$, $V$ are collinear.

To finish, note that $BICI_A$ is cyclic with diameter $\seg{II_A}$, but if the problem hypothesis holds, then $\angle IOI_A=\angle POI_A=90\dg$, so $O\in(BICI_A)$. Finally $90+\angle A/2=\angle BIC=\angle BOC=2\angle A$, and $\angle A=60\dg$. End proof.

