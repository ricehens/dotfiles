% Input your problem and solution below.
% Three dashes on a newline indicate the breaking points.

---

In convex quadrilateral $KLMN$ side $\overline{MN}$ is perpendicular to diagonal $\overline{KM}$, side $\overline{KL}$ is perpendicular to diagonal $\overline{LN}$, $MN=65$, and $KL=28$. The line through $L$ perpendicular to side $\overline{KN}$ intersects diagonal $\overline{KM}$ at $O$ with $KO=8$. Find $MO$.

---

\begin{center}
    \begin{asy}
        size(8cm);
        pair K, L, M, NN, X, O;
        K=(-sqrt(98^2+65^2)/2, 0);
        NN=(sqrt(98^2+65^2)/2, 0);
        L=sqrt(98^2+65^2)/2*dir(180-2*aSin(28/sqrt(98^2+65^2)));
        M=sqrt(98^2+65^2)/2*dir(2*aSin(65/sqrt(98^2+65^2)));
        X=foot(L, K, NN);
        O=extension(L, X, K, M);
        draw(K -- L -- M -- NN -- K -- M); draw(L -- NN); draw(arc((K+NN)/2, NN, K));
        draw(L -- X, dashed); draw(arc((O+NN)/2, NN, X), dashed);

        draw(rightanglemark(K, L, NN, 100));
        draw(rightanglemark(K, M, NN, 100));
        draw(rightanglemark(L, X, NN, 100));
        dot("$K$", K, SW);
        dot("$L$", L, unit(L));
        dot("$M$", M, unit(M));
        dot("$N$", NN, SE);
        dot("$X$", X, S);
    \end{asy}
\end{center}
Notice that $KLMN$ is inscribed in the circle with diameter $\overline{KN}$ and $XOMN$ is inscribed in the circle with diameter $\overline{ON}$. Furthermore, $(XLN)$ is tangent to $\overline{KL}$. Then, \[KO\cdot KM=KX\cdot KN=KL^2\implies KM=\frac{28^2}{8}=98,\]
and $MO=KM-KO=90$.

---

090
