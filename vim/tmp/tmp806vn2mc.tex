% Input your problem and solution below.
% Three dashes on a newline indicate the breaking points.

---

Mathematical Chunks of Sentient Protoplasm (MCSPs, for short) are smart blobs who dream of merging together into one huge blob. But they can only do it following certain rules:
\begin{itemize}
    \item If two MCSPs have the same mass, or if their masses are $1$ apart, they can merge into a single MCSP, whose mass will be the sum of the original two.

    \item If an MCSP has even mass, it can split into two MCSPs, each with half the original mass.
\end{itemize}
Suppose we start with $n$ MCSPs, with masses $1$ through $n$. For what values of $n$ is there a finite sequence of steps that will allow all $n$ MCSPs to merge together into a single MCSP and achieve their dream of unity?

---

We consider the list of current MCSPs as a list of numbers, so that the original position is $(1,2,\ldots,n)$. The answer is $1$, $2$, $3$, $4$, $6$, achieved as follows:
\begin{itemize}[itemsep=0em]
    \item $(1)$ is already done;
    \item $(1,2)\to(3)$;
    \item $(1,2,3)\to(3,3)\to(6)$;
    \item $(1,2,3,4)\to(3,3,4)\to(3,3,2,2)\to(5,5)\to(10)$;
    \item $(1,2,3,4,5,6)\to(3,3,4,5,6)\to(3,3,2,2,5,6)\to(5,5,5,6)\to(10,11)\to(21)$.
\end{itemize}
Say an integer $m$ is \emph{$n$-tasty} if there exists an integer $k$ with $2^k(n-1)<m<2^k(n+1)$. The pith of this problem is the following claim.
\setcounter{iclaim}0
\begin{iclaim}
    For $n$ to satisfy the problem statement, $1+2+\cdots+n$ must be $t$-tasty for all odd $t\le n$.
\end{iclaim}
\begin{proof}
    Fix an odd integer $t$. We say an integer $j$ is \emph{palatable} if the MCSP of mass $t$ can eventually become a MCSP of mass $j$. In particular, if $T$ is palatable, then $2T-1$, $2T$, $2T+1$ are palatable, and if $T$ is even, then so is $T/2$.

    For some palatable $T$, consider the list of moves that turn the MCSP of mass $t$ into the MCSP of mass $T$. For instance, if $t=11$ and $T=45$, the list could be $11\to22\to45$ or $11\to23\to45$. If any number in the process is halved, we just reverse the previous step; that is, we can ignore the last two steps of any process of ending in the form $T\to2T\to T$.

    Hence a complete description of the set of palatable integers is as follows: (i) $t$ is palatable; and (ii) if $T$ is palatable, then so are $2T-1$, $2T$, $2T+1$.

    It follows (say, by induction) that all palatable integers are of the form $2^kt\pm j$, where $|j|<2^k$, so ``$t$-tasty'' and ``palatable'' are synonymous. In particular, for all the MCSPs to conglomerate into a single MCSP, $1+2+\cdots+n$ must be $t$-tasty for all odd $t\le n$. This proves the claim.
\end{proof}
\begin{iclaim}
    No positive integer is both $5$-tasty and $7$-tasty.
\end{iclaim}
\begin{proof}
    Assume $m$ is both $5$-tasty and $7$-tasty; then we can find $a$ and $b$ with
    \begin{align*}
        2^a(5-1)&<m<2^a(5+1),\\
        2^b(7-1)&<m<2^b(7+1).
    \end{align*}
    This rewrites to
    \begin{align*}
        2^{a+2}<m<3\cdot2^{a+1},\\
        3\cdot2^{b+1}<m<2^{b+3}.
    \end{align*}
    If $a\le b$, then $m<3\cdot2^{a+1}\le3\cdot2^{b+1}<m$, and if $a>b$, then $m>2^{a+2}\ge2^{b+3}>m$. Both cases are absurd.
\end{proof}

The two claims prove all $n\ge7$ fail. Finally $n=5$ fails by Claim 1, since $1+2+3+4+5=15$ is not $5$-tasty. This completes the proof.


