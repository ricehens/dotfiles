% Input your problem and solution below.
% Three dashes on a newline indicate the breaking points.

---

Let $P=A_1A_2\cdots A_k$ be a convex polygon in the plane. The vertices $A_1$, $A_2$, $\ldots$, $A_k$ have integral coordinates and lie on a circle. Let $S$ be the area of $P$. An odd positive integer $n$ is given such that the squares of the side lengths of $P$ are integers divisible by $n$. Prove that $2S$ is an integer divisible by $n$.

---

Induct on $k$: first we settle the base case $k=3$. Let $A_i=(x_i,y_i)$, and shift $A_1$ to the origin. Then by shoelace formula, $2S=x_2y^3-x_3y_2$. We are given
\begin{align*}
    0&\equiv(x_2-x_3)^2+(y_2-y_3)^2\\
    &\equiv\left(x_2^2+y_2^2\right)+\left(x_3^2+y_3^2\right)-2(x_2x_3+y_2y_3)\\
    &\equiv-2(x_2x_3+y_2y_3)\pmod n.
\end{align*}
Since $n$ is odd, $n\mid T:=x_2x_3+y_2y_3$. But $n^2\mid\left(x_2^2+y_2^2\right)\left(x_3^2+y_3^2\right)=T^2+4S^2$, so $n\mid2S$, as required.

It suffices to prove the problem for $n=p^e$, where $p$ is prime. I claim for $k\ge4$, there is a diagonal whose length squared is divisible by $p^e$. Then this diagonal dissects $P$ into two polygons with fewer sides, so we are done by strong induction.

Select $m$ so that $3\le m\le k-1$ and $\nu_p(A_1A_{m-1}^2)\ge\nu_p(A_1A_m^2)\le\nu_p(A_1A_{m+1}^2)$. If this is not possible, then either $2e\le\nu_p(A_1A_2^2)\le\nu_p(A_1A_3^2)$ or $\nu_p(A_1A_3^2)>\nu_p(A_1A_4^2)>\cdots>\nu_p(A_1A_k^2)\ge2e$. In either case, $n\mid A_1A_3^2$, and we are already done.

Let $A$ denote the set of algebraic integers. Let $E=e+\nu_p(A_1A_m)$, so $\nu_p(A_1A_{m-1}\cdot A_mA_{m+1})\ge E$ and $\nu_p(A_1A_{m+1}\cdot A_mA_{m-1})\ge E$. By Ptolemy's theorem on $A_1A_{m-1}A_mA_{m+1}$, we have \[\sqrt{\frac{A_1A_m^2\cdot A_{m-1}A_{m+1}^2}{p^{2E}}}=\sqrt{\frac{A_1A_{m-1}^2\cdot A_mA_{m+1}^2}{p^{2E}}}+\sqrt{\frac{A_1A_{m+1}^2\cdot A_mA_{m-1}^2}{p^{2E}}}\in A.\]
Hence $p^{2E}\mid A_1A_m^2\cdot A_{m-1}A_{m+1}^2$, so either $p^E\mid A_1A_m^2$ or $p^E\mid A_{m-1}A_{m+1}^2$. This completes the inductive step.

