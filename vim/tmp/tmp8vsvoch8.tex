% Input your problem and solution below.
% Three dashes on a newline indicate the breaking points.

---

Let $ABC$ be a right triangle with $\angle B=90\dg$, and let $M$ be the midpoint of $\seg{AC}$. The incircle of triangle $ABM$ touches sides $AB$ and $AM$ at points $A_1$ and $A_2$; points $C_1$, $C_2$ are defined similarly. Prove that lines $A_1A_2$ and $C_1C_2$ meet on the bisector of $\angle ABC$.

---

\begin{center}
    \begin{asy}
        size(6cm); defaultpen(fontsize(10pt));

        pair M,A,B,C,I1,I2,A1,A2,C1,C2,X;
        M=(0,0);
        A=dir(145);
        B=reflect(M,M+(1,0))*A;
        C=-A;
        I1=incenter(A,B,M);
        I2=incenter(B,C,M);
        A1=foot(I1,A,B);
        A2=foot(I1,A,C);
        C1=foot(I2,B,C);
        C2=foot(I2,A,C);
        X=extension(A1,A2,C1,C2);

        draw(incircle(A,B,M),gray);
        draw(incircle(B,C,M),gray);
        draw(B--X,gray);
        draw(A1--C1,gray);
        draw(B--C--A--B--M);
        draw(A1--X--C1);

        dot("$A$",A,NW);
        dot("$B$",B,SW);
        dot("$C$",C,SE);
        dot("$A_1$",A1,W);
        dot("$A_2$",A2,N);
        dot("$C_1$",C1,S);
        dot("$C_2$",C2,dir(30));
        dot("$M$",M,N);
        dot("$X$",X,NE);
    \end{asy}
\end{center}
The key observation is that $\seg{A_1X}$ is the angle bisector of $\angle AA_1C_1$. Symmetrically applying this argument, $X$ is the $B$-excenter of $\triangle A_1BC_1$, from which the result is obvious.

To show this, note that since $M$ is the circumcenter of $\triangle ABC$, $\triangle MAB$ is isosceles, so $A_1$ is the midpoint of $\seg{AB}$. It follows that $2\da AA_1A_2=\da A_1AA_2=\da AA_1C_1$, as desired.

