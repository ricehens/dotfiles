% Input your problem and solution below.
% Three dashes on a newline indicate the breaking points.

---

Triangle $ABC$ has circumcircle $\Gamma$. A circle with center $O$ is tangent to $BC$ at $P$ and internally to $\Gamma$ at $Q$, so that $Q$ lies on arc $BC$ of $\Gamma$ not containing $A$. Prove that if $\angle BAO=\angle CAO$ then $\angle PAO=\angle QAO$.

---

\begin{center}
    \begin{asy}
        size(10cm);
        defaultpen(fontsize(10pt));
        pen pri=purple+linewidth(0.5);
        pen sec=deepmagenta+linewidth(0.5);
        pen tri=pink+linewidth(0.5);
        pen fil=purple+opacity(0.05);
        pen sfil=deepmagenta+opacity(0.05);
        pen tfil=pink+opacity(0.05);

        pair O, P, Q, A, B, C, M, NN;
        O=2/3*dir(255);
        P=O+(0, 1/3);
        Q=dir(255);
        M=dir(90);
        NN=dir(270);
        A=intersectionpoints(((O+NN)/2) -- (NN+(O-NN)*100), circle((0, 0), 1))[0];
        B=intersectionpoints(P -- (P - (100, 0)), circle((0, 0), 1))[0];
        C=intersectionpoints(P -- (P + (100, 0)), circle((0, 0), 1))[0];

        draw(A -- B -- C -- A, pri);
        draw(P -- A -- Q -- O -- P, sec);
        draw(A -- NN -- M, sec+dashed);
        draw(Q -- M, sec+dotted);
        filldraw(circle((0, 0), 1), fil, pri);
        filldraw(circle(O, 1/3), sfil, sec);

        dot("$A$", A, A);
        dot("$B$", B, B);
        dot("$C$", C, C);
        dot("$O$", O, NE);
        dot("$P$", P, N);
        dot("$Q$", Q, Q);
        dot("$M$", M, N);
        dot("$N$", NN, S);
    \end{asy}
\end{center}
Let $I$ denote the center of $\Gamma$, and let $\seg{AO}$ and $\seg{PQ}$ intersect $\Gamma$ again at $N$ and $M$ respectively. Since $\seg{AO}$ bisects $\angle BAC$, $N$ is the midpoint of the arc $BC$ not containing $A$.

Note that $(O)$ and $\Gamma$ are homothetic at $Q$, so $M$ is the midpoint of arc $\widehat{BAC}$. Furthermore, \[\da QPO=\da QMI=\da QMN=\da QAN=\da QAO,\]
whence $APOQ$ is cyclic. Since $OP=OQ$, $\seg{AO}$ bisects $\angle PAQ$, as desired.

