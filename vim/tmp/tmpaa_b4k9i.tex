% Input your problem and solution below.
% Three dashes on a newline indicate the breaking points.

---

Let $ABC$ be a triangle with $AB=5$, $BC=4$, and $CA=3$. On each side of $\triangle ABC$, externally erect a semicircle whose diameter is the corresponding side. Let $X$ be on the semicircular arc erected on side $BC$ such that $\angle CBX$ has measure $15^\circ$. Let $Y$ be on the semicircular arc erected on side $CA$ such that $\angle ACY$ has measure $15^\circ$. Similarly, let $Z$ be on the semicircular arc erected on side $AB$ such that $\angle BAZ$ has measure $15^\circ$. What is the area of triangle $XYZ$? Express your answer as a fraction in simplest form.

---

\begin{center}
    \begin{asy}
        size(6cm);
        defaultpen(fontsize(10pt));
        pen fil=opacity(0.05);
        pair A,B,C,M,NN,P,X,Y,Z;
        A=(0,3);
        B=(4,0);
        C=(0,0);
        M=(B+C)/2;
        NN=(C+A)/2;
        P=(A+B)/2;
        X=M+2*dir(210);
        Y=NN+3/2*dir(120);
        Z=P+rotate(30)*(B-P);

        fill(A--B--C--cycle,purple+fil);
        fill(arc(M,C,B)--cycle,red+fil);
        fill(arc(NN,A,C)--cycle,red+fil);
        fill(arc(P,B,A)--cycle,red+fil);
        fill(A--C--Z--cycle,springgreen+fil);
        fill(B--C--Z--cycle,springgreen+fil);
        fill(A--Z--B--cycle,royalblue+fil);
        draw(A--Y,deepcyan+linewidth(1));
        draw(B--X,deepcyan+linewidth(1));
        filldraw(X--Y--Z--cycle,cyan+fil,Cyan+linewidth(1));
        draw(C--Z,deepcyan+linewidth(1));
        draw(A--B--C--A--Z--B);
        draw(arc(M,C,B));
        draw(arc(NN,A,C));
        draw(arc(P,B,A));

        dot("$A$",A,NW);
        dot("$B$",B,SE);
        dot("$C$",C,SW);
        dot("$X$",X,SW);
        dot("$Y$",Y,NW);
        dot("$Z$",Z,E);
    \end{asy}
\end{center}
This problem relies on a few synthetic observations.
\begin{iclaim}
    $\angle ZCX=\angle ZCY=90^\circ$. In particular, $X$, $C$, $Y$ are collinear.
\end{iclaim}
\begin{proof}
    By Thale's Theorem, $ACBZ$ is cyclic with diameter $\seg{AB}$. Now, $\angle ZCB=\angle ZAB=15^\circ$ and $\angle BCX=90^\circ-\angle XBC=75^\circ$, so $\angle ZCX=15^\circ+75^\circ=90^\circ$. By symmetry, $\angle ZCY=90^\circ$, as desired.
\end{proof}
\begin{iclaim}
    $\seg{AY}\parallel\seg{BX}\parallel\seg{CZ}$.
\end{iclaim}
\begin{proof}
    We show that $\seg{CZ}\parallel\seg{BX}$, and the other case follows from symmetry. Notice again that $\angle ZCB=\angle ZAB=15^\circ=\angle XBC$, and thus we are done by congruent alternate interior angles.
\end{proof}

Finally since $B$ and $X$ are equidistant from $\seg{CZ}$, $[CZX]=[CZB]$ and similarly $[CYZ]=[CAZ]$, and thus $$[XYZ]=[CYZ]+[CZX]=[CAZ]+[CZB]=[CAZB]=[ABC]+[AZB].$$
Now $[ABC]=\tfrac12\cdot3\cdot4=6$, and $$[AZB]=\frac12AZ\cdot ZB=\frac12(5\cos15^\circ)(5\sin15^\circ)=\frac{25}4\sin30^\circ=\frac{25}8,$$
and so the answer is $6+25/8=\boxed{73/8}$, and we're done.

---

$73/8$
