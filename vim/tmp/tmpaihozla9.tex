% Input your problem and solution below.
% Three dashes on a newline indicate the breaking points.

---

For a positive integer $n\ge3$, plot $n$ equally spaced points around a circle. Label one of them $A$, and place a marker at $A$. One may move the marker forward in a clockwise direction to either the next point or the point after that. Hence there are a total of $2n$ distinct moves available; two from each point. Let $a_n$ count the number of ways to advance around the circle exactly twice, beginning and ending at $A$, without repeating a move. Prove that $a_{n-1}+a_n=2^n$ for all $n\ge4$.

---

Let $B$ be the point directly counterclockwise to $A$.
\begin{claim*}
    We land on each point either once or twice.
\end{claim*}
\begin{proof}
    If we never land on a point, then we jump over it twice.
\end{proof}

Consider a circle with $n$ points and a marker labeled $A$, and assign each point either the number $1$ or the number $2$; this may be done in $2^n$ ways. Then proceed around the circle starting from $A$, such that the first time the marker reaches a certain point labeled $i$, it moves $i$ points forward. (If it reaches a point labeled $i$ a second time, its only option is to move forward $3-i$ points.)
\begin{itemize}
    \item The paths where we reach the point $A$ the second time around biject with the $a_n$ ways to advance around a circle with $n$ points.
    \item If we do not reach the point $A$ the second time around, then we must have jumped over $A$ from $B$. Just delete the point $B$ completely; it is easy to see these paths bijects with the $a_{n-1}$ ways to advance around a circle with $n-1$ points.
\end{itemize}
Hence $a_n+a_{n-1}=2^n$ as needed.

