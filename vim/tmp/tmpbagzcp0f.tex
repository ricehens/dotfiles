% Input your problem and solution below.
% Three dashes on a newline indicate the breaking points.

---

Call a positive integer \textit{palatable} if when expressed in binary, each contiguous block of zeros that is not a subsequence of another contiguous block of zeros has even length, and each contiguous block of ones that is not a subsequence of another contiguous block of ones has odd length. For example, $57=111001_2$ is palatable while $69=1000101_2$ is not. Find the number of palatable positive integers $N$ such that $2^{18}<N<2^{19}$.

---

Let $a_k$ be the number of palatable integers with $k$ digits in base $2$ that end in $0$, and $b_k$ the number that end in $1$. If the number $T$ ends in $0$ in binary, we can append two $0$'s or one $1$ to form another palatable integer, while if $T$ ends in $1$ in binary, we can append two $0$'s or two $1$'s to form another. Hence, $a_k=a_{k-2}+b_{k-2}$ and $b_k=a_{k-1}+b_{k-2}$. It is easy to check that $a_1=0$, $a_2=0$, $b_1=1$, and $b_2=0$. Then,
\begin{center}
    \begin{tabular}{c|c|c}
        $k$ & $a_k$ & $b_k$ \\ \hline
        1 & 0 & 1 \\
        2 & 0 & 0 \\
        3 & 1 & 1 \\
        4 & 0 & 1 \\
        5 & 2 & 1 \\
        6 & 1 & 3 \\
        7 & 3 & 2 \\
        8 & 4 & 6 \\
        9 & 5 & 6 \\
        10 & 10 & 11 \\
        11 & 11 & 16 \\
        12 & 21 & 22 \\
        13 & 27 & 37 \\
        14 & 43 & 49 \\
        15 & 64 & 80 \\
        16 & 92 & 113 \\
        17 & 144 & 172 \\
        18 & 205 & 257 \\
        19 & 316 & 377
    \end{tabular}
\end{center}
Since $2^{18}$ is palatable, the answer is $a_{19}+b_{19}-1=692$.

---

692
