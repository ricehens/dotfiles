% Input your problem and solution below.
% Three dashes on a newline indicate the breaking points.

---

Do there exist real numbers $a_0,a_1,\ldots,a_{2018}$, with $a_0\ne 0$, such that the roots of the polynomial $P(x)=x^{2019}+a_{2018}x^{2018}+\cdots+a_1x+a_0$ are $a_0,a_1,\ldots,a_{2018}$?

---

I claim that no such polynomial exists. We will prove a stronger result: That no such polynomial exists with $\deg P=n\ge 6$.

Suppose there exist $a_0,a_1,\cdots,a_{n-1}$ satisfying the problem statement. By Vieta's, we have that \[\sum_{i=0}^{n-1}a_i=-a_{n-1};\qquad\sum_{i<j}a_ia_j=a_{n-2};\qquad\left|\prod\limits_{i=0}^{n-1} a_i\right| =a_0\Rightarrow \left|\prod_{i=1}^{n-1} a_i\right|=1 \text{ since }a_0\neq 0.\]
Since
\begin{equation}\label{*}
    \sum\limits_{i=0}^{n-1}a_i^2=\left(\sum\limits_{i=0}^{n-1}a_i\right)^2-2\sum\limits_{i<j}a_ia_j=a_{n-1}^2-2a_{n-2},\tag{$\star$}
\end{equation}
subtracting $a_{n-1}^2+a_{n-2}^2$ from both sides we get \[\sum\limits_{i=0}^{n-3}a_i^2=-a_{n-2}^2-2a_{n-2}=1-(a_{n-2}+1)^2.\]
By the trivial inequality, \[\sum\limits_{i=0}^{n-3}a_i^2=a_0^2+\sum\limits_{i=1}^{n-3}a_i^2\geq a_0^2+0>0\] since $a_0\neq 0,$ hence we have that $1-(a_{n-2}+1)^2>0 \Rightarrow -2<a_{n-2}<0.$ Also, for the same reason, $1-(a_{n-2}+1)^2\leq 1-0=1,$ hence we have that \[\sum\limits_{i=0}^{n-3}a_i^2\leq 1\Rightarrow 0<|a_0|\leq 1, |a_i|<1\] for $i=1,\cdots, n-3.$

We can now see that $0<-2a_{n-2}<4$ and $0<1-(a_{n-2}+1)^2\leq 1.$ By AM-GM, \small{\[\sqrt[n-3]{\prod\limits_{i=1}^{n-3}a_i^2}\leq \frac{\sum\limits_{i=1}^{n-3}a_i^2}{n-3}<\frac{\sum\limits_{i=0}^{n-3}a_i^2}{n-3}=\frac{1-(a_{n-2}+1)^2}{n-3}\Rightarrow \left|\prod\limits_{i=1}^{n-3} a_i\right|<\left[\frac{1-(a_{n-2}+1)^2}{n-3}\right]^{\frac{n-3}{2}}\leq \left(\frac{1}{n-3}\right)^{\frac{n-3}{2}}.\]}
By the special case of Cauchy-Schwarz inequality $\left(\sum\limits_{i=1}^n t_i\right)^2\leq n \sum\limits_{i=1}^n t_i^2,$ we get that \small{\[\frac{\sum\limits_{i=0}^{n-2}|a_i|}{n-1}\leq \sqrt{\frac{\sum\limits_{i=0}^{n-2}a_i^2}{n-1}}=\sqrt{\frac{-2a_{n-2}}{n-1}} \text { by \eqref{*}} \Rightarrow \sum\limits_{i=0}^{n-2}|a_i|\leq \sqrt{-2a_{n-2}(n-1)}<\sqrt{4(n-1)}=2\sqrt{n-1}.\]}
Coming back to the original Vieta's relationships we wrote at the beginning, we can now see that \[\sum\limits_{i=0}^{n-1}a_i=-a_{n-1} \Rightarrow |a_{n-1}|=\frac{\left|\sum\limits_{i=0}^{n-2}a_i\right|}{2}<\frac{\sum\limits_{i=0}^{n-2}|a_i|}{2}<\frac{2\sqrt{n-1}}{2}=\sqrt{n-1}.\]Also$,$ since $\left|\prod\limits_{i=1}^{n-1} a_i\right|=1$ and $|a_{n-2}|<2,$ we have that \small{\[1=|a_{n-1}||a_{n-2}|\left|\prod\limits_{i=1}^{n-3} a_i\right|<
\sqrt{n-1}\text{ } \cdot \text{ }2\text{ }\cdot\text{ } \left(\frac{1}{n-3}\right)^{\frac{n-3}{2}}=2\text{ }\sqrt{\frac{n-1}{(n-3)^{n-3}}}\Rightarrow (n-3)^{n-3}<4(n-1)\]}
which is true only if $n<6$, and we win.
