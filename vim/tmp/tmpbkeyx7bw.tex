% Input your problem and solution below.
% Three dashes on a newline indicate the breaking points.
% vim: tw=72

---

Let $\mathbf{Z}$ denote the set of all integers. Find all real numbers $c>0$ such that there exists a labeling of the lattice points $(x,y)\in\mathbf{Z}^2$ with positive integers for which:
\begin{itemize}[itemsep=0em]
    \item only finitely many distinct labels occur, and
    \item for each label $i$, the distance between any two points labeled $i$ is at least $c^i$.
\end{itemize}

---

The answer is $c<\sqrt2$. We first show that a construction exists if $c<\sqrt2$. Suppose that $k$ is the smallest integer such that $c^k<(\sqrt2)^{k-1}$. Toss on the complex plane, and denote \[S_1=\{z\mid\text{Re}(z)+\text{Im}(z)\equiv 1\hspace{-0.75em}\pmod2\}.\]
Now, let \[S_t=\{z(1+i)\mid z\in S_{t-1}\}\]
for all $1<t<k$. Label all points in $S_t$ the label $t$, and label the rest of the points the label $k$. It is easy to see that each point is in exactly one of $S_1,S_2,\ldots,S_k$, so this labeling works.

Henceforth assume that $c=\sqrt2$. We will show that no labeling exists. First we prove a lemma.
\begin{boxlemma*}
    It is impossible to place four points in the interior of a unit square such that any two points are a distance of at least $1$ apart.
\end{boxlemma*}
\begin{proof}
    Consider any four points in the interior of the unit square, and let $O$ be the center of the unit square. By the Pigeonhole Principle there are two points $P$ and $Q$ such that $\angle POQ\le90^\circ$. Then $PQ^2\le OP^2+OQ^2<1$, as desired.
\end{proof}

Now we claim the following, which obviously implies the desired result.
\begin{iclaim*}
    Any square of size $2^n\times2^n$ must contain a point with label greater than $2n$.
\end{iclaim*}
We use induction. It is easy to verify the claim for $n=1$. Now suppose the claim is true for $n-1$, and assume for the sake of contradiction that it is possible to label the squares of a $2^n\times2^n$ such that no label exceeds $2n$.

First we claim that there can be at most $3$ squares labeled $2n$. Indeed this follows immediately from the lemma. By the Pigeonhole Principle, one of the four nonoverlapping subgrids with dimensions $2^{n-1}\times2^{n-1}$ cannot contain a point labeled $2n$. Assume without loss of generality it is the top-left square. By the inductive hypothesis, it must contain a point labeled $2n-1$.

Consider the grid of dimensions $2^{n-1}\times2^{n-1}$ directly to the right of the point labeled $2n-1$, with top row coinciding with the top row of the $2^n\times2^n$ grid. Clearly this grid cannot contain a $2n-1$, so it contains a $2n$. Then it is easy to construct a grid of dimensions $2^{n-1}\times2^{n-1}$ directly adjacent to both the square labeled $2n-1$ and the square labeled $2n$. But this grid contains neither a $2n-1$ nor a $2n$, a contradiction.

Hence, $c=\sqrt2$ is impossible, so it is immediate that any $c$ obeying $c\ge\sqrt2$ does not work. Thus the answer is $c<\sqrt2$, and we are done.
