% Input your problem and solution below.
% Three dashes on a newline indicate the breaking points.

---

Consider solutions to the equation
\[x^2-cx+1=\frac{f(x)}{g(x)},\]
where $f$ and $g$ are polynomials with nonnegative real coefficients. For each $c>0$, determine the minimum possible degree of $f$, or show that no such $f$, $g$ exist.

---

If $c\ge2$, we would have $f(1)/g(1)\le0$, which is absurd, so no $f$, $g$ exist. For $0<c<2$, let $c=2\cos\theta$ for $0<\theta<\pi/2$. The answer is $\lceil\pi/\theta\rceil$. In what follows, we let
\begin{align*}
    f(x)&=a_0+a_1x+a_2x^2+\cdots+a_nx^n,\\
    g(x)&=b_0+b_1x+b_2x^2+\cdots+b_{n-2}x^{n-2}.
\end{align*}

\bigskip

\textbf{Proof that $n\ge\pi/\theta$:} Observe that $e^{i\theta}$ is a root of $x^2-(2\cos\theta)x+1$, so $e^{i\theta}$ is also a root of $f$. Taking the imaginary part,
\[0=\opname{Im}\left(e^{i\theta}\right)=\sum_{k=1}^na_k\sin(k\theta).\]
(The summation starts at $k=1$ since $a_k\sin(k\theta)=0$ at $k=0$.)

But note that if $n<\pi/\theta$, then $\sin(k\theta)>0$ for $1\le k\le n$, which is absurd unless $n=0$. In that case, $f$ is constant, contradiction.

\bigskip

\textbf{Construction for $n=\left\lceil\pi/\theta\right\rceil$:} With this choice of $n$, we have $\sin(k\theta)>0$ for $1\le k\le n-1$ and $\sin(n\theta)\le0$. I claim that selecting $b_k=\sin( (k+1)\theta)$ for $0\le k\le n-2$ works. (Note that all these $b_k$ are nonnegative.)

For convenience, let $b_j=0$ for $j<0$ and $j>n-2$. Observe that $a_{k+1}=b_{k+1}+b_{k-1}-(2\cos\theta)b_k$, so for $f$ to have nonnegative coefficients, it is sufficient to verify
\[b_{k+1}+b_{k-1}\ge(2\cos\theta)b_k\quad\text{for all }0\le k\le n-2.\]
Indeed, most of these are sharp: the three statements below verify the hypothesis for $k=0$, $1\le k\le n-3$, and $k=n-2$ respectively:
\begin{align*}
    (2\cos\theta)b_0&=2\cos\theta\sin\theta=\sin2\theta=b_1;\\
    (2\cos\theta)b_k&=2\cos\theta\sin( (k+1)\theta)=\sin( (k+2)\theta)+\sin(k\theta)=b_{k+1}+b_{k-1};\\
    (2\cos\theta)b_{n-2}&=2\cos\theta\sin( (n-1)\theta)=\sin( (n-2)\theta)+\sin(n\theta)\le\sin( (n-2)\theta)=b_{n-3}.
\end{align*}

In conclusion,
\[\Big[x^2-(2\cos\theta)x+1\Big]\cdot\left[\sum_{k=0}^{n-2}\sin( (k+1)\theta)\cdot x^k\right]=\sin(\theta)-\sin(n\theta)\cdot x^{n-1}+\sin( (n-1)\theta)\cdot x^n,\]
as desired.
\begin{remark}
    I will show how the construction is derived: just as we did in the solution above, working forwards, we reduce the construction to finding a sequence of nonnegative reals $b_0$, \ldots, $b_{n-2}$ (with $b_j=0$ for all other $j$) such that
    \[b_{k+1}+b_{k-1}\ge(2\cos\theta)b_k\quad\text{for all }0\le k\le n-2.\]
    Now we can freely choose $b_{k+1}\ge(2\cos\theta)b_k-b_{k-1}$ at each $k$ until $k=n-2$, where the key requirement is that $b_{n-3}\ge(2\cos\theta)b_{n-2}$. This sort-of means we want to keep the sequence small, so let's force equality to hold for $1\le k\le n-1$:

    Therefore, our sequence obeys the recurrence $b_{k+1}=(2\cos\theta)b_k-b_{k-1}$ for $1\le k\le n-2$. The characteristic polynomial is the familiar $x^2-(2\cos\theta)x+1$ with roots $e^{i\theta}$ and $e^{-i\theta}$, so $b_k$ is a linear combination of $e^{i\theta k}$ and $e^{-i\theta k}$.

    The easy-to-think-of $b_k$ are $\cos k\theta$ and $\sin k\theta$. The former obviously fails, but the latter \emph{almost} works. At the end, we reduce $b_{n-3}\ge(2\cos\theta)b_{n-2}$ to
    \[\sin( (n-3)\theta)\stackrel?\ge(2\cos\theta)\sin( (n-2)\theta)=\sin( (n-3)\theta)+\sin( (n-1)\theta),\]
    i.e.\ $\sin( (n-1)\theta)\le0$, which is almost true but still false.

    So we really just want to shift the indices. But note that $\sin( (k+1)\theta)$ is also a linear combination of $e^{i\theta k}$ and $e^{-i\theta k}$, and it works! Amusingly enough, equality holds at $k=0$ now: $b_1\ge(2\cos\theta)b_0$.
\end{remark}

