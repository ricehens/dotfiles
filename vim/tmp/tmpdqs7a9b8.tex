% Input your problem and solution below.
% Three dashes on a newline indicate the breaking points.

---

Let $ABC$ be an scalene triangle with circumcircle $\Gamma$ and orthocenter $H$, and let $K$, $M$ be the midpoints of $\seg{AH}$, $\seg{BC}$, respectively. Line $AH$ intersects $\Gamma$ again at $T$, and line $KM$ intersects $\Gamma$ at $U$, $V$. Lines $TU$, $TV$ intersect lines $AB$, $AC$ at $X$, $Y$, respectively, and $W$ lies on line $KM$ such that $\seg{AW}\perp\seg{HM}$. If $Z$ is the reflection of $A$ over $W$, show that $X$, $Y$, $Z$ are collinear.

---

By symmetry in $U$, $V$, the clear interpretation of the problem is as follows:
\begin{quote}
    Let $ABC$ be a triangle with circumcircle $\Gamma$, orthocenter $H$, and let $K$, $M$ be the midpoints of $\seg{AH}$, $\seg{BC}$. Line $KM$ intersects $\Gamma$ at $U$, $V$. Let $\seg{TU}$ intersect $\seg{AB}$, $\seg{AC}$ at $X_1$, $X_2$ and let $\seg{TV}$ intersect $\seg{AB}$, $\seg{AC}$ at $Y_1$, $Y_2$. Finally, $Z=\seg{X_1Y_2}\cap\seg{X_2Y_1}$.
    Prove that (i) $\seg{AZ}\perp\seg{HM}$ and that (ii) line $KM$ bisects $\seg{AZ}$.
\end{quote}
In what follows, $DEF$ is the orthic triangle, and $Q$ lies on $\Gamma$ so that $\angle AQH=90\dg$. It follows that $Q$ lies on $\seg{HM}$ and $(AEF)$.
\begin{center}
\begin{asy}
    size(9cm); defaultpen(fontsize(10pt));
    pen pri=red;
    pen sec=purple+pink;
    pen tri=orange;
    pen qua=fuchsia;
    pen fil=red+opacity(0.05);
    pen sfil=purple+pink+opacity(0.05);
    pen tfil=yellow+opacity(0.05);
    pen qfil=fuchsia+opacity(0.05);

    pair A,B,C,H,T,D,EE,F,K,M,Q,U,V,X1,X2,Y1,Y2,Z;
    A=dir(110);
    B=dir(220);
    C=dir(320);
    H=A+B+C;
    T=reflect(B,C)*H;
    D=foot(A,B,C);
    EE=foot(B,C,A);
    F=foot(C,A,B);
    K=(A+H)/2;
    M=(B+C)/2;
    Q=foot(A,H,M);
    U=intersectionpoint(unitcircle,K--(100K-99M));
    V=2*foot(origin,K,M)-U;
    X1=extension(A,B,T,U);
    X2=extension(A,C,T,U);
    Y1=extension(A,B,T,V);
    Y2=extension(A,C,T,V);
    Z=extension(X1,Y2,X2,Y1);

    draw(A--Q--M,qua);
    draw(Z--Y2,tri);
    draw(X2--Y1,tri);
    filldraw(circumcircle(A,Q,X1),tfil,tri+dashed);
    filldraw(circumcircle(A,Q,Y1),tfil,tri+dashed);
    draw(Y1--Y2,sec);
    draw(X2--T,sec);
    filldraw(circumcircle(A,EE,F),sfil,sec);
    filldraw(circumcircle(D,EE,F),fil,pri);
    filldraw(unitcircle,fil,pri);
    filldraw(A--B--C--cycle,fil,pri);
    draw(X2--A,pri); draw(Y1--B,pri); draw(Y2--C,pri);
    draw(A--T,pri+Dotted);
    draw(U--V,pri);

    dot("$A$",A,dir(75));
    dot("$B$",B,dir(195));
    dot("$C$",C,dir(0));
    dot("$H$",H,dir(45));
    dot("$T$",T,dir(270));
    dot("$D$",D,dir(300));
    dot("$E$",EE,dir(10));
    dot("$F$",F,dir(210));
    dot("$K$",K,dir(60));
    dot("$M$",M,dir(250));
    dot("$Q$",Q,dir(220));
    dot("$U$",U,dir(135)/2);
    dot("$V$",V,dir(270));
    dot("$X_1$",X1,dir(210));
    dot("$X_2$",X2,dir(80));
    dot("$Y_1$",Y1,dir(225));
    dot("$Y_2$",Y2,dir(315));
    dot("$Z$",Z,dir(135)/2);
\end{asy}
\end{center}
\paragraph{Proof of (i)} Since $\seg{AQ}\perp\seg{HM}$, it suffices to prove $A$, $Z$, $Q$ collinear.

Recall that $-1=(BC;TQ)$, say by noting that \[\frac{QB}{QC}=\frac{BF}{CE}=\frac{BH}{CH}=\frac{BT}{CT}.\]
By Ceva-Menelaus in $\triangle X_2Y_1Y_2$, we have \[-1=(Y_1,Y_2;T,\seg{AZ}\cap\seg{Y_1Y_2})\stackrel A=(B,C;T,\seg{AZ}\cap\Gamma),\]
as needed.

\paragraph{Proof of (ii)} Observe that $Q$ is the spiral center between $\seg{BC}$, $\seg{EF}$. The key is the following claim:
\begin{claim*}
    A spiral similarity sends $\seg{BC}\cup M$ to $\seg{X_1X_2}\cup U$.
\end{claim*}
\begin{proof}
    Consider the (unique) points $X_1'$, $X_2'$ on $\seg{AB}$, $\seg{AC}$ such that $U\in\seg{X_1'X_2'}$ and a spiral similarity at $Q$ sends $\seg{BC}$ to $\seg{X_1'X_2'}$. By gliding principle, the midpoints of $\seg{BC}$, $\seg{EF}$, $\seg{X_1'X_2'}$ are collinear; i.e.\ $U$ is the midpoint of $\seg{X_1'X_2'}$.

    It will suffice to show $X_1=X_1'$ and $X_2=X_2'$, i.e.\ $T\in\seg{X_1'X_2'}$. Recall that $BMEQ$ is cyclic by properties of the Miquel point, so \[\da QUX_1'=\da QMB=\da QEB=\da QEH=\da QAH=\da QAT=\da QUT,\]
    and the claim follows.
\end{proof}

Finally, $U$ is the midpoint of $\seg{X_1X_2}$ and $V$ is the midpoint of $\seg{Y_1Y_2}$, so $\seg{UV}$ is the Gauss line of $X_1Y_1X_2Y_2$. It passes through the midpoint of $\seg{AZ}$, and we are done.

