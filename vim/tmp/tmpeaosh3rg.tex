% Input your problem and solution below.
% Three dashes on a newline indicate the breaking points.

---

Let $P$ and $Q$ be on segment $BC$ of an acute triangle $ABC$ such that $\angle PAB=\angle BCA$ and $\angle CAQ=\angle ABC$. Let $M$ and $N$ be the points on $\seg{AP}$ and $\seg{AQ}$, respectively, such that $P$ is the midpoint of $\seg{AM}$ and $Q$ is the midpoint of $\seg{AN}$. Prove that the intersection of $\seg{BM}$ and $\seg{CN}$ lies on the circumcircle of triangle $ABC$.

---

\begin{center}
    \begin{asy}
        size(6cm);
        defaultpen(fontsize(10pt));
        pair A,B,C,P,Q,M,NN,X;
        A=dir(120);
        B=dir(190);
        C=dir(350);
        P=B+(C-B)*length(A-B)^2/length(C-B)^2;
        Q=C+(B-C)*length(A-C)^2/length(B-C)^2;
        M=2*P-A;
        NN=2*Q-A;
        X=extension(B,M,C,NN);

        draw(circumcircle(A,B,C));
        draw(A--B--C--A);
        draw(B--M--A--NN--C,gray);

        dot("$A$",A,A);
        dot("$B$",B,B);
        dot("$C$",C,C);
        dot("$P$",P,SE);
        dot("$Q$",Q,SE);
        dot("$M$",M,S);
        dot("$N$",NN,dir(255));
        dot("$X$",X,dir(255));
    \end{asy}
\end{center}
\begin{customenv}{First solution, by harmonic bundles}
    Let $\seg{BM}$ intersect $(ABC)$ again at $X$. By the angle conditions, the tangent to $(ABC)$ at $B$ is parallel to $\seg{AP}$. Thus \[-1=(AM;P\infty_{AP})\stackrel B=(AX;CB),\]
    so $X$ is the harmonic conjugate of $A$ wrt.\ $\seg{BC}$. By symmetry, $\seg{CN}$ also intersects the circumcircle at $X$, so $X$ is the desired intersection.
\end{customenv}
\begin{customenv}{Second solution, by similar triangles}
    Let $Y$ and $Z$ lie on $\seg{AB}$ and $\seg{AC}$ such that $B$ is the midpoint of $\seg{AY}$ and $C$ is the midpoint of $\seg{AZ}$. Thus $\triangle MYA\sim\triangle PBA\sim\triangle ABC\sim\triangle QAC\sim\triangle NAZ$. But $\seg{MB}$ and $\seg{NC}$ correspond to medians, so $\triangle MBA\sim\triangle NCZ$, whence $\da(\seg{MB},\seg{NC})=\da(\seg{MA},\seg{NZ})=\da(\seg{BA},\seg{CZ})=\da BAC$, done.
\end{customenv}
\begin{customenv}{Third solution, by barycentric coordinates}
    Since $BP=c^2/a$, we denote $P=(0:a^2-c^2:2c^2)$, and similarly $Q=(0:b^2:a^2-b^2)$. From these it follows that $M=\left(-a^2:2(a^2-c^2):c^2\right)$ and $N=\left(-a^2:2b^2:2(a^2-b^2)\right)$. Finally $X=(-a^2:2b^2:2c^2)$ lies on $\seg{BM}$, $\seg{CN}$, and the circumcircle, completing the proof.
\end{customenv}
\begin{customenv}{Fourth solution, by complex numbers (Justin Lee)}
    Toss on the complex plane. Write $\tfrac{p-b}{a-b}=\overline{(\tfrac{a-b}{c-b})}$, from which\[m-b=b+2(a-b)\overline{\left(\frac{a-b}{c-b}\right)}-a.\]Thusly
    \begin{align*}
        \frac{m-b}{n-c}\div\frac{a-b}{a-c}&=\frac{b+2(a-b)\overline{(\frac{a-b}{c-b})}-a}{c+2(a-c)\overline{(\frac{a-c}{b-c})}-a}\div\frac{a-b}{a-c}\\
        &=\frac{2\overline{(\frac{a-b}{c-b})}-1}{2\overline{(\frac{a-c}{b-c})}-1}=-\frac{2\overline a-\overline b-\overline c}{2\overline a-\overline b-\overline c}=-1,
    \end{align*}
    which is sufficient.
\end{customenv}

