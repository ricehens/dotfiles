% Input your problem and solution below.
% Three dashes on a newline indicate the breaking points.

---

Let $ABC$ be a triangle with $\angle B>\angle C$. Let $P$ and $Q$ be two different points on line $AC$ such that $\angle PBA=\angle QBA=\angle ACB$ and $A$ is located between $P$ and $C$. Suppose that there exists an interior point $D$ of segment $BQ$ for which $PD=PB$. Let the ray $AD$ intersect the circle $ABC$ at $R\ne A$. Prove that $QB=QR$.

---

To begin, first note that $\triangle ABQ\sim\triangle ACB$ by the angle conditions; furthermore by $\da ABD=-\da ARB$, we have $\triangle ABD\sim\triangle ARB$. Hence
\[AC\cdot AQ=AB^2=AD\cdot AR.\]
so quadrilateral $CQDR$ is cyclic.
    \begin{center}
    \begin{asy}
        size(6cm); defaultpen(fontsize(10pt));
        pen pri=lightred;
        pen sec=lightblue;
        pen tri=purple+pink;
        pen fil=pri+opacity(0.05);
        pen sfil=sec+opacity(0.05);
        pen tfil=tri+opacity(0.05);

        pair A,B,C,U,Q,P,D,R,X,SS,T;
        A=dir(140);
        B=dir(205);
        C=dir(335);
        U=extension(B,B+rotate(90)*(B-A),(B+C),origin);
        Q=2*foot(U,A,C)-C;
        P=extension(A,C,B,reflect(A,B)*Q);
        D=2*foot(P,B,Q)-B;
        R=2*foot(origin,A,D)-A;
        X=2*foot(origin,B,Q)-B;
        SS=2*foot(circumcenter(C,Q,D),P,D)-D;
        T=extension(B,Q,C,SS);

        filldraw(circumcircle(B,Q,C),sfil,sec);
        draw(A--R,tri);
        draw(D--P--B--Q--R,sec);
        draw(Q--D,sec);
        //draw(Q--X,sec);
        draw(A--P,pri);
        filldraw(unitcircle,fil,pri);
        filldraw(A--B--C--cycle,fil,pri);
        filldraw(circumcircle(C,Q,D),tfil,tri);

        //delete later
        //draw(anglemark(A,C,B));
        //draw(anglemark(A,B,P));
        //draw(anglemark(Q,B,A));
        dot("$A$",A,dir(100));
        dot("$B$",B,B);
        dot("$C$",C,C);
        dot("$Q$",Q,S);
        dot("$P$",P,NW);
        dot("$D$",D,N);
        dot("$R$",R,R);
        //dot("$X$",X,X);
    \end{asy}
\end{center}

By the given angle conditions, $\triangle PAB\sim\triangle PBC$, as well. In addition, observe from $PD^2=PB^2=PA\cdot PC$ that $\triangle PAD\sim\triangle PCD$. Therefore,
\begin{align*}
    \da QRB&=\da ARB+\da QRA=\da ACB+\da ACD=\da ACB+\da ADP\\
    &=\da ACB+\da ADP+\da BDA+\da RCA=\da ACB+\da BDP+\da RCA\\
    &=\da BCA+\da RCA=\da ABQ+\da RBA=\da RBQ,
\end{align*}
so $QB=QR$, as desired.


