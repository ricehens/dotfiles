% Input your problem and solution below.
% Three dashes on a newline indicate the breaking points.

---

Let $ABC$ be an equilateral triangle and $P$ a point in its interior obeying $AP=5$, $BP=7$, $CP=8$. Line $CP$ intersects $\seg{AB}$ at $Q$. Compute $AQ$.

---

Assume $\angle CAB$ is oriented counterclockwise. Let $R$ be the image of $P$ under $60\dg$ clockwise rotation at $A$. Since $AP=AR=5$ and $\angle PAR=60\dg$, $\triangle APR$ is equilateral; i.e.\ $PR=5$. Since $BR=8$, $BP=7$, we have $\angle BRP=60\dg$, say by cosine law on $\triangle BPR$.
\begin{center}
\begin{asy}
    size(5cm); defaultpen(fontsize(10pt));

    real t=sqrt(43);
    pair A,B,C,P,Q,R;
    A=t*dir(90);
    B=t*dir(210);
    C=t*dir(330);
    P=intersectionpoints(circle(B,7),circle(C,8))[0];
    Q=extension(C,P,A,B);
    R=2*foot(origin,C,P)-C;

    draw(circle(origin,t),gray);
    draw(A--P--B--R--A,gray);
    draw(C--A--B--C--R);

    dot("$A$",A,A/t);
    dot("$B$",B,B/t);
    dot("$C$",C,C/t);
    dot("$P$",P,S);
    dot("$Q$",Q,dir(110));
    dot("$R$",R,R/t);
\end{asy}
\end{center}
Now $\angle APR=60\dg$ and $\angle APC=120\dg$, so $C$, $P$, $R$ collinear; thus $\seg{RC}$ bisects $\angle ARB$ and \[\frac{AQ}{BQ}=\frac{AR}{BR}=\frac58.\]
Finally, cosine law on $\triangle ARB$ gives $AB=\sqrt{129}$, so $AQ=\frac{5\sqrt{129}}{13}$ as desired.

---

$\frac{5\sqrt{129}}{13}$
