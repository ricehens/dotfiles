% Input your problem and solution below.
% Three dashes on a newline indicate the breaking points.

---

Let $ABCC_1B_1A_1$ be a convex hexagon with $AB=BC$, and suppose that segments $AA_1$, $BB_1$, $CC_1$ have the same perpendicular bisector. Let diagonals $AC_1$ and $A_1C$ meet at $D$, and denote by $\omega$ the circumcircle of $\triangle ABC$. Let $\omega$ intersect the circumcircle of $\triangle A_1BC_1$ again at $E\ne B$. Prove that lines $BB_1$ and $DE$ intersect on $\omega$.

---

Let $\ell$ be the common perpendicular bisector of $\seg{AA_1}$, $\seg{BB_1}$, $\seg{CC_1}$, and let $\seg{AC}$, $\seg{A_1C_1}$, $\ell$ concur at $F$. Applying radical axis on $\omega$, $(A_1BC_1)$, $(AA_1C_1C)$, we find $B$, $E$, $F$ collinear.
\begin{center}
\begin{asy}
    size(8cm); defaultpen(fontsize(10pt));
    pen pri=orange;
    pen sec=lightblue;
    pen tri=lightred;
    pen qua=lightolive;
    pen fil=yellow+opacity(0.05);
    pen sfil=sec+opacity(0.05);
    pen tfil=tri+opacity(0.05);

    pair O,A,B,C,U,V,X,Y,A1,C1,D,EE,F,Z;
    O=origin;
    A=dir(185);
    B=dir(115);
    C=reflect(O,B)*A;
    U=dir(200);
    V=reflect(O,O+(0,1))*U;
    X=-U*V/B;
    Y=-B;
    A1=reflect(U,V)*A;
    C1=reflect(U,V)*C;
    D=extension(A,C1,A1,C);
    EE=2*foot(O,X,D)-X;
    F=extension(B,EE,A,C);
    Z=extension(U,V,B,Y);

    draw(B--X,qua+dashed);
    draw(A--A1,qua+Dotted);
    draw(C--C1,qua+Dotted);
    draw(F--C1,qua+Dotted);
    draw(A--D--C,qua);
    filldraw(circumcircle(B,EE,D),tfil,tri);
    draw(X--Y,tri);
    draw(B--Y,sec+dashed);
    filldraw(circle(O,1),fil,pri);
    draw(B--F--C,pri);
    draw(EE--X,linewidth(0.3)+pri);
    draw(F--V+(.2,0),royalblue);

    dot("$A$",A,dir(150));
    dot("$B$",B,dir(110));
    dot("$C$",C,C);
    dot("$X$",X,X);
    dot("$Y$",Y,Y);
    dot("$A_1$",A1,dir(240));
    dot("$C_1$",C1,SE);
    dot("$D$",D,SW);
    dot("$E$",EE,EE);
    dot("$F$",F,W);
    dot("$Z$",Z,dir(255));
\end{asy}
\end{center}
Let $Y$ be the antipode of $B$, so that $\seg{BY}$ perpendicularly bisects $\seg{AC}$. Then let $Z=\seg{BY}\cap\ell$, so that $ZA=ZC$. By design, $\seg{FDZ}$ externally bisects $\angle ADC$, so $ACZD$ is cyclic.

By $FE\cdot FB=FA\cdot FC=FD\cdot FZ$, we have $EBZD$ cyclic. By Reim's theorem on $\omega$, $(EBZD)$, the second intersection $X$ of $\seg{ED}$ with $\omega$ should obey $\seg{XY}\parallel\seg{DZ}$. Recall $Y$ is the antipode of $B$, so $\seg{BX}\perp\ell$. Then $X$ lies on $\seg{BB_1}$, end proof.

