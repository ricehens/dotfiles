% Input your problem and solution below.
% Three dashes on a newline indicate the breaking points.

---

Let $ABC$ be a triangle with incenter $I$, and suppose the incircle is tangent to $\seg{CA}$ and $\seg{AB}$ at $E$ and $F$. Denote by $G$ and $H$ the reflections of $E$ and $F$ over $I$. Lines $BC$ and $GH$ meet at $Q$, and $M$ is the midpoint of $BC$. Prove that $\angle QIM=90\dg$.

---

We present four solutions.

\begin{center}
\begin{asy}
    size(7cm); defaultpen(fontsize(10pt));
    pen pri=lightred;
    pen sec=lightblue;
    pen tri=purple+pink;
    pen fil=pri+opacity(0.05);
    pen sfil=sec+opacity(0.05);
    pen tfil=tri+opacity(0.05);

    pair A,B,C,I,D,EE,F,G,H,Q,Ap,Dp,T,M,SS;
    A=dir(120);
    B=dir(210);
    C=dir(330);
    I=incenter(A,B,C);
    D=foot(I,B,C);
    EE=foot(I,C,A);
    F=foot(I,A,B);
    G=2I-EE;
    H=2I-F;
    Q=extension(B,C,G,H);
    Ap=2I-A;
    Dp=2I-D;
    T=B+C-D;
    M=(B+C)/2;
    SS=reflect(Q,I)*D;

    draw(SS--Ap,sec);
    draw(I--M,sec);
    draw(A--T,sec);
    draw(Q--I,sec);
    draw(Q--H,tri+dashed);
    draw(G--Ap--H,tri);
    draw(Q--SS,pri+Dotted);
    filldraw(incircle(A,B,C),fil,pri);
    fill(A--B--C--cycle,fil);
    draw(B--A--C--Q,pri);

    dot("$A$",A,N);
    dot("$B$",B,S);
    dot("$C$",C,SE);
    dot("$I$",I,N);
    dot("$D$",D,dir(240));
    dot("$E$",EE,NE);
    dot("$F$",F,W);
    dot("$G$",G,dir(75));
    dot("$H$",H,E);
    dot("$Q$",Q,W);
    dot("$A'$",Ap,S);
    dot("$D'$",Dp,NE);
    dot("$T$",T,S);
    dot("$M$",M,S);
    dot(SS);
\end{asy}
\end{center}
\paragraph{First solution, by polarity} Let $A'$, $D'$ be the reflection of $A$, $D$ across $I$, and let $T=\seg{BC}\cap\seg{AD'}$ be the $A$-extouch point.
Note $Q$ lies on $\seg{BC}$, $\seg{GH}$, the polars of $D$, $A'$, so by La Hire, the polar of $Q$ is line $DA'$. But
\begin{itemize}[itemsep=0em]
    \item $\seg{DA'}\parallel\seg{AD'T}$ by reflecting through $I$; and
    \item $\seg{AD'T}\parallel\seg{IM}$ since $\seg{IM}$ is the $D$-midsegment of $\triangle DD'T$.
\end{itemize}
The polar of $Q$ is parallel to $\seg{IM}$, so $\angle QIM=90\dg$, as needed.

\paragraph{Second solution, by similar triangles} Let $N$ be the midpoint of $\seg{GH}$, so $INDQ$ is cyclic.
\begin{claim*}
    $\triangle IBC\sim\triangle DGH$.
\end{claim*}
\begin{proof}
    If $D'$ is the reflection of $D$ over $I$, then $\seg{DH}\parallel\seg{D'F}\perp\seg{DF}\perp\seg{BI}$, and similarly $\seg{DG}\parallel\seg{CI}$, so $\da DGH=\da CDH=-\da IBC$ and $\da GHD=-\da BCI$.
\end{proof}

Finally, $\triangle IBC\cup M\sim\triangle DGH\sim N$ (inversely), so $\da IMB=-\da DNG=\da QND=\da QID$, and the result follows.

\paragraph{Third solution, by Iran lemma} Let $\seg{BI}$, $\seg{CI}$, $\seg{QI}$ meet $\seg{EF}$ at $U$, $V$, $P$. By Iran lemma, $B$, $C$, $U$, $V$ lie on a circle $\omega$ with center $M$, and by symmetry through $I$, we have $IQ=IP$.
\begin{center}
\begin{asy}
    size(7cm); defaultpen(fontsize(10pt));
    pen pri=lightred;
    pen sec=lightblue;
    pen tri=purple+pink;
    pen fil=pri+opacity(0.05);
    pen sfil=sec+opacity(0.05);
    pen tfil=tri+opacity(0.05);

    pair A,B,C,I,D,EE,F,G,H,Q,M,U,V,P,SS,T;
    A=dir(120);
    B=dir(210);
    C=dir(330);
    I=incenter(A,B,C);
    D=foot(I,B,C);
    EE=foot(I,C,A);
    F=foot(I,A,B);
    G=2I-EE;
    H=2I-F;
    Q=extension(B,C,G,H);
    M=(B+C)/2;
    U=extension(EE,F,I,B);
    V=extension(EE,F,I,C);
    P=extension(EE,F,I,Q);
    SS=intersectionpoint(I--P,circle(M,abs(B-C)/2));
    T=2I-SS;

    draw(I--M,sec);
    draw(Q--P,sec);
    draw(B--U,tri+dashed);
    draw(C--V,tri+dashed);
    draw(Q--H,tri);
    fill(arc(M,C,B,CCW)--cycle,sfil);
    draw(arc(M,C,B,CCW),sec+dashed);
    draw(F--P,pri);
    filldraw(incircle(A,B,C),fil,pri);
    fill(A--B--C--cycle,fil);
    draw(B--A--C--Q,pri);

    dot("$A$",A,N);
    dot("$B$",B,S);
    dot("$C$",C,SE);
    dot("$I$",I,N);
    dot("$D$",D,dir(240));
    dot("$E$",EE,N);
    dot("$F$",F,W);
    dot("$G$",G,S);
    dot("$H$",H,E);
    dot("$Q$",Q,W);
    dot("$M$",M,S);
    dot("$U$",U,N);
    dot("$V$",V,dir(290));
    dot("$P$",P,NE);
    dot("$S$",SS,S);
    dot("$T$",T,NW);
\end{asy}
\end{center}
Let line $PQ$ intersect $\omega$ again at $S$, $T$. By converse Butterfly theorem, $IS=IT$, so $\seg{MI}\perp\seg{PSITQ}$, as needed.

\paragraph{Fourth solution, by complex numbers} Let the incircle be the unit circle. To begin, \[g=-e,\quad h=-f,\quad b=\frac{2df}{d+f},\quad c=\frac{2de}{d+e}.\]
To locate the midpoint of $\seg{BC}$, write \[m=\frac{b+c}2=\frac{d(de+df+2ef)}{(d+e)(d+f)}.\]
With $Q=\seg{DD}\cap\seg{GH}$, the chord intersection formula gives \[q=\frac{2dgh-d^2(g+h)}{gh-d^2}=\frac{d(de+df+2ef)}{ef-d^2}.\]
Thus, we have \[\frac mq=\frac{ef-d^2}{(d+e)(d+f)}\in\mathbb I\]
by taking the conjugate.

