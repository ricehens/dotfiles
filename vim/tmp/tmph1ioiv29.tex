% Input your problem and solution below.
% Three dashes on a newline indicate the breaking points.

---

Let $m$ be a fixed integer greater than $1$. The sequence $x_0$, $x_1$, $x_2$, $\ldots$ is defined as follows:
\[x_i=\begin{cases}2^i&\text{if }0\le i\le m-1;\\ \sum_{j=1}^mx_{i-j}&\text{if }i\ge m.\end{cases}\]
Find the greatest $k$ for which the sequence contains $k$ consecutive terms divisible by $m$.

---

The answer is $m-1$. It is impossible for $m$ consecutive terms to be $0\pmod m$, else the entire sequence is $0\pmod m$, absurd.

By nature of being a linear recurrence, the sequence $\ldots$, $x_{-2}$, $x_{-1}$, $x_0$, $x_1$, $x_2$, $\ldots$ is periodic modulo $m$. But $x_{-2}=\cdots=x_{-m}=0$, so we can find $m-1$ consecutive zeros modulo $m$.


