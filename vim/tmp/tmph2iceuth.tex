% Input your problem and solution below.
% Three dashes on a newline indicate the breaking points.
% vim: tw=72

---

Let $ABC$ be a triangle with $\angle ABC$ obtuse. The $A$-\emph{excircle} is a circle in the exterior of $\triangle ABC$ that is tangent to side $\overline{BC}$ and and tangent to the extensions of the other two sides. Let $E,F$ be the feet of the altitudes from $B$ and $C$ to lines $AC$ and $AB$, respectively. Can line $EF$ be tangent to the $A$-excircle?

---

The answer is no.
\begin{center}
    \begin{asy}
        size(6cm);
        defaultpen(fontsize(9pt));

        pen pri=royalblue;
        pen sec=Cyan;
        pen tri=deepgreen;
        pen fil=royalblue+opacity(0.05);

        pair A, B, C, EE, IA, Bp, Cp, SS, F;
        A=dir(140);
        B=dir(230);
        C=dir(310);
        EE=foot(B, A, C);
        IA=2*dir(270)-incenter(A, B, C);
        Bp=foot(IA, A, C);
        Cp=foot(IA, A, B);
        SS=intersectionpoints(circle(IA, length(Bp-IA)), circle((EE+IA)/2, length(EE-IA)/2))[1];
        F=extension(A, B, EE, SS);

        draw(extension(A, B, IA-(0, length(IA-Bp)), IA-(1, length(IA-Bp))) -- A -- extension(A, C, IA-(0, length(IA-Bp)), IA-(1, length(IA-Bp))), pri);
        draw(B -- C, pri);
        filldraw(circle(IA, length(Bp-IA)), fil, pri);
        fill(A--B--C--cycle,fil);
        draw(B -- EE, sec); draw(C -- F, sec); 
        draw(EE -- F, tri); draw(Bp -- IA -- Cp, pri); draw(SS -- IA, tri);

        dot("$A$", A, N);
        dot("$B$", B, W);
        dot("$C$", C, NE);
        dot("$E$", EE, NE);
        dot("$I_A$", IA, SE);
        dot("$B'$", Bp, NE);
        dot("$C'$", Cp, W);
        dot("$S$", SS, NW);
        dot("$F'$", F, W);
    \end{asy}
\end{center}
Assume for the sake of contradiction such a configuration exists. Since $\overline{BC}$ and $\overline{EF}$ are antiparallel with respect to $\angle A$, $\triangle ABC\sim\triangle AEF$. Note that the problem states that $\triangle ABC$ and $\triangle AEF$ share an $A$-excircle and therefore an $A$-exradius. Thus, $\triangle ABC\cong\triangle AEF$, so $\cos A=AE/AB=1,$ whence $\angle A=0^\circ$, a contradiction.
