% Input your problem and solution below.
% Three dashes on a newline indicate the breaking points.

---

A set of postive integers is called fragrant if it contains at least two elements and each of its elements has a prime factor in common with at least one of the other elements. Let $P(n)=n^2+n+1$. What is the least possible positive integer value of $b$ such that there exists a nonnegative integer $a$ for which the set
\[\{P(a+1),P(a+2),\ldots,P(a+b)\}\]
is fragrant?

---

The answer is $b=6$.

We can check using Euclidean algorithm that
\begin{align*}
    \gcd(P(n),P(n+1))&=1\\
    \gcd(P(n),P(n+2))&=\begin{cases}7&n\equiv2\pmod7\\1&\text{else}\end{cases}\\
    \gcd(P(n),P(n+3))&=\begin{cases}3&n\equiv1\pmod3\\1&\text{else}\end{cases}\\
    \gcd(P(n),P(n+4))&=\begin{cases}19&n\equiv7\pmod{19}\\1&\text{else.}\end{cases}
\end{align*}
From here, we can manually check that all $2\le b\le5$ fail.

Furthermore, $b=6$ works by taking $a=195$ via Chinese Remainder theorem on
\begin{align*}
    a+1&\equiv1\pmod3\\
    a+2&\equiv7\pmod{19}\\
    a+3&\equiv2\pmod7.
\end{align*}
\begin{remark}
    For $b=6$, we can check $\gcd(P(n),P(n+5))=7$ for $n\equiv4\pmod7$ and 1 otherwise. Then by Chinese Remainder theorem we deduce that the only solutions are $a\equiv195\pmod{399}$ and $a\equiv196\pmod{399}$.
\end{remark}

