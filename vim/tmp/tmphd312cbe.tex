% Input your problem and solution below.
% Three dashes on a newline indicate the breaking points.

---

A sequence of real numbers $a_1$, $a_2$, $\ldots$ satisfies the relation
\[a_n=-\max_{i+j=n}(a_i+a_j)\quad\text{for all}\quad n>2017.\]
Prove that the sequence is bounded, i.e., there is a constant $M$ such that $|a_n|\le M$ for all positive integers $n$.

---

Observe that it suffices to bound the sequence from above; if we bound $a_n\le M$, then the sequence is always at least $-2M$.

Let $n$ be a \emph{peak} if $a_n>\min\{a_1,\ldots,a_{k-1}\}$; Assume for contradiction the sequence is unbounded above, so there are infinitely many peaks. Indeed, for large $2017\ll k<\ell$, if $k$ and $\ell$ are consecutive peaks, then $a_k+a_{\ell-k}<-a_\ell<-a_k$, so $a_{\ell-k}<-2a_k$. 

Assume $\ell-k>2017$. This means that for some $i$, we have $a_i+a_{\ell-k-i}>2a_k$; without loss of generality $a_i\ge a_{\ell-k-i}$, so $a_i>a_k$. But this is a contradiction, since
\begin{itemize}[itemsep=0em]
    \item if $i<k$, then $k$ would not be a peak; and
    \item if $k<i<\ell$, then $k$ and $\ell$ would not be consecutive peaks.
\end{itemize}

Hence $\ell-k\le2017$. But then for all peaks $k$, we have $a_k<\frac12a_{\ell-k}$, which is bounded above since $a_{\ell-k}$ takes finitely many values.
