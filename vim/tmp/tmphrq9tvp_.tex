% Input your problem and solution below.
% Three dashes on a newline indicate the breaking points.
% vim: tw=72

---

The isosceles triangle $ABC$, with $AB=AC$, is inscribed in the circle $\omega$. Let $P$ be a variable point on the arc $BC$ that does not contain $A$, and let $I_B$ and $I_C$ denote the incenters of triangles $ABP$ and $ACP$, respectively. Prove that as $P$ varies, the circumcircle of triangle $PI_BI_C$ passes through a fixed point.

---

\begin{center}
    \begin{asy}
        size(8cm);
        defaultpen(fontsize(10pt));
        pen pri=blue;
        pen sec=royalblue;
        pen tri=purple;
        pen fil=blue+opacity(0.05);
        pen sfil=royalblue+opacity(0.05);
        pen tfil=purple+opacity(0.05);
        pair A, B, C, M, MB, MC, P, IB, IC;
        A=dir(90);
        B=dir(190);
        C=dir(-10);
        M=dir(270);
        MB=dir(140);
        MC=dir(40);
        P=dir(220);
        IB=incenter(A, B, P);
        IC=incenter(A, C, P);

        draw(A -- B -- C -- A -- P, pri);
        filldraw(circumcircle(A, B, C), fil, pri);
        filldraw(incircle(A, B, P), sfil, sec);
        filldraw(incircle(A, C, P), sfil, sec);
        draw(arc(MB,A, B, CW), sec);
        draw(arc(MC, A, C, CCW), sec);
        draw(IB -- P -- IC, sec); draw(B -- P -- C, pri);
        filldraw(M -- IB -- MB -- cycle, tfil, tri);
        filldraw(M -- IC -- MC -- cycle, tfil, tri);

        dot("$A$", A, A);
        dot("$B$", B, B);
        dot("$C$", C, C);
        dot("$M$", M, M);
        dot("$M_B$", MB, MB);
        dot("$M_C$", MC, MC);
        dot("$P$", P, P);
        dot("$I_B$", IB, SW);
        dot("$I_C$", IC, N);
    \end{asy}
\end{center}

I claim that the fixed point is $M$, the midpoint of $\widehat{BPC}$. Let $\seg{PI_B}$ and $\seg{PI_C}$ intersect $(ABC)$ again at $M_B$ and $M_C$, respectively. By the Incenter-Excenter Lemma, \[M_BI_B=M_BA=M_CA=M_CI_C.\]
Furthermore, $MM_B=MM_C$ and \[\da MM_BI_B=\da MM_BP=\da MM_CP=\da MM_CI_C,\]
so $\triangle MM_BI_B\cong\triangle MM_CI_C$. Since spiral similarities come in pairs, $M$ is the center of spiral similarity sending $\overline{I_BI_C}$ to $\overline{M_BM_C}$. Hence, $\overline{M_BI_B}\cap\overline{M_CI_C}=P$ lies on $(MI_BI_C)$, and we are done.
