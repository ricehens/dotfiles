% Input your problem and solution below.
% Three dashes on a newline indicate the breaking points.

---

Given a triangle $ABC$, let $P$ and $Q$ be points on segments $\seg{AB}$ and $\seg{AC}$, respectively, such that $AP=AQ$. Let $S$ and $R$ be distinct points on segment $\seg{BC}$ such that $S$ lies between B and R, $\angle BPS=\angle PRS$, and $\angle CQR=\angle QSR$. Prove that $P$, $Q$, $R$, $S$ are concyclic.

---

\begin{center}
    \begin{asy}
        size(8cm);
        defaultpen(fontsize(10pt));
        pen pri=red+linewidth(0.5);
        pen sec=orange+linewidth(0.5);
        pen fil=red+opacity(0.05);
        pen sfil=orange+opacity(0.05);
        pair A, B, C, O, P, Q, R, SS;
        A=dir(110);
        B=dir(210);
        C=dir(330);
        O=(4*incenter(A, B, C)-A)/3;
        P=foot(O, A, B);
        Q=foot(O, A, C);
        R=intersectionpoints(circle(O, length(P-O)), B -- C)[1];
        SS=intersectionpoints(circle(O, length(P-O)), B -- C)[0];
        draw(A -- B -- C -- A, pri);
        fill(A -- B -- C -- cycle, fil);
        draw(P -- SS -- Q -- R -- P, sec);
        filldraw(circle(O, length(P-O)), sfil, sec);
        dot("$A$", A, N);
        dot("$B$", B, SW);
        dot("$C$", C, SE);
        dot("$P$", P, unit(P-R));
        dot("$Q$", Q, unit(Q-SS));
        dot("$R$", R, SE);
        dot("$S$", SS, SW);
    \end{asy}
\end{center}
Assume for the sake of contradiction that $PQRS$ is not cyclic. Notice that $\angle BPS=\angle PRS$ is equivalent to $(PRS)$ being tangent to $\seg{AB}$ at $P$. Similarly, we have that $(QRS)$ is tangent to $\seg{AC}$ at $Q$. Clearly, $\seg{RS}$ is the radical axis of $(PRS)$ and $(QRS)$. However, \[\pow(A, (PRS))=AP^2=AQ^2=\pow(A, (QRS)),\]
so $A\in\seg{RS}$, which is absurd.
