% Input your problem and solution below.
% Three dashes on a newline indicate the breaking points.

---

Let $\mathcal S$ be a finite set of at least two points in the plane. Assume that no three points of $\mathcal S$ are collinear. A \emph{windmill} is a process that starts with a line $\ell$ going through a single point $P\in\mathcal S$. The line rotates clockwise about the \emph{pivot} $P$ until the first time that the line meets some other point belonging to $\mathcal S$. This point, $Q$, takes over as the new pivot, and the line now rotates clockwise about $Q$, until it next meets a point of $\mathcal S$. This process continues indefinitely.

Show that we can choose a point $P$ in $\mathcal S$ and a line $\ell$ going through $P$ such that the resulting windmill uses each point of $\mathcal S$ as a pivot infinitely many times.

---

For each point $P$, there is a line $\ell$ that partitions $\mathcal S\setminus\{P\}$ into two subsets whose cardinalities differ by at most $1$. I claim $\ell$ works.

Color one side of $\ell$ red, and the other side blue. Without loss of generality, the number of points on the red side minus the number of points on the blue side is either $0$ or $1$. Note the following:
\begin{itemize}
    \item The number of points on the red side is invariant. Analogously so is the number of points on the blue side.
    \item There is an angle $\theta$ such that whenever the pivot changes, the change in horizontal angle formed by $\ell$ is at least $\theta$.
    \item For each direction, there is a unique point $P$ such that the line through $P$ in that direction has either $0$ or $1$ more points on the red side than the blue side.
\end{itemize}
Each time $\ell$ performs a complete $360\dg$ rotation, every point in $\mathcal S$ is the pivot at least once.

