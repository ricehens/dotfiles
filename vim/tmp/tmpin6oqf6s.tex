% Input your problem and solution below.
% Three dashes on a newline indicate the breaking points.

---

Say an arrangement of 2500 kings on a $100\times100$ chessboard is \emph{good} if
\begin{enumerate}[label=(\roman*),itemsep=0em]
    \item no king can capture any other one (i.e. no two kings are placed in two squares sharing a common vertex); and
    \item each row and each column contains exactly 25 kings.
\end{enumerate}
Find the number of good arrangements. (Two arrangements differing by rotation or symmetry are supposed to be different.)

---

The answer is two. Replace 25 with $n$, 100 with $4n$, and 2500 with $4n^2$. Divide the $4n\times4n$ grid into $4n^2$ two-by-two squares.
\setcounter{claim}0
\begin{claim}
    In a good arrangement, each two-by-two square contains exactly one king.
\end{claim}
\begin{proof}
    No square contains two kings.
\end{proof}

We can assign each two-by-two square a \emph{row bit}, which is zero if its king is in the top row and one if in the bottom row, and \emph{width bit}, which is zero if the king is in the left column and one if in the right column.

For example, consider this good arrangement on an $8\times8$ grid ($n=2$):
\begin{center}
\begin{asy}
    size(5cm); defaultpen(fontsize(16pt));
    usepackage("skak");

    for (int i=0; i<=8; i+=2) {
        draw( (i,0)--(i,8));
        draw( (0,i)--(8,i));
    }
    for (int i=1; i<=7; i+=2) {
        draw( (i,0)--(i,8),gray);
        draw( (0,i)--(8,i),gray);
    }

    for (int i=0; i<=2; i+=2)
    for (int j=1; j<=3; j+=2)
    label("\symking", (i+.5,j+.5));

    for (int i=4; i<=6; i+=2)
    for (int j=0; j<=2; j+=2)
    label("\symking", (i+.5,j+.5));

    for (int i=1; i<=3; i+=2)
    for (int j=5; j<=7; j+=2)
    label("\symking", (i+.5,j+.5));

    for (int i=5; i<=7; i+=2)
    for (int j=4; j<=6; j+=2)
    label("\symking", (i+.5,j+.5));
\end{asy}
\end{center}
We can represent its row bits and column bits in two $4\times4$ matrices:
\[R=\begin{bmatrix}0&0&1&1\\0&0&1&1\\0&0&1&1\\0&0&1&1\end{bmatrix}\quad\text{and}\quad C=\begin{bmatrix}1&1&1&1\\1&1&1&1\\0&0&0&0\\0&0&0&0\end{bmatrix}.\]
In general, $R$ and $C$ are $2n\times2n$ matrices.
\begin{claim}
    $R$ contains $n$ columns filled with zeros and $n$ columns filled with ones. (Similarly $C$ contains $n$ rows filled with zeros and $n$ rows filled with ones.)
\end{claim}
\begin{proof}
    Since each row and column contains exactly $n$ kings. Half the columns of $R$ must have top element 1. But if a cell of $R$ contains a 1, then the cell below contains a 1 as well, else the two corresponding kings can capture each other. Thus exactly $n$ columns are filled with 1's.

    An analogous argument, this time starting from the bottom cell of each column, proves $n$ columns are filled with zeros.
\end{proof}

Therefore, each good arrangement can be represented as an \href{https://en.wikipedia.org/wiki/Outer_product}{outer product}. For example, the $8\times8$ grid above can be represented as
\[\mathbf c\otimes\mathbf r=\begin{bmatrix}1\\1\\0\\0\end{bmatrix}\otimes\begin{bmatrix}0\\0\\1\\1\end{bmatrix}.\]
\begin{claim}
    Let $\mathbf c=\langle c_1,\ldots,c_n\rangle$ and $\mathbf r=\langle r_1,\ldots,r_n\rangle$. There are no $i$, $j$ for which $(c_i,c_{i+1})=(r_j,r_{j+1})=(0,1)$. Similarly $(c_i,c_{i+1})=(r_j,r_{j+1})=(1,0)$ is impossible as well.
\end{claim}
\begin{proof}
    If $(c_i,c_{i+1})=(r_j,r_{j+1})=(0,1)$, then the kings in the $(i+1,j)$ and $(i,j+1)$ two-by-two squares touch at a vertex. If $(c_i,c_{i+1})=(r_j,r_{j+1})=(1,0)$, then the kings in the $(i,j)$ and $(i+1,j+1)$ two-by-two squares touch at a vertex.
\end{proof}

Therefore the only good arrangements are
\[
    \begin{bmatrix}
        1\\\vdots\\1\\0\\\vdots\\0
    \end{bmatrix}
    \otimes
    \begin{bmatrix}
        0\\\vdots\\0\\1\\\vdots\\1
    \end{bmatrix}
    \quad\text{and}\quad
    \begin{bmatrix}
        0\\\vdots\\0\\1\\\vdots\\1
    \end{bmatrix}
    \otimes
    \begin{bmatrix}
        1\\\vdots\\1\\0\\\vdots\\0
    \end{bmatrix},
\]
which work.

