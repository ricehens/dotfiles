% Input your problem and solution below.
% Three dashes on a newline indicate the breaking points.
% vim: tw=72

---

Let $ABC$ be a triangle with circumcircle $\omega$ and incenter $I$. A line $\ell$ meets the lines $AI$, $BI$, and $CI$ at points $D$, $E$, and $F$, respectively, all distinct from $A$, $B$, $C$ and $I$. Prove that the circumcircle of the triangle determined by the perpendicular bisectors of $\overline{AD}$, $\overline{BE}$, and $\overline{CF}$ is tangent to $\omega$.

---

\begin{customenv}{First solution, by angle chasing}
    Let $\ell=\overline{DEF}$ and $\ell_A$, $\ell_B$, and $\ell_C$ denote its reflections across the perpendicular bisectors of $\overline{AD}$, $\overline{BE}$, and $\overline{CF}$, respectively.
    \begin{center}
        \begin{asy}
            size(8cm);
            defaultpen(fontsize(10pt));

            pen pri=deepgreen;
            pen sec=royalblue;
            pen tri=green;
            pen qua=chartreuse;
            pen fil=pri+opacity(0.05);
            pen sfil=sec+opacity(0.05);
            pen tfil=tri+opacity(0.05);
            pen qfil=qua+opacity(0.05);

            pair A, B, C, I, M, NN, P, D, EE, F, G, T, H, X, Y, Z, Mp, Np, Pp;
            A=dir(110);
            B=dir(210);
            C=dir(330);
            I=incenter(A, B, C);
            M=dir(270);
            NN=dir(40);
            P=dir(160);
            D=(2I+A)/3;
            EE=(3B-1I)/2;
            F=extension(C, I, D, EE);
            G=2*foot(F, (A+D)/2, (A+D)/2+NN-P)-F;
            T=intersectionpoint(circumcircle(A, B, C), (A + 0.01*(A-G)) -- (A + 100*(A-G)));
            H=extension(T, I, D, F);
            X=T+(A-T)*length(H-T)/length(I-T);
            Y=T+(B-T)*length(H-T)/length(I-T);
            Z=T+(C-T)*length(H-T)/length(I-T);
            Mp=T+(M-T)*length(H-T)/length(I-T);
            Np=T+(NN-T)*length(H-T)/length(I-T);
            Pp=T+(P-T)*length(H-T)/length(I-T);

            filldraw(circumcircle(A, B, C), fil, pri);
            filldraw(circumcircle(X, Y, Z), sfil, sec);
            draw(A -- B -- C -- A, pri);
            draw(Mp -- Np -- Pp -- Mp, tri);
            draw(X -- Y -- Z -- X, sec);
            draw(A -- I -- EE, pri); draw(C -- F, pri);
            draw(X -- T -- H, qua); draw(Y -- T -- Z, qua);
            draw(extension(A, T, Np, Pp) -- EE, sec);

            dot("$A$", A, N);
            dot("$B$", B, S);
            dot("$C$", C, S);
            dot("$X$", X, A);
            dot("$Y$", Y, B);
            dot("$Z$", Z, C);
            dot("$M$", Mp, M);
            dot("$N$", Np, NN);
            dot("$P$", Pp, P);
            dot("$T$", T, T);
            dot("$I$", I, S);
            dot("$D$", D, dir(150));
            dot("$E$", EE, W);
            dot("$F$", F, NW);
            dot("$H$", H, dir(100));
            dot(extension(A, T, Np, Pp));
        \end{asy}
    \end{center}
    \begin{iclaim*}
        Lines $\ell_A$, $\ell_B$, and $\ell_C$ concur at a point $T$ on $\omega$.
    \end{iclaim*}
    \begin{proof}
        Check that \[\da(\ell_B,\ell_C)=\da(\ell_B,\ell)+\da(\ell,\ell_C)=2\da(\ell_B,\overline{BI})+2\da(\overline{CI},\ell_C)=2\da(\ell_B,\ell_C)+2\measuredangle CIB,\]
        whence $\da(\ell_B,\ell_C)=2\measuredangle BIC=\measuredangle BAC$, as desired.
    \end{proof}

    Now let $H=\overline{TI}\cap\ell$. A homothety at $T$ maps $I$ to $H$ and $\triangle ABC$ to some triangle $\triangle XYZ$ whose incenter is $H$ such that $(ABC)$ and $(XYZ)$ are tangent at $T$.

    However, $\overline{XH}\parallel\overline{AD}$, so the perpendicular bisector of $\overline{XH}$ is the perpendicular bisector of $\overline{AD}$, whence the vertices of the triangle formed by the perpendicular bisectors of $\overline{AD}$, $\overline{BE}$, and $\overline{CF}$ are the arc midpoints of $\triangle XYZ$, which completes the proof. 
\end{customenv}
\begin{customenv}{Second solution, by Simson Lines}[Pitchayut Saengrungkongka]
    Let the midpoints of arcs $BC$, $CA$, $AB$ not containing $A$, $B$, $C$ be $M_A$, $M_B$, $M_C$ respectively, and let lines through $D$, $E$, $F$ perpendicular to $\seg{AI}$, $\seg{BI}$, $\seg{CI}$, respectively form a triangle $A_1B_1C_1$. Also let $XYZ$ be the triangle described in the problem. Since $\triangle M_AM_BM_C$ and $\triangle XYZ$ are homothetic, $\seg{XM_A}$, $\seg{YM_B}$, $\seg{ZM_C}$ concur at a point $T$.
    \begin{iclaim*}
        $IA_1B_1C_1$ and $TM_AM_BM_C$ are homothetic.
    \end{iclaim*}
    \begin{proof}
        Since $\triangle A_1B_1C_1$ and $\triangle M_AM_BM_C$ are homothetic it suffices to show that $\seg{IA_1}\parallel\seg{M_AX}$. Let $I_A$ be the $A$-excenter. Then $\seg{M_AX}$ is the $I_A$-midsegment of $\triangle I_AIA_1$, since the projections of $I_A$, $A_1$, $X$ onto $\seg{BI}$ are $B$, the midpoint of $\seg{BE}$, and $E$. Thus the claim is true.
    \end{proof}

    To finish, note that $\seg{DEF}$ is the Simson Line from $I$ to $\triangle A_1B_1C_1$, so $IA_1B_1C_1$ and thus $TM_AM_BM_C$ is cyclic. By homothety, $(ABC)$ and $(XYZ)$ are tangent at $T$, so we are done. 
\end{customenv}

