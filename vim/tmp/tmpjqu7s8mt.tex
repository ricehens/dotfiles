% Input your problem and solution below.
% Three dashes on a newline indicate the breaking points.

---

A circle $\Gamma$ through $A$ of triangle $ABC$ meets sides $AB$, $AC$ at $E$, $F$ respectively and the circumcircle of $ABC$ at $P$. Prove that the reflection of $P$ across $\seg{EF}$ is on $\seg{BC}$ if and only if $\Gamma$ passes through the circumcenter of $ABC$.

---

Note that $\triangle PEB\sim\triangle PFC$. The core idea is this:
\begin{claim*}
    The circumcenter $O$ lies on $\Gamma$ if and only if $EB=EP$ and $FC=FP$.
\end{claim*}
\begin{proof}
    Let the spiral similarity at $O$ sending $\seg{EF}$ to $\seg{BC}$ send $O$ to $O'$. Then $O$ lies on $\Gamma$ if and only if $O'$ lies on $(ABC)$, which occurs if and only if $OA=OO'$. The claim follows from $\triangle PEB\sim\triangle POO'\sim\triangle PFC$.
\end{proof}
\begin{center}
    \begin{asy}
        size(6cm); defaultpen(fontsize(10pt));
        pen pri=heavygreen;
        pen sec=lightblue;
        pen tri=olive;
        pen fil=green+opacity(0.05);
        pen sfil=sec+opacity(0.05);
        pen tfil=yellow+opacity(0.05);

        pair O,A,B,C,EE,F,Q,P;
        O=origin;
        A=dir(100);
        B=dir(210);
        C=dir(330);
        EE=(3C+4A)/7;
        F=2*foot(circumcenter(A,EE,O),A,B)-A;
        Q=2*foot(F,B,C)-B;
        P=reflect(EE,F)*Q;

        filldraw(circumcircle(A,EE,F),sfil,sec);
        draw(EE--Q--F,sec);
        filldraw(A--B--C--cycle,fil,pri);
        filldraw(unitcircle,fil,pri);
        filldraw(P--EE--C--cycle,tfil,tri);
        filldraw(P--F--B--cycle,tfil,tri);

        dot("$A$",A,N);
        dot("$B$",B,B);
        dot("$C$",C,C);
        dot("$Q$",Q,S);
        dot("$P$",P,P);
        dot("$F$",EE,dir(15));
        dot("$E$",F,dir(195));
        dot("$O$",O,NW);
    \end{asy}
\end{center}

Let $Q$ be the reflection of $P$ over $\seg{EF}$, so that $\angle EQF=\angle FPE=\angle CAB$. Now we prove the two directions:

\paragraph{If direction} Suppose $O$ lies on $\Gamma$, so $EQ=EP=EB$ and $FQ=FP=FC$. In what follows, $\angle BQC$ is the angle interior to $ABQC$: we have
\[\angle BQC=\angle BQE+\angle EQF+\angle FQC=\angle ABQ+\angle CAB+\angle QCA=360\dg-\angle BQC,\]
so $\angle BQC=180\dg$.

\paragraph{Only if direction} Suppose $Q$ lies on $\seg{BC}$, but $O$ does not lie on $\Gamma$. Then from the above argument,
\[\frac{EQ}{BF}=\frac{FQ}{CE}\ne1.\]
\begin{itemize}
\item If $EQ<BF$ and $FQ<CE$: then
    \[\angle BQC=\angle BQE+\angle EQF+\angle FQC<\angle B+\angle A+\angle C=180\dg.\]
\item If $EQ>BF$ and $FQ>CE$: then
    \[\angle BQC=\angle BQE+\angle EQF+\angle FQC>\angle B+\angle A+\angle C=180\dg.\]
\end{itemize}
Both cases yield contradiction.
