% Input your problem and solution below.
% Three dashes on a newline indicate the breaking points.

---

Let $\mathbb Z/n\mathbb Z$ denote the set of integers considered modulo $n$ (hence $\mathbb Z/n\mathbb Z$ has $n$ elements). Find all positive integers $n$ for which there exists a bijective function $g:\mathbb Z/n\mathbb Z\to\mathbb Z/n\mathbb Z$, such that the 101 functions \[g(x),\quad g(x)+x,\quad g(x)+2x,\quad\ldots,\quad g(x)+100x\]

---

The answer is all $n$ with no prime factors less than or equal to $101$. To see that $g$ exists for such $n$, just take $g(x)=x$.

Let $p$ be the smallest prime factor of $n$. Assume that $p\le101$; we will show \[g(x),\quad g(x)+x,\quad g(x)+2x,\quad\ldots,\quad g(x)+(p-1)x\]
cannot all be bijections on $\mathbb Z/n\mathbb Z$.
\begin{iclaim*}
    We can eliminate $n$ even; i.e.\ henceforth assume $n$ is odd and $p\ge3$.
\end{iclaim*}
\begin{proof}
    Suppose that the function $g$ exists. We know \[\sum_{x=0}^{n-1}g(x)\equiv\sum_{x=0}^{n-1}x\equiv\frac{n(n-1)}2\pmod n,\]
    and so \[\frac{n(n-1)}2\equiv\sum_{x=0}^{n-1}\Big[g(x)+x\Big]\equiv\sum_{x=0}^{n-1}g(x)+\sum_{x=0}^{n-1}x=0\pmod n,\]
    thus $n$ is odd, contradiction.
\end{proof}
\setcounter{boxlemma}0
\begin{boxlemma}
    If $e=\nu_p(n)$, then \[\nu_p\left(\sum_{x=0}^{n-1}x^{p-1}\right)=e-1.\]
\end{boxlemma}
\begin{proof}
    Note that \[\sum_{x=0}^{n-1}x^{p-1}\equiv\frac n{p^e}\sum_{x=0}{p^e-1}x^{p-1}\pmod{p^e},\]
    so we prove $\nu_p\left(\sum_{x=0}^{p^e-1}x^{p-1}\right)=e-1$. Define \[T_k:=\sum_{\nu_p(x)=k}x^{p-1},\quad\text{so that}\quad\sum_{x=0}^{p^e-1}x^{p-1}\equiv\sum_{k=0}^eT_k\pmod{p^e}.\]
    If $g$ denotes a primitive root mod $p^k$, then \[T_k\equiv p^{k(p-1)}\sum_{i=0}^{\phi(p^{e-k}-1)}g^{i(p-1)}\equiv\frac{g^{(p-1)\phi(p^{e-k})}-1}{g^{p-1}-1}p^{k(p-1)}\pmod{p^e}.\]
    By Lifting the Exponent, \[\nu_p\left(g^{(p-1)\phi(p^{e-k})}-1\right)=\nu_p\left(g^{p-1}-1\right)+e-k-1,\]
    so $\nu_p(T_k)=e-k-1+k(p-1)$, which equals $e-1$ when $k=0$ and is at least $e$ otherwise. Thus \[\nu_p\left(\sum_{k=0}^eT_k\right)=\nu_p(T_0)=e-1,\]
    as desired.
\end{proof}
\begin{boxlemma}
    If such a function $g$ exists, then \[\sum_{x=0}^{n-1}x^{p-1}\equiv0\pmod{p^e}.\]
\end{boxlemma}
\begin{proof}
    Consider the polynomial \[f(k):=\sum_{k=0}^{n-1}\Big(g(x)+kx\Big)^{p-1}-\sum_{x=0}^{n-1}x^{p-1}.\]
    Note that $f$ has degree $p-1$, but $p^e\mid f(x)$ for $x=0,\ldots,p-1$. I claim that these two conditions imply all of $f$'s coefficients are divisible by $p^e$. Looking at the leading coefficient proves the lemma.

    We proceed by induction on $e$. When $e=1$, $f$ is always divisible by $p$, whence so are its coefficients. For the inductive step, if $p^e\mid f(x)$ for all $x=0,\ldots,p-1$, by the hypothesis for $e-1$, $p^{e-1}$ divides the coefficients of $f(x)$, but $g(x)=f(x)/p$ is still divisible by $p^e$ by the inductive step, as desired.
\end{proof}

We conclude by combining Lemma 1 and Lemma 2.

