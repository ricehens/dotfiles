% Input your problem and solution below.
% Three dashes on a newline indicate the breaking points.
% vim: tw=72

---

Let $ABC$ be a triangle and let $D$ be a point on the interior of segment $BC$. The circumcircles of $\triangle ABD$ and $\triangle ACD$ meet segments $AC$ and $AB$, respectively, again at points $E$ and $F$, respectively. Let $A'$ be the reflection of $A$ in $\overline{BC}$. Prove that if $P=\overline{A'C}\cap\overline{DE}$ and $Q=\overline{A'B}\cap\overline{DF}$, then lines $AD$, $BP$, and $CQ$ concur (or are all parallel).

---

Set $X=\overline{BC}\cap\overline{PQ}$, $Y=\overline{CA}\cap\overline{FD}$, and $Z=\overline{AB}\cap\overline{DE}$. Let $E'$ be the reflection of $E$ across $\overline{BC}$.
\begin{center}
    \begin{asy}
        size(10cm);
        defaultpen(fontsize(10pt));

        pen pri=purple;
        pen sec=magenta;
        pen fil=purple+opacity(0.05);

        pair A, B, C, D, EE, F, Ap, Ep, P, Q, X, Y, Z;
        A=dir(135);
        B=dir(200);
        C=dir(340);
        D=foot(A, B, C)+0.1;
        EE=intersectionpoints(circumcircle(A, B, D), A -- C)[0];
        F=intersectionpoints(circumcircle(A, C, D), A -- B)[1];
        Ap=2*foot(A, B, C)-A;
        Ep=2*foot(EE, B, C)-EE;
        P=extension(C, Ap, D, EE);
        Q=extension(B, Ap, D, F);
        X=extension(D, EE, A, B);
        Y=extension(P, Q, B, C);
        Z=extension(D, F, A, C);
        draw(A -- B -- C -- A, pri+linewidth(1.2)); draw(Y -- B, pri); draw(Z -- A, pri);
        draw(Z -- Ep, pri);
        draw(X -- EE, pri); draw(Q -- Ep, pri);
        draw(P -- C, pri); draw(Q -- Ap, pri);
        filldraw(circumcircle(A, B, D), fil, pri); filldraw(circumcircle(A, C, D), fil, pri);
        draw(A -- Ap, sec); draw(EE -- Ep, sec);
        draw(X -- Z, sec+dashed);
        draw(A -- X, pri); draw(P -- Q, pri);
        dot("$A$", A, N);
        dot("$B$", B, SW);
        dot("$C$", C, SE);
        dot("$D$", D, S);
        dot("$E$", EE, N);
        dot("$F$", F, W);
        dot("$A'$", Ap, S);
        dot("$E'$", Ep, S);
        dot("$P$", P, SE);
        dot("$Q$", Q, N);
        dot("$Z$", X, S);
        dot("$X$", Y, W);
        dot("$Y$", Z, NE);
    \end{asy}
\end{center}
Since $ABDE$ and $ACDF$ are cyclic, \[\measuredangle FDC=\measuredangle FAC=\measuredangle BAC=\measuredangle BAE=\measuredangle BDE=\measuredangle CDE=\measuredangle E'DC,\]
whence $E'$ lies on $\overline{DF}$. Thus, $P$ and $Y$ are reflections across $\overline{BC}$, and so are $Q$ and $Z$. Since $P,X,Q$ are collinear, $X,Y,Z$ are collinear, so $\triangle ABC$ and $\triangle DPQ$ are centrally perspective and by Desargue's Theorem, axially perspective as well.

