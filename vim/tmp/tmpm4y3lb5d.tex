% Input your problem and solution below.
% Three dashes on a newline indicate the breaking points.

---

In triangle $ABC$ the bisector of $\angle BCA$ intersects the circumcircle again at $R$, the perpendicular bisector of $\seg{BC}$ at $P$, and the perpendicular bisector of $\seg{AC}$ at $Q$. The midpoint of $\seg{BC}$ is $K$ and the midpoint of $\seg{AC}$ is $L$. Prove that the triangles $RPK$ and $RQL$ have the same area.

---

Let $O=\seg{KP}\cap\seg{LQ}$ be the circumcenter.
\begin{claim*}
    $CP=RQ$ (and thus $CQ=RP$).
\end{claim*}
\begin{proof}
    Just note that $\angle OPQ=90-\frac12\angle ACB=\angle OQP$, so $OP=OQ$.
\end{proof}
\begin{center}
\begin{asy}
    size(5cm); defaultpen(fontsize(10pt));
    pen pri=blue;
    pen sec=heavygreen;
    pen tri=heavycyan;
    pen fil=cyan+opacity(0.05);
    pen sfil=green+opacity(0.05);
    pen tfil=heavycyan+opacity(0.05);

    pair O,C,A,B,R,K,L,P,Q;
    C=dir(125);
    A=dir(210);
    B=dir(330);
    R=dir(270);
    K=(C+B)/2;
    L=(C+A)/2;
    P=extension(C,R,O,K);
    Q=extension(C,R,O,L);

    filldraw(unitcircle,tfil,tri+dashed);
    draw(C--Q--O,tri);
    filldraw(R--P--K--cycle,sfil,sec);
    filldraw(R--Q--L--cycle,sfil,sec);
    filldraw(C--A--B--cycle,fil,pri);

    dot("$C$",C,C);
    dot("$A$",A,A);
    dot("$B$",B,B);
    dot("$R$",R,R);
    dot("$K$",K,NE);
    dot("$L$",L,W);
    dot("$P$",P,W);
    dot("$Q$",Q,NE);
    dot("$O$",O,N);
\end{asy}
\end{center}

Check that $\angle RPK=\angle RQL$. Furthermore, $\triangle CKP\sim\triangle CLQ$ gives \[\frac{RQ}{PK}=\frac{CP}{PK}=\frac{CQ}{QL}=\frac{RP}{QL},\]
so $RP\cdot PK=RQ\cdot QL$, and we are done.

