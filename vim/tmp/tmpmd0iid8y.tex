% Input your problem and solution below.
% Three dashes on a newline indicate the breaking points.

---

Let $ABC$ be an acute triangle with circumcircle $\Gamma$ and altitudes $\seg{AD}$, $\seg{BE}$, and $\seg{CF}$ meeting at $H$. Let $\omega$ be the circumcircle of $\triangle DEF$. Point $S\ne A$ lies on $\Gamma$ such that $DS=DA$. Line $\seg{AD}$ meets $\seg{EF}$ at $Q$, and meets $\omega$ at $L\ne D$. Point $M$ is chosen such that $\seg{DM}$ is a diameter of $\omega$. Point $P$ lies on $\seg{EF}$ with $\seg{DP}\perp\seg{EF}$. Prove that lines $SH$, $MQ$, and $PL$ are concurrent.

---

\begin{center}
    \begin{asy}
        size(10cm);
        defaultpen(fontsize(10pt));

        pair A, B, C, H, D, EE, F, L, NN, M, P, Q, X, SS;
        A=dir(110);
        B=dir(220);
        C=dir(320);
        H=orthocenter(A,B,C);
        D=foot(A,B,C);
        EE=foot(B,C,A);
        F=foot(C,A,B);
        L=(A+H)/2;
        NN=((0,0)+H)/2;
        M=2NN-D;
        P=foot(D,EE,F);
        Q=extension(A,D,EE,F);
        X=extension(L,P,M,Q);
        SS=intersectionpoint(H--(H+100*(H-X)),circumcircle(A,B,C));

        draw(circumcircle(A,B,C));
        draw(circumcircle(D,EE,F));
        draw(circumcircle(A,H,F),gray);
        draw(circumcircle(A,D,X),gray);
        draw(A--B--C--A);
        draw(A--D,gray);
        draw(B--EE,gray);
        draw(C--F,gray);
        draw(D--EE--F--D);
        draw(L--X--M);
        draw(X--D);
        draw(A--X,dotted);
        draw(X--SS,dashed);

        dot("$A$",A,N);
        dot("$B$",B,SW);
        dot("$C$",C,SE);
        dot("$D$",D,S);
        dot("$E$",EE,E);
        dot("$F$",F,dir(210));
        dot("$L$",L,NW);
        dot("$M$",M,NE);
        dot("$H$",H,SE);
        dot("$P$",P,NW);
        dot("$Q$",Q,NW);
        dot("$X$",X,SW);
        dot("$S$",SS,SS);
    \end{asy}
\end{center}
Let $\seg{LP}$ intersect $\omega$ again at $X$. Since $\seg{DHLA}$ bisects $\angle FDE$, $L$ is the midpoint if $\widehat{EF}$ on $\omega$. Note that inversion at $L$ with power $LE^2=LF^2$ swaps $\seg{EF}$ and $\omega$, so it swaps $(Q,D)$ and $(P,X)$. Thus, $LQ\cdot LD=LE^2=LP\cdot LX$, so $PQDX$ is cyclic and $\angle QXD=\angle QPD=90^\circ=\angle MXD$, and $X\in\seg{MQ}$.

Let $S'$ be the intersection of ray $XH$ and $\Gamma$. It is sufficient to show that $DA=DS'$.
\begin{iclaim*}
    $AXDS'$ is cyclic.
\end{iclaim*}
\begin{proof}
    We present two proofs:
    \begin{itemize}
        \item Notice that the negative inversion at $H$ with power $-AH\cdot HD$ swaps $(A,D)$, $(B,E)$, and $(C,F)$, so it swaps $\Gamma$ and $\omega$. It is immediate that it swaps $S'$ and $X$, so $HX\cdot HS'=HA\cdot HD$, as desired.
        \item Notice that homothety at $H$ sends $D$ and $X$ to points $D'$ and $X'$ on $\Gamma$. Check that \[HA\cdot HD=\tfrac12HA\cdot HD'=\tfrac12HS'\cdot HX'=HS'\cdot HX,\]
            as desired.
    \end{itemize}\
\end{proof}

Since $\angle AEH=90^\circ$ and $\seg{EH}$ bisects $\angle DEQ$, $-1=(DQ;AH)$. However, $\angle DXQ=90^\circ$, so $\seg{XD}$ bisects $\angle AXH$ by a well-known lemma on Apollonius circles. It follows that $D$ is the midpoint of one of the arcs $\widehat{AS'}$ on $(AXDS')$, so $DA=DS'$, and $S=S'$, as desired.
