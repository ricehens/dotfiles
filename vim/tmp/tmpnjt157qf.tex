% Input your problem and solution below.
% Three dashes on a newline indicate the breaking points.

---

Let $ABCD$ be a convex quadrilateral, and let $\omega_A$ and $\omega_B$ be the incircles of $\triangle ACD$ and $\triangle BCD$, with centers $I$ and $J$. The second common external tangent to $\omega_A$ and $\omega_B$ touches $\omega_A$ at $K$ and $\omega_B$ at $L$. Prove that lines $AK$, $BL$, $IJ$ are concurrent.

---

We present three solutions.

\paragraph{First solution, by Iran lemma (Nikolai Beluhov)} Let $S=\seg{KL}\cap\seg{IJ}\cap\seg{CD}$. Let $\omega_A$ touch $\seg{AC}$, $\seg{AD}$ at $W$, $X$ and let $\omega_B$ touch $\seg{BC}$, $\seg{BD}$ at $Y$, $Z$.
\begin{claim*}
    $\seg{KW}$, $\seg{LY}$, $\seg{IJ}$ concur at a point $P$. Similarly, $\seg{KX}$, $\seg{LZ}$, $\seg{IJ}$ concur at a point $Q$.
\end{claim*}
\begin{proof}
    Let $\seg{AC}$, $\seg{BD}$ intersect $\seg{IJ}$ at $U$, $V$. Since $\seg{IJ}$ bisects $\angle S$, the first concurrence follows from Iran lemma on $\triangle USC$, and the second concurrence follows from Iran lemma on $\triangle VXD$.
\end{proof}
\begin{center}
\begin{asy}
    size(10cm); defaultpen(fontsize(10pt));
    pen pri=lightblue;
    pen sec=purple+pink;
    pen tri=lightred;
    pen fil=pri+opacity(0.05);
    pen sfil=sec+opacity(0.05);
    pen tfil=tri+opacity(0.05);

    pair D,C,A,B,I,J,Kp,Lp,K,L,X,P,Q,ID,IC,JD,JC,T;
    D=(-1,0);
    C=(1,0);
    A=1.2*dir(135);
    B=1.3*dir(85);
    I=incenter(A,C,D);
    J=incenter(B,C,D);
    Kp=foot(I,C,D);
    Lp=foot(J,C,D);
    K=reflect(I,J)*Kp;
    L=reflect(I,J)*Lp;
    X=extension(C,D,I,J);
    P=foot(C,I,J);
    Q=foot(D,I,J);
    ID=foot(I,A,C);
    IC=foot(I,A,D);
    JD=foot(J,B,C);
    JC=foot(J,B,D);
    T=extension(A,K,B,L);

    draw(A--T--B,sec+Dotted);
    draw(C--P,sec);
    draw(D--Q,sec);
    draw(P--K--Q,tri);
    draw(P--L--Q,tri);
    filldraw(incircle(A,C,D),fil,pri);
    filldraw(incircle(B,C,D),fil,pri);
    fill(A--D--C--cycle,fil);
    fill(B--D--C--cycle,fil);
    draw(C--B--D--A--C--X--P,pri);
    draw(X--extension(K,L,B,C),pri);

    dot("$A$",A,N);
    dot("$B$",B,N);
    dot("$C$",C,SE);
    dot("$D$",D,S);
    dot("$I$",I,dir(240));
    dot("$J$",J,dir(-30));
    dot("$K$",K,dir(60));
    dot("$L$",L,S);
    dot("$S$",X,SW);
    dot("$P$",P,NE);
    dot("$Q$",Q,NW);
    dot("$W$",ID,N);
    dot("$X$",IC,dir(15));
    dot("$Y$",JD,NE);
    dot("$Z$",JC,NW);
\end{asy}
\end{center}
Let $\seg{AK}$, $\seg{BL}$ meet $\omega_A$, $\omega_B$ again at $K'$, $L'$. Then \[K(PQ;SA)=(WX;KK')=-1=(YZ;LL')=L(PQ;SB),\]
so the desired concurrence follows.

\paragraph{Second solution, by Desargue involution} Let $\omega_A$, $\omega_B$ touch $\seg{CD}$ at $A_0$, $B_0$. The necessary claim is this:
\begin{lemma*}
    Let $PQRS$ be a tangential quadrilateral with incenter $I$. For a point $X$, line $XI$ bisects $\angle PXR$ if and only if line $XI$ bisects $\angle QXS$.
\end{lemma*}
\begin{proof}
    Apply DDIT from $X$ to $PQRS$: if the tangents from $X$ to the incircle (in $\mathbb C\mathbb P^2$) touch the incircle at $U$, $V$, then $(\seg{XP},\seg{XR})$, $(\seg{XQ},\seg{XS})$, $(\seg{XU},\seg{XV})$ are reciprocal pairs of some involution, which must be reflection over line $XI$.
\end{proof}

Let $T$ be the point on line $IJ$ such that $\seg{IJ}$ (externally) bisects $\angle CTD$. By the lemma on degenerate quadrilaterals $ACA_0D$, $BCB_0D$, we find line $IJ$ bisects $\angle ATA_0$, $\angle BTB_0$. Hence $K\in\seg{AT}$, $L\in\seg{BT}$, and we are done.

\paragraph{Third solution, by ELMO proposal} Let $\overline{CD}$ touch $\omega_A$, $\omega_B$ at $A_0$, $B_0$, and let $\overline{AK}$, $\overline{BL}$ intersect $\omega_A$, $\omega_B$ again at $K'$, $L'$.
\begin{claim*}
    The tangents to $\omega_A$, $\omega_B$ at $K'$, $L'$ intersect on $\overline{CD}$.
\end{claim*}
\begin{proof}
    Let $\omega_A$ touch $\overline{AC}$, $\overline{AD}$ at $U_1$, $V_1$, and let $Y=\overline{CD}\cap\overline{K'K'}$. Taking the dual,\[(CD;YX)=A_0(U_1V_1;K'K)=-1.\]
    Similarly $\overline{L'L'}$ intersects $\overline{CD}$ at the harmonic conjugate of $X$ wrt.\ $\overline{CD}$, so the claim follows.
\end{proof}
\begin{center}
\begin{asy}
    size(8cm); defaultpen(fontsize(10pt));
    pen pri=lightblue;
    pen sec=purple+pink;
    pen tri=lightred;
    pen fil=pri+opacity(0.05);
    pen sfil=sec+opacity(0.05);
    pen tfil=tri+opacity(0.05);

    pair D,C,A,B,I,J,Kp,Lp,K,L,X,T,A0,B0,Kp,Lp,Y;
    D=(-1,0);
    C=(1,0);
    A=1.45*dir(140);
    B=1.55*dir(85);
    I=incenter(A,C,D);
    J=incenter(B,C,D);
    Kp=foot(I,C,D);
    Lp=foot(J,C,D);
    K=reflect(I,J)*Kp;
    L=reflect(I,J)*Lp;
    X=extension(C,D,I,J);
    T=extension(A,K,B,L);
    A0=foot(I,C,D);
    B0=foot(J,C,D);
    Kp=2*foot(I,A,K)-K;
    Lp=2*foot(J,B,L)-L;
    Y=extension(C,D,T,T+rotate(90)*(I-J));

    filldraw(circumcircle(I,T,Y),tfil,tri);
    filldraw(circumcircle(J,T,Y),tfil,tri);
    draw(A--Kp--Y--Lp--B,sec);
    filldraw(incircle(A,C,D),fil,pri);
    filldraw(incircle(B,C,D),fil,pri);
    fill(A--D--C--cycle,fil);
    fill(B--D--C--cycle,fil);
    draw(C--B--D--A--C--D,pri);
    draw(extension(I,J,D,D+(-.3,1))--extension(I,J,C,C+(-0.3,1)),pri);
    draw(extension(K,L,D,D+(-.3,1))--extension(K,L,C,C+(-0.3,1)),pri);

    dot("$A$",A,N);
    dot("$B$",B,N);
    dot("$C$",C,SE);
    dot("$D$",D,SW);
    dot("$I$",I,dir(300));
    dot("$J$",J,dir(75));
    dot("$K$",K,N);
    dot("$L$",L,SE);
    dot("$T$",T,N);
    dot("$A_0$",A0,S);
    dot("$B_0$",B0,SE);
    dot("$K'$",Kp,dir(195));
    dot("$L'$",Lp,dir(165));
    dot("$Y$",Y,S);
\end{asy}
\end{center}
Let $T$ be the projection of $Y$ onto $\seg{IJ}$, let $\seg{KL}$, $\seg{IJ}$, $\seg{CD}$ concur at $S$, and let $\seg{YK'}$, $\seg{YL'}$ intersect line $KL$ at $U$, $V$. By the Iran lemma on $\triangle SYU$ and $\triangle SYV$, we have $T\in\seg{KK'}$ and $T\in\seg{LL'}$, as needed.
\begin{remark}
    I (unsuccessfully) proposed the below problem to ELMO 2020. Turns out it's just a corollary of the Iran lemma. The solution above is basically turning the USMCA problem into the ELMO proposal and applying Iran lemma to finish. (Hence the problem only took me somewhere between 10--20 minutes :D.)
    \begin{quote}
        Distinct lines $\ell_1$ and $\ell_2$ are tangent to circle $\omega_1$ at $A_1$ and $A_2$ respectively and circle $\omega_2$ at $B_1$ and $B_2$ respectively. Let $C$ be a point on line $A_1B_1$ distinct from $A_1$ and $B_1$, and let $D\ne A_1$ and $E\ne B_1$ be points on $\omega_1$ and $\omega_2$ respectively such that $\overline{CD}$ is tangent to $\omega_1$ and $\overline{CE}$ is tangent to $\omega_2$. Let lines $DA_2$ and $EB_2$ intersect at $F$. Show that $F$ is equidistant from lines $\ell_1$ and $\ell_2$.
    \end{quote}
\end{remark}

