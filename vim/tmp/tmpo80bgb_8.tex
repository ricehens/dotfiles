% Input your problem and solution below.
% Three dashes on a newline indicate the breaking points.

---

Let $ABC$ be a triangle and let $P$ be a point not lying on any of the sidelines of the triangle. Distinct points $D$, $E$, $F$ lies on lines $BC$, $CA$, $AB$, respectively, such that $\seg{DE}\parallel\seg{CP}$ and $\seg{DF}\parallel\seg{BP}$. Show that there exists a point $Q$ on the circumcircle of $\triangle AEF$ such that $\triangle BAQ$ is similar to $\triangle PAC$.

---

\begin{center}
    \begin{asy}
        size(5cm); defaultpen(fontsize(10pt));
        pair A,B,C,D,EE,F,P,Q,Ep;
        real x,y,z;
        A=dir(110);
        B=dir(210);
        C=dir(330);
        x=0.4;
        y=0.7;
        z=0.4;
        D=x*B+(1-x)*C;
        EE=y*C+(1-y)*A;
        F=z*A+(1-z)*B;
        P=extension(B,B+D-F,C,C+D-EE);
        Q=2*foot(circumcenter(A,EE,F),A,reflect(A,incenter(A,B,C))*P)-A;
        Ep=extension(A,C,B,B+EE-D);

        draw(circumcircle(A,B,Q),gray);
        draw(A--Q,gray);
        draw(EE--D--F,gray);
        draw(circumcircle(A,EE,F));
        draw(A--B--C--A);
        draw(B--P--C);
        draw(B--Ep);

        dot("$A$",A,N);
        dot("$B$",B,SW);
        dot("$C$",C,E);
        dot("$D$",D,S);
        dot("$E$",EE,E);
        dot("$F$",F,W);
        dot("$P$",P,S);
        dot("$Q$",Q,NE);
        dot("$E_0$",Ep,NE);
    \end{asy}
\end{center}
Given point $P$, define $Q$ as the unique point such that $\triangle BAQ$ and $\triangle PAC$ are directly similar. This means $\angle BAQ=\angle PAC$ and $AB\cdot AC=AP\cdot AQ$, so $Q$ is the image of $P$ under inversion at $A$ with radius $\sqrt{AB\cdot AC}$ followed by reflection of the angle bisector. It suffices to show that $Q$ lies on $\triangle AEF$.

Move $D$ along line $BC$ at a linear rate. Since $\seg{DE}$ and $\seg{DF}$ are in fixed directions, $E$ and $F$ also move linearly along $\seg{CA}$ and $\seg{AB}$ respectively. Then as $E$ and $F$ move along their respective sidelines, there is a fixed spiral similarity centered at a point $Q'$ sending $E$ to $F$. By spiral similarity lemma, $Q'$ lies on all circles $(AEF)$. We will show that $Q=Q'$.

It suffices to verify that $Q$ lies on $(AEF)$ for two values of $D$ on $\seg{BC}$. We verify the hypothesis for $D=B$, and the case $D=C$ follows symmetrically, thus concluding the proof. If $E_0$ lies on $\seg{CA}$ such that $\seg{BE_0}\parallel\seg{CP}$, it suffices to show that $Q\in(ABE)$. But since $\triangle BAQ\sim\triangle PAC$ and $\seg{BE_0}\parallel\seg{CP}$, we have $\da AE_0B=\da ACP=\da AQB$, as desired.

