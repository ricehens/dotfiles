% Input your problem and solution below.
% Three dashes on a newline indicate the breaking points.
% vim: tw=72

---

An equilateral pentagon $AMNPQ$ is inscribed in triangle $ABC$ such that $M\in\seg{AB}$, $Q\in\seg{AC}$, and $N,P\in\seg{BC}$. Let $S$ be the intersection of lines $MN$ and $PQ$. Denote by $\ell$ the angle bisector of $\angle MSQ$.

Prove that $\seg{OI}$ is parallel to $\ell$, where $O$ is the circumcenter and $I$ is the incenter of triangle $ABC$.

---

\begin{center}
    \begin{asy}
        size(8cm);
        defaultpen(fontsize(9pt));

        pair NN,P,M,Q,A,B,C,SS,O,I,X,Y,Z;
        NN=(-1/2,0);
        P=(1/2,0);
        M=NN+dir(115);
        Q=P+dir(80);
        A=intersectionpoints(circle(M,1),circle(Q,1))[0];
        B=extension(A,M,NN,P);
        C=extension(A,Q,NN,P);
        SS=extension(M,NN,P,Q);
        O=circumcenter(A,B,C);
        I=incenter(A,B,C);
        X=extension(A,I,O,(B+C)/2);
        Y=extension(B,I,O,(C+A)/2);
        Z=extension(C,I,O,(A+B)/2);

        fill(A--M--NN--P--Q--cycle,lightgray);
        draw(circumcircle(A,B,C));
        draw(A--B--C--A);
        draw(M--SS--Q);
        draw(A--X,dotted);
        draw(B--Y,dotted);
        draw(C--Z,dotted);
        dot("$A$",A,unit(A-O));
        dot("$B$",B,W);
        dot("$C$",C,E);
        dot("$M$",M,dir(120));
        dot("$N$",NN,SW);
        dot("$P$",P,SE);
        dot("$Q$",Q,dir(75));
        dot("$S$",SS,S);
        dot("$O$",O,SE);
        dot("$I$",I,NW);
        dot("$X$",X,unit(X-O));
        dot("$Y$",Y,unit(Y-O));
        dot("$Z$",Z,unit(Z-O));
    \end{asy}
\end{center}
Let $X$, $Y$, $Z$ be the midpoints of arcs $BC$, $CA$, $AB$ not containing $A$, $B$, $C$ on the circumcircle of $\triangle ABC$. Toss on the complex plane, with the circumcircle of $\triangle ABC$ as the unit circle, so that $x+y+z$ denotes the incenter. We just need to show that $x+y+z$ is in the direction perpendicular to the external angle bisector of $\angle MSQ$, but since $MN=PQ$, the external angle bisector of $\angle MSQ$ is just \[(m-n)+(p-q)=(m-a)+(p-n)+(a-q)=ti(x+y+z),\]
where $t=AM=NP=QA$. This completes the proof. 
