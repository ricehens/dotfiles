% Input your problem and solution below.
% Three dashes on a newline indicate the breaking points.

---

Show that the solution set of the inequality
\[\sum^{70}_{k=1}\frac k{x-k}\ge\frac54\]
is a union of disjoint intervals, the sum of whose length is 1988.

---

I will show that the total length of the solution set to
\[f(x):=\sum_{k=1}^m\frac k{x-k}\ge r\]
is given by $\frac1r\frac{m(m+1)}2$.
\begin{claim*}
    $f'(x)<0$, except at the discontinuities $x=1,\ldots,m$.
\end{claim*}
\begin{proof}
    Each addend $\frac k{x-k}$ is strictly decreasing except at the discontinuity $x=k$.
\end{proof}

Therefore, (by intermediate value theorem) the solution set should be of the form $(1,1+\veps_1)\cup(2,2+\veps_2)\cup\cdots\cup(m,m+\veps_m)$ for some $0<\veps_i<1$; here, $1+\veps_1$, \ldots, $m+\veps_m$ are roots of $f(x)=r$. The goal is to evaluate $\veps_1+\cdots+\veps_m$. 

Let $g(x)=(x-1)^1(x-2)^2\cdots(x-m)^m$. It readily follows that
\[f(x)=\frac{g'(x)}{g(x)}.\]
Hence the roots of $g'-rg$ are $1+\veps_1$, \ldots, $m+\veps_m$, and also $k$ with multiplicity $k-1$ for each $k=1,\ldots,m$, so the sum of the roots of $g'-rg$ is $(1^2+\cdots+m^2)+(\veps_1+\cdots+\veps_m)$. We can also use Vieta's to determine the sum of the roots of $g'-rg$.

If $M=\binom{m+1}2$, then
\begin{align*}
    g(x)&=x^M-\left(1^2+\cdots+m^2\right)x^{M-1}+\cdots\\
    g'(x)&=Mx^{M-1}-\cdots,\quad\text{thus}\\
    g'(x)-rg(x)&=-rx^M+x^{M-1}\left[M+r\left(1^2+\cdots+m^2\right)\right]+\cdots
\end{align*}
so the sum of the roots of $g'-rg$ is $(1^2+\cdots+m^2)+\frac Mr$.

It follows that $\veps_1+\cdots+\veps_m=\frac Mr$, which is what we wanted to show.
