% Input your problem and solution below.
% Three dashes on a newline indicate the breaking points.

---

An island has two cities: Mathopolis and Campville. There are three roads connecting the cities: Red Road, Green Road, and Blue Road. The roads can intersect each other, but only at right angles. Also, no road intersects itself. (Roads don't always go left to right --- they can loop around and cross however they want, obeying the restrictions.) All along each road, there are signposts indicating the direction to Campville.

There are six types of intersections, which we divide into two groups:
\begin{center}
    \begin{asy}
        size(6cm); defaultpen(fontsize(11pt));
        picture pic1, pic2;
        real t=0.1;
        label(pic1,"Group A:",(0,0));
        draw(pic1,(2,-1)--(4+t,1+t),heavyred+linewidth(2),ArcArrow(5bp));
        draw(pic1,(4,-1)--(2-t,1+t),heavyblue+linewidth(2),ArcArrow(5bp));
        draw(pic1,(5,-1)--(7+t,1+t),heavyblue+linewidth(2),ArcArrow(5bp));
        draw(pic1,(7,-1)--(5-t,1+t),heavygreen+linewidth(2),ArcArrow(5bp));
        draw(pic1,(8,-1)--(10+t,1+t),heavygreen+linewidth(2),ArcArrow(5bp));
        draw(pic1,(10,-1)--(8-t,1+t),heavyred+linewidth(2),ArcArrow(5bp));
        label(pic2,"Group B:",(0,0));
        draw(pic2,(2,-1)--(4+t,1+t),heavyblue+linewidth(2),ArcArrow(5bp));
        draw(pic2,(4,-1)--(2-t,1+t),heavyred+linewidth(2),ArcArrow(5bp));
        draw(pic2,(5,-1)--(7+t,1+t),heavygreen+linewidth(2),ArcArrow(5bp));
        draw(pic2,(7,-1)--(5-t,1+t),heavyblue+linewidth(2),ArcArrow(5bp));
        draw(pic2,(8,-1)--(10+t,1+t),heavyred+linewidth(2),ArcArrow(5bp));
        draw(pic2,(10,-1)--(8-t,1+t),heavygreen+linewidth(2),ArcArrow(5bp));
        add(shift( (0,1.5))*pic1);
        add(shift( (0,-1.5))*pic2);
    \end{asy}
\end{center}
Let $a$ be the number of intersections in Group A, and $b$ the number in Group B. Determine all possible values of $a-b$.

---

In all the following diagrams, pretend the intersections are all right angles. (It is easy to see that we can change slopes near the point of contact so that they are right angles.)

The answer is $-1$, $0$, $1$. Clearly $0$ is achievable by having no intersections; here is a construction for $1$, which if we reflect across the line connecting Mathopolis and Campville will give a construction for $-1$.
\begin{center}
    \begin{asy}
        size(4cm); defaultpen(fontsize(10pt));
        path r=(0,0)..(5,3)..(10,0);
        path b=(0,0)..(5,-3)..(10,0);
        path g=(0,0)..(4,5)..(6,1)..(10,0);
        draw(r,heavyred+linewidth(1));
        draw(b,heavyblue+linewidth(1));
        draw(g,heavygreen+linewidth(1));
        dot("$M$",(0,0),SW);
        dot("$C$",(10,0),E);
    \end{asy}
\end{center}

Now we will prove $a-b\in\{-1,0,1\}$.
\begin{center}
    \begin{asy}
        size(6cm); defaultpen(fontsize(10pt));
        path r=(0,0)..(5,-2)..(10,0)..(15,2)..(20,0);
        path b=(0,0)..(8,6)..(16,0)..(10,0)..(4,0)..(12,-6)..(20,0);
        draw(r,heavyred+linewidth(1));
        draw(b,heavyblue+linewidth(1));
        dot("$M$",(0,0),W);
        dot("$C$",(20,0),E);
        label("\tiny A",(8,6),S);
        label("\tiny A",(16,0),4*dir(195));
        label("\tiny A",(10,0),8*dir(155));
        label("\tiny A",(12,-7.2),N);
        label("\tiny B",(8,7.2),S);
        label("\tiny B",(16,-1),4*dir(195));
        label("\tiny B",(10,-1),8*dir(155));
        label("\tiny B",(12,-6),N);

        label("\tiny A",(0,0),4*dir(-25));
        label("\tiny A",(5,-2),NE);
        label("\tiny A",(15,2),NW);
        label("\tiny A",(20,1),4*dir(150));
        label("\tiny B",(0,-1.2),4*dir(-25));
        label("\tiny B",(5,-3),NE);
        label("\tiny B",(15,1),NW);
        label("\tiny B",(20,-0.2),4*dir(150));
    \end{asy}
\end{center}
(In the diagram above, a letter on a side of a curve indicates the type of intersection if Green Road crosses the road from the respective side.)

We can treat Red Road and Blue Road as a giant loop $\omega$ that consists of first going from $M$ to $C$ along the direction of Red Road, and then returning from $C$ to $M$ along Blue Road in the opposite direction. Then each side of $\omega$ has a fixed letter. For any curve $\gamma$, let $f(\gamma)$ be the number of $A$'s it crosses minus the number of $B$'s it crosses.
\setcounter{iclaim}0
\begin{iclaim}
    For any simple loop $\gamma$, $f(\gamma)=0$.
\end{iclaim}
\begin{proof}
    By the Jordan curve theorem $\gamma$ has an interior and an exterior. Thus $\omega$ passes into the interior of $\gamma$ and leaves the interior of $\gamma$ equally often.

    However every time $\omega$ leaves $\gamma$, the intersection is in the same group (without loss of generality, it is $A$), and every time $\omega$ enters the interior of $\gamma$, the intersection is in group $B$. Thus $f(\gamma)=0$, as desired.
\end{proof}
\begin{iclaim}
    For any loop $\gamma$, $f(\gamma)=0$.
\end{iclaim}
\begin{proof}
    If $\gamma$ contains a simple loop $\gamma_1$, then $f(\gamma_1)=0$ by Claim 1, so we may delete $\gamma_1$ from $\gamma$ without changing $f(\gamma)$.

    Repeat this process until $\gamma$ becomes a simple loop. The claim follows from Claim 1.
\end{proof}
\begin{iclaim}
    For any curves $\gamma_1$ and $\gamma_2$ starting and ending at the same points, we have $f(\gamma_1)=f(\gamma_2)$.
\end{iclaim}
\begin{proof}
    Define the loop $\gamma$ as follows: traverse along $\gamma_1$, and return to the starting position by traversing along $\gamma_2$ in the opposite direction. Then $0=f(\gamma)=f(\gamma_1)-f(\gamma_2)$, and the claim is immediate.
\end{proof}

As shown in the diagram above, Green Road can be considered as a curve $\gamma$ starting from either one of the two adjacent regions whose boundaries contain $M$ (call the regions $M_1$ and $M_2$) and ending at either one of the two adjacent regions whose boundaries contain $C$ (call the regions $C_1$ and $C_2$). For convenience, let $M_iC_j$ be any curve starting in $M_i$ and ending in $C_j$ (by Claim 3, $f(M_iC_j)$ is fixed no matter which curve we choose).

Let $x$ be the contribution to $a-b$ by intersections between Red Road and Blue Road. We want to prove that no matter what Green Road $\gamma$ we choose, $f(\gamma)\in\{-x-1,-x,-x+1\}$.

Consider a possible Green Road that starts on the side of Blue Road labeled $A$, stays sufficiently close to Blue Road, but never intersects Blue Road. Each intersection between Blue Road and Red Road corresponds to an intersection between Green Road and Red Road, so choosing this Green Road gives $a-b=0$. Without loss of generality this road is $M_1C_1$, so $f(M_1C_1)=-x$.
\begin{center}
    \begin{asy}
        size(6cm); defaultpen(fontsize(10pt));
        path r=(0,0)..(5,-2)..(10,0)..(15,2)..(20,0);
        path b=(0,0)..(8,6)..(16,0)..(10,0)..(4,0)..(12,-6)..(20,0);
        path g=(0,0)..(8,5)..(15,0)..(10,1)..(3,0)..(12,-7)..(20,0);
        draw(r,heavyred+linewidth(1));
        draw(b,heavyblue+linewidth(1));
        draw(g,heavygreen+linewidth(1));
        dot("$M$",(0,0),W);
        dot("$C$",(20,0),E);
    \end{asy}
\end{center}


Consider another Green Road defined similarly to above, except we stay on the side of Blue Road labeled $B$. This must be $M_2C_2$, and $f(M_2C_2)=-x$ as well.

Finally regions $M_1$, $M_2$ are adjacent, so we have $f(M_2C_1)=f(M_1C_1)\pm1$; analogously we have $f(M_1C_2)=f(M_1C_1)\pm1$, so $f(M_iC_i)\in\{-x-1,-x,-x+1\}$, and we are done.

