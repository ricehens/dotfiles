% Input your problem and solution below.
% Three dashes on a newline indicate the breaking points.

---

Find the smallest positive integer $n$ such that $n^3+5n^2+2n+2$ is divisible by $1000$. 

---

By inspection, $n\equiv 3\pmod{8}$. Finding $n$ modulo $125$ would be quite tedious. Instead, we will find $n$ mod $5$, and \enquote{lift} the modulus to $125$. By inspection, $n\equiv 1,3\pmod{5}$. We can split this into two cases: $n\equiv 1$ and $n\equiv 3$ modulo $5$.
\begin{itemize}
\item \textit{Case 1:} $n\equiv 1\pmod{5}$. Then, let $n\equiv 5u+1\pmod{25}$. It follows that $(5u+1)^3+5(5u+1)^2+2(5u+1)+2\equiv 0\pmod{25}$. Notice that when we expand, many terms are divisible by $25$. It follows that $15u+1+5+10u+2+2\equiv 0$, or $10\equiv 0\pmod{25}$. This is obviously false, so there are no solutions for $n\equiv 1\pmod{5}$.
\item \textit{Case 2:} $n\equiv 3\pmod{5}$. Then, let $n\equiv 5u+3\pmod{25}$. It follows that $(5u+3)^3+5(5u+3)^2+2(5u+3)+2\equiv 0\pmod{25}$. Expanding gives $135u+27+45+10u+6+2\equiv 0$, or $20u+5\equiv 0\pmod{25}$. Dividing by $5$ gives $4u+1\equiv 0\pmod{5}$, or $u\equiv 1\pmod{5}$.
\end{itemize}
It follows that $n\equiv 8\pmod{25}$. Now we can let $n\equiv 25v+8$. It follows that $(25v+8)^3+5(25v+8)^2+2(25v+8)+2\equiv 0\pmod{125}$. Expanding gives $4800v+512+320+50v+16+2\equiv 0$, or $100v+100\equiv 0\pmod{125}$. Dividing by $25$ gives $4v+4\equiv 0\pmod{5}$, or $v\equiv 4\pmod{5}$. It follows that $n\equiv 108\pmod{125}$. Since $n\equiv 3\pmod{8}$, the Chinese Remainder Theorem gives $n\equiv 483\pmod{1000}$, and we are done.

---

483
