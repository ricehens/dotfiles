% Input your problem and solution below.
% Three dashes on a newline indicate the breaking points.

---

Let $ABC$ be a triangle with circumcircle $\Omega$, circumcenter $O$, and orthocenter $H$, and let $K$ be the midpoint of arc $BAC$ on $\Omega$. Suppose the line through $O$ parallel to $\overline{AK}$ meets $\overline{AB}$ and $\overline{AC}$ at $X$ and $Y$ respectively, and let $M$ and $N$ be the reflections of $X$ and $Y$ over the midpoints of $\overline{AB}$ and $\overline{AC}$ respectively. Show that $KMHN$ is a rhombus.

---

\begin{customenv}{First solution, by spiral similarity}
    Let $L$ be the midpoint of the arc $BC$ of $\Omega$ not containing $A$, and let $P$ denote the orthocenter of $\triangle AXY$. We will show that $KMHN\sim LYAX$, which is obviously sufficient.
    \begin{center}
        \begin{asy}
            size(10cm);
            defaultpen(fontsize(10pt));
            pen fil=opacity(0.02);

            pair A, B, C, O, H, L, K, I, X, Y, P, M, NN, Mp, Np;
            A=dir(120);
            B=dir(212);
            C=dir(328);
            O=(0,0);
            H=A+B+C;
            L=dir(270);
            K=dir(90);
            I=(A+L)/2;
            X=extension(O,O+A-K,A,B);
            Y=extension(O,O+A-K,A,C);
            P=orthocenter(A,X,Y);
            M=A+B-X;
            NN=A+C-Y;
            Mp=(L+M)/2;
            Np=(L+NN)/2;

            filldraw(circle(O,1),red+fil,red);
            filldraw(circumcircle(L,X,Y),orange+fil,orange);
            filldraw(circumcircle(A,M,K),springgreen+fil,green);
            draw(A--B--C--A,red);
            draw(H--A--L--K,orange);
            draw(X--Y,orange);
            draw(P--Y--L--X,fuchsia);
            draw(M--H--NN--K--M--NN,royalblue);
            draw(Mp--Np, royalblue+dashed);
            draw(B--Y,purple);
            draw(C--X,purple);
            draw(M--L--NN, deepblue+dashed);
            draw(H--K,deepgreen+Dotted);

            dot("$A$",A,NW);
            dot("$B$",B,B);
            dot("$C$",C,C);
            dot("$O$",O,dir(140));
            dot("$X$",X,W);
            dot("$Y$",Y,NE);
            dot("$L$",L,S);
            dot("$K$",K,N);
            dot("$H$",H,W);
            dot("$P$",P,dir(150));
            dot("$M$",M,W);
            dot("$N$",NN,dir(-5));
            dot("$M'$",Mp,dir(170));
            dot("$N'$",Np,NE);
        \end{asy}
    \end{center}
    \setcounter{iclaim}0
    \begin{iclaim}
        $A$, $K$, $M$, $N$, $P$ are concyclic, and $\overline{KP}\perp\overline{MN}$.
    \end{iclaim}
    \begin{proof}
        Since $BM=AX=AY=CN$, by SAS, $\triangle KMB\cong\triangle KNC$. It follows that $KM=KN$, and furthermore $K$ is the center of spiral similarity sending $\overline{BM}$ to $\overline{CN}$. In particular, $\measuredangle MKN=\measuredangle BKC=\measuredangle BAC=\measuredangle YLK$, thus $\triangle KMN\sim\triangle LYX$.

        By properties of the Miquel point, $K$ lies on $(AMN)$, and since $\angle KAP=90^\circ$ and $\overline{AP}$ bisects $\angle MAN$, $P$ also lies on $(AMN)$. Since $KM=KN$, $\overline{KP}\perp\overline{MN}$, as desired.
    \end{proof}
    \begin{iclaim}
        $H$, $P$, $K$ are collinear.
    \end{iclaim}
    \begin{proof}
        Let $M'$ and $N'$ be the midpoints of $\overline{BY}$ and $\overline{CX}$ respectively. Since $\overline{BM}\parallel\overline{LY}$ and $BM=AX=LY$, $BMYL$ is a parallelogram. Similarly, $CNXL$ is a parallelogram, so $\overline{M'N'}$ is the $L$-midsegment of $\triangle LMN$.

        Recall that $\overline{KP}\perp\overline{MN}$. However, $\overline{MN}$ is parallel to $\overline{M'N'}$, the Gauss line of $BCYX$. Since $P$ is the orthocenter of $\triangle AXY$, $H$ lies on $\overline{KP}$, as desired.
    \end{proof}

    Since $\overline{AH}\parallel\overline{KL}$, by homothety at $P$, $\triangle PAH\sim\triangle PLK$. It follows that the spiral similarity at $P$ sending $LYPX$ to $KMPN$ sends $A$ to $H$, so $KMHN\sim LYAX$, and we are done.
\end{customenv}
\newpage
\begin{customenv}{Second solution, by complex numbers}
    First we claim that $KM=KN$. Indeed, since $BM=AX=AY=CN$, by SAS, $\triangle KMB\cong\triangle KNC$. Thus it suffices to show that if the perpendicular bisector of $\overline{HK}$ intersects $\overline{AB}$ at $M$ and $\overline{AC}$ at $N$, and $X$ and $Y$ are the reflections of $M$ and $N$ over the midpoints of their respective sides, then $\overline{XO}\parallel\overline{AK}$ (and similarly $\overline{YO}\parallel\overline{AK}$).

    Invoke complex numbers with $\Omega$ as the unit circle. Set $a=u^2$, $b=v^2$, and $c=w^2$, so that $k=vw$ and $h=u^2+v^2+w^2$. We are given that $m=u^2+v^2-x$ is equidistant from $H$ and $K$.

    Since $X$ lies on $\overline{AB}$, $x+u^2v^2\overline x=u^2+v^2$, so $\overline x=\tfrac{u^2+v^2-x}{u^2v^2}$. Since $x+h-u^2-v^2$ and $x+k-u^2-v^2$ have the same magnitude, and $h-u^2-v^2=w^2$,
    \begin{align*}
        \frac{x+w^2}{x+vw-u^2-v^2}&=\overline{\left(\frac{x+vw-u^2-v^2}{x+w^2}\right)}\\
        &=\frac{\left(\frac{u^2+v^2-x}{u^2v^2}\right)+\frac1{vw}-\frac1{u^2}-\frac1{v^2}}{\left(\frac{u^2+v^2-x}{u^2v^2}\right)-\frac1{w^2}}\\
        &=\frac{xw^2-u^2vw}{xw^2+u^2v^2-u^2w^2-v^2w^2}.
    \end{align*}
    This simplifies to
    \begin{align*}
        &x(u^2v^2-u^2w^2-v^2w^2+w^4)+w^2(u^2v^2-u^2w^2-v^2w^2)\\
        &\hspace{5em}=x(vw^3-u^2w^2-v^2w^2-u^2vw)-u^2vw(uw-u^2-v^2)
    \end{align*}
    Rearranging, \[x=\frac{u^2w^4+v^2w^4+u^4vw+u^2v^3w}{u^2v^2+w^4-vw^3+u^2vw}=\frac{w(u^2+v^2)(w^3+u^2v)}{(w-v)(w^3+u^2v)}=\frac{w(u^2+v^2)}{w-v}.\]
    Finally, \[\overline{\left(\frac x{u^2-vw}\right)}=\frac{\tfrac1w\left(\frac1{u^2}+\frac1{v^2}\right)}{\left(\frac1w-\frac1v\right)\left(\frac1{u^2}-\frac1{vw}\right)}=\frac{\frac1{u^2v^2w^2}v\left(u^2+v^2\right)}{\frac1{u^2v^2w^2}(w-v)(u^2-vw)}=\frac x{u^2-vw},\]
    and we are done.
\end{customenv}

