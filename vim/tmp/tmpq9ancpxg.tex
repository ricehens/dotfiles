% Input your problem and solution below.
% Three dashes on a newline indicate the breaking points.

---

Let the excircle of triangle $ABC$ opposite the vertex $A$ be tangent to the side $BC$ at the point $A_1$. Define the points $B_1$ on $\seg{CA}$ and $C_1$ on $\seg{AB}$ analogously, using the excircles opposite $B$ and $C$, respectively. Suppose that the circumcenter of triangle $A_1B_1C_1$ lies on the circumcircle of triangle $ABC$. Prove that triangle $ABC$ is right-angled.

---

\begin{center}
    \begin{asy}
        size(8cm);
        defaultpen(fontsize(10pt));
        pair A,B,C,MA,MB,MC,I,IA,IB,IC,A1,B1,C1;
        real r=110;
        A=dir(r);
        B=dir(180);
        C=dir(0);
        MA=dir(90);
        MB=-dir(r/2);
        MC=-dir(90+r/2);
        I=incenter(A,B,C);
        IA=-2*MA-I;
        IB=-2*MB-I;
        IC=-2*MC-I;
        A1=foot(IA,B,C);
        B1=foot(IB,C,A);
        C1=foot(IC,A,B);
        
        draw(circumcircle(A,B1,C1),gray);
        draw(circumcircle(B,C1,A1),gray);
        draw(circumcircle(C,A1,B1),gray);
        draw(MB--MA--MC,gray);
        draw(MB--IC--IB--MC,gray);
        draw(circumcircle(A,B,C));
        draw(A--B--C--A);
        draw(A1--B1--C1--A1);

        dot("$A$",A,A);
        dot("$B$",B,dir(195));
        dot("$C$",C,dir(-15));
        dot("$M_A$",MA,MA);
        dot("$M_B$",MB,MB);
        dot("$M_C$",MC,MC);
        dot("$I_B$",IB,NE);
        dot("$I_C$",IC,dir(150));
        dot("$A_1$",A1,SW);
        dot("$B_1$",B1,NE);
        dot("$C_1$",C1,dir(290));
    \end{asy}
\end{center}
Let $M_A$, $M_B$, $M_C$ be the midpoints of arcs $CAB$, $ABC$, and $BCA$, respectively, and let $I_A$, $I_B$, $I_C$ be the $A$-, $B$-, and $C$-excenters. Furthermore let $X$ be the circumcenter of $\triangle A_1B_1C_1$.
\begin{boxlemma*}
    For any triangle $ABC$, $M_A$ lies on $(AB_1C_1)$, and $M_AB_1=M_AC_1$.
\end{boxlemma*}
\begin{proof}
    Since $M_AB=M_AC$, $BC_1=CB_1$, and $\da M_ABC_1=\da M_ABA=\da M_ACA=\da M_ACB_1$, by SAS $\triangle M_ABC_1\cong\triangle M_ACB_1$. It follows that $M_A$ is the Miquel point of $BCB_1C_1$, so the lemma is clearly true.
\end{proof}
\begin{iclaim*}
    $X\in\{M_A,M_B,M_C\}$.
\end{iclaim*}
\begin{proof}
    Assume otherwise. Then by our lemma, $\seg{XM_A}\perp\seg{B_1C_1}$, $\seg{XM_B}\perp\seg{C_1A_1}$, and $\seg{XM_C}\perp\seg{A_1B_1}$, so $-\da M_CM_AM_B=\da M_BXM_C=\da C_1A_1B_1$. Hence $\triangle M_AM_BM_C\sim\triangle A_1B_1C_1$, which is acute. This implies that $X$ lies strictly within $\triangle A_1B_1C_1$, and therefore cannot lie on $(ABC)$, contradiction.
\end{proof}

Without loss of generality $X=M_A$. Then $\seg{M_AM_B}\perp\seg{A_1C_1}$ and $\seg{M_AM_C}\perp\seg{A_1B_1}$, so \[\angle A=\angle B_1M_AC_1=360\dg-2\angle B_1A_1C_1=2\angle M_BM_AM_C=180\dg-\angle A,\]
id est $\angle A=90\dg$, as desired.
