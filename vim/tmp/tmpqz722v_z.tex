% Input your problem and solution below.
% Three dashes on a newline indicate the breaking points.

---

Let $\alpha\ge1$ be a real number. Hephaestus and Poseidon play a turn-based game on an infinite grid of unit squares. Before the game starts, Poseidon chooses a finite number of cells to be \emph{flooded}. Hephaestus is building a \emph{levee}, which is a subset of unit edges of the grid (called \emph{walls}) forming a connected, non-self-intersecting path or loop\footnote{More formally, there must exist lattice points $A_0$, $A_1$, $\ldots$, $A_k$, pairwise distinct except possibly $A_0=A_k$, such that the set of walls is exactly $\{A_0A_1,A_1A_2,\ldots,A_{k-1}A_k\}$. Once a wall is built it cannot be destroyed; in particular, if the levee is a closed loop (i.e.\ $A_0=A_k$) then Hephaestus cannot add more walls. Since each wall has length $1$, the length of the levee is $k$.}.

The game begins with Hephaestus moving first. On each of Hephaestus's turns, he adds one or more walls to the levee, as long as the total length of the levee is at most $\alpha n$ after his $n$th turn. On each of Poseidon's turns, every cell which is adjacent to an already flooded cell and with no wall between them becomes flooded as well.

Hephaestus wins if the levee forms a closed loop such that all flooded cells are contained in the interior of the loop \textemdash\ hence stopping the flood and saving the world. For which $\alpha$ can Hephaestus guarantee victory in a finite number of turns no matter how Poseidon chooses the initial cells to flood?

---

The answer is $\alpha>2$.
\begin{customenv}{Proof of lower bound}
    Suppose Poseidon chooses a single unit square to begin flooded. Assume Hephaestus can contain the flood; we will show $\alpha>2$. (In the figure below, blue denotes the flooded region.)
    \begin{center}
        \begin{asy}
            size(2.5cm); defaultpen(fontsize(10pt));
            pen pri=black+linewidth(2);
            pen sec=black;
            pen fil=royalblue;
            pen sfil=springgreen;

            filldraw( (0,0)--(4,0)--(4,4)--(0,4)--cycle,sfil,pri);
            fill( (0,0)--(1,0)--(1,1)--(2,1)--(2,2)--(3,2)--(3,3)--(4,3)--(4,4)--(2,4)--(2,3)--(0,3)--cycle,fil);
            for (real i=0; i<=4+1e-9; i=i+1) {
                draw( (0,i)--(4,i),sec);
                draw( (i,0)--(i,4),sec);
            }
            draw( (0,0)--(1,0)--(1,1)--(2,1)--(2,2)--(3,2)--(3,3)--(4,3)--(4,4)--(2,4)--(2,3)--(0,3)--cycle,pri);
        \end{asy}
    \end{center}
    We say the \emph{taxicab distance} between $(x_1,y_1)$ and $(x_2,y_2)$ is $|x_2-x_1|+|y_2-y_1|$, and the \emph{taxicab diameter} $D$ of the (contained) region is the maximum taxicab distance between two cells of the flooded region.

    Note that it took at most $D$ moves for the flooded region to take its current shape. This is because at any point, the diameter of the flooded region increases by $1$, or is forever stuck. Given that $D$ moves have passed, the perimeter $P$ of the levee must be at most $\alpha D$.

    It is easy to see that the minimum perimeter of a region containing the flood, all of whose sides are either horizontal or vertical, is achieved by the smallest rectangle containing the flood. The perimeter of this minimal rectangle is $2(D+1)$, so we have \[2(D+1)\le\alpha D\implies\alpha\ge2\left(1+\frac1D\right)>2,\]
    as claimed.
\end{customenv}
\begin{customenv}{Proof of sufficiency}
    Take some $\alpha>2$. We show it is possible to contain the flood. Our strategy is as follows.
    \begin{center}
        \begin{asy}
            size(7cm); defaultpen(fontsize(12pt));
            pen pri=black+linewidth(2);
            pen sec=black+linewidth(1);
            pen fil=royalblue;
            pair lbl=2*S;
            real r=1/2;
            pair A,B,C,D,O,del;

            del=(0,0);
            A=(0,0)+del; B=(0,4)+del; C=(6,4)+del; D=(6,0)+del; O=(A+C)/2;
            draw(A--B--C--D--cycle,pri);
            filldraw(circle(O,r),fil);
            draw( (A+(1,1))--(B+(1,-1)),sec);
            label("Step I: Build",D--A,lbl);

            del=(8,0);
            A=(0,0)+del; B=(0,4)+del; C=(6,4)+del; D=(6,0)+del; O=(A+C)/2;
            draw(A--B--C--D--cycle,pri);
            filldraw(circle(O,r),fil);
            draw( ( (A+D)/2+(r,1))--(A+(1,1))--(B+(1,-1))--( (B+C)/2+(r,-1)),sec);
            label("Step II: Engulf",D--A,lbl);

            del=(0,-6);
            A=(0,0)+del; B=(0,4)+del; C=(6,4)+del; D=(6,0)+del; O=(A+C)/2;
            draw(A--B--C--D--cycle,pri);
            filldraw(circle(O,r),fil);
            draw( (D+(-1,1))--(A+(1,1))--(B+(1,-1))--(C+(-1,-1)),sec);
            label("Step III: Zoom",D--A,lbl);

            del=(8,-6);
            A=(0,0)+del; B=(0,4)+del; C=(6,4)+del; D=(6,0)+del; O=(A+C)/2;
            draw(A--B--C--D--cycle,pri);
            filldraw(circle(O,r),fil);
            draw( (D+(-1,1))--(A+(1,1))--(B+(1,-1))--(C+(-1,-1))--cycle,sec);
            label("Step IV: Eat",D--A,lbl);
        \end{asy}
    \end{center}
    \begin{enumerate}[label=\Roman*.]
        \item \textbf{Build a giant wall.} The total vertical height of the flood changes by at most $2$ a move. Start by building a wall sufficiently far away of arbitrary height. Since $\alpha>2$, the wall can be arbitrarily tall compared to the flood, while remaining a constant distance away from the center of the flood.
        \item \textbf{Engulf the flood.} After the wall is sufficiently large, begin constructing walls rightward until the rightmost point on our walls is to the right of the rightmost point of the flood.

            The flood moves rightward at a rate of at most $1$ cell per move, while we can alternate between extending the top wall and the bottom wall, each increasing at a rate of $\alpha/2>1$ cells per move. If the original wall was large enough, the wall can extend past the flood without colliding with it.
        \item \textbf{Zoom past the flood.} Now, we essentially repeat the above process. The wall can be built rightward at a rate of $\alpha/2>1$, so we may extend an arbitrarily large distance past the rightmost point of the flood.
        \item \textbf{Eat the flood.} Finally, build the eastern wall. If we have ``zoomed'' sufficiently far past the flood, we can contain the entire flood, thus completing the process.
    \end{enumerate}
    This concludes the proof.
\end{customenv}

