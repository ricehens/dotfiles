% Input your problem and solution below.
% Three dashes on a newline indicate the breaking points.

---

Let $I_B$ be the $B$-excenter of $\triangle ABC$ and $\omega$ the circumcircle. Let $M$ be the midpoint of the arc $BC$ of $\omega$ not containing $A$. Line $MI_B$ meets $\omega$ again at $T$. Show that $TB\cdot TC=TI_B^2$.

---

First observe $\da BTI_B=\da I_BTC$ since $\seg{TM}$ bisects $\angle BTC$.
\begin{center}
    \begin{asy}
        size(6cm); defaultpen(fontsize(10pt));

        pair B,A,C,I,M,NN,IB,T,X;
        B=dir(110);
        A=dir(210);
        C=dir(330);
        I=incenter(A,B,C);
        M=extension( (B+C)/2,origin,A,I);
        NN=extension( (C+A)/2,origin,B,I);
        IB=2NN-I;
        T=2*foot(origin,M,IB)-M;
        X=2*foot(origin,IB,C)-C;

        draw(circumcircle(A,C,IB),gray);
        draw(M--A,gray);
        draw(B--IB,dashed);
        draw(C--IB,dashed);
        draw(unitcircle);
        draw(A--B--C--cycle);
        draw(M--IB);

        dot("$A$",A,A);
        dot("$B$",B,B);
        dot("$C$",C,C);
        dot("$M$",M,M);
        dot("$N$",NN,SW);
        dot("$I_B$",IB,S);
        dot("$T$",T,SE);
        dot("$I$",I,NW);
    \end{asy}
\end{center}

In addition,
\[\da I_BTC=\da MTC=\da IAC=\da BI_BC=\da BI_BT+\da TI_BC,\]
thus $\da BI_BT=\da I_BTC+\da CI_BT=\da I_BCT$, so 
\[\triangle TBI_B\sim\triangle TI_BC.\]
It follows that $TB\cdot TC=TI_B^2$, as needed.
