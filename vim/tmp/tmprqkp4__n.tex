% Input your problem and solution below.
% Three dashes on a newline indicate the breaking points.

---

A set of lines in the plane is in general position if no two are parallel and no three pass through the same point. A set of lines in general position cuts the plane into regions, some of which have finite area; we call these its finite regions. Prove that for all sufficiently large $n$, in any set of $n$ lines in general position it is possible to colour at least $\sqrt n$ lines blue in such a way that none of its finite regions has a completely blue boundary.

---

In-contest, results with $\sqrt n$ replaced by $c\sqrt n$ will be awarded points depending on the value of the constant $c$.

The first two solutions will prove the result for certain small values of $c$ for partial credit. Then the third solution will solve the problem as stated for $c=1$, and finally the fourth solution will prove a stronger bound of $O(\sqrt{n\log n})$.

\paragraph{First probabilistic solution, for $\frac12\sqrt n$ lines} We will actually achieve $c=(2/3)^{3/2}$. We directly apply the probabilistic method, coloring each line blue with probability.

The number of polygonal regions is easily computed to be $\binom{n-2}2<n^2/2$, and each region has at least three sides, so the probability its boundary is completely blue is at most $p^3$. Hence,
\begin{align*}
    \mathbb E[\#\text{ blue lines}]&=pn\\
    \mathbb E[\#\text{ blue regions}]&<p^3n^2/2.
\end{align*}

By selecting $p=\sqrt{2/(3n)}$, we have
\[
    \mathbb E[\#\text{ blue lines}-\#\text{ blue regions}]>n\sqrt{\frac2{3n}}-\frac{n^2}2\frac2{3n}\sqrt{\frac2{3n}}=\left(\frac23\right)^{3/2}\sqrt n,
\]
so some configuration has $\#\text{ blue lines}-\#\text{ blue regions}$ greater than $(2/3)^{3/2}\sqrt n$. Uncolor a blue line from each blue region, so the resulting figure has no blue regions but $(2/3)^{3/2}\sqrt n$ blue lines, as needed.

\paragraph{Second probabilistic solution, for $\frac23\sqrt n$ lines (Evan Chen)} We will strengthen the above argument to distinguish triangles and other polygons. Again color each line blue with probability $p$.
\begin{claim*}
    There are at most $n^2/3$ triangles.
\end{claim*}
\begin{proof}
    Of the less than $n^2/2$ intersection points, each is clearly a vertex of at most two triangles, so double counting gives $n^2/3$ triangles as an upper bound.
\end{proof}

We can count
\begin{align*}
    \mathbb E[\#\text{ blue lines}]&=pn\\
    \mathbb E[\#\text{ blue triangles}]&<p^3n^2/3\\
    \mathbb E[\#\text{ blue non-triangular regions}]&<p^4n^2/2.
\end{align*}
Hence, we obtain
\[\mathbb E[\#\text{ blue lines}-\#\text{ blue regions}]>np-\frac{n^2p^3}3-\frac{n^2p^4}2.\]
Select $p=n^{-1/2}$ to obtain a lower bound of
\[\mathbb E>\frac23\sqrt n-\text{tiny},\]
and finish as above.

\paragraph{Third greedy solution, for $\sqrt n$ lines (official)} Let lines not colored blue be colored red, and let the intersection of two blue lines be a \emph{blue point}. Call a polygon with exactly one red side a \emph{bruise}.

Consider the configuration with the maximal number of lines; let this maximum number of lines be $k$. We will show $k\ge\sqrt n$. Since $k$ is maximal, for every red line $\ell$, we can find a bruise that has $\ell$ as a side. (Otherwise, we could color $\ell$ blue.)

So for each red line $\ell$, select a bruise, and let its \emph{representative} be the first vertex of the bruise when traversing counterclockwise from the red side. Evidently each red line's representative is a blue point. (The representative of $\ell$ is highlighted green below.)
\begin{center}
\begin{asy}
    size(2cm); defaultpen(fontsize(10pt));
    fill(dir(120)--dir(180)--dir(240)--dir(300)--dir(0)--dir(60)--cycle,purple+pink+white+white+white);
    draw(dir(60)--dir(120),red);
    label("$\ell$",dir(60)--dir(120),N);
    draw(dir(120)--dir(180)--dir(240)--dir(300)--dir(0)--dir(60),blue);
    for (int i=1; i<6; i+=1) dot(dir(60*i));
    dot( (1,0),heavygreen);
\end{asy}
\end{center}

\begin{claim*}
    Every blue point is the representative of at most two red lines.
\end{claim*}
\begin{proof}
    Let the two blue lines be $\ell_1$, $\ell_2$, and let $P$ be their intersection. I claim that the below picture is impossible: two vertical angles cannot simultaneously be representatives of distinct red lines. Assume for contradiction so.
    \begin{center}
    \begin{asy}
        size(5cm); defaultpen(fontsize(10pt));
        fill( (0,0)--(.8,0)--.8*dir(320)--.8*dir(280)--cycle,green+white+white+white);
        fill( (0,0)--.6*dir(280)--.6*dir(230)--(-.6,0)--cycle,lightred+white+white+white);
        draw( (.8,0)--.8*dir(320)--.8*dir(280),heavygreen);
        draw( .6*dir(280)--.6*dir(230)--(-.6,0),lightred);

        draw( (-1,0)--(1,0),blue);
        draw( -.2*dir(280)--dir(280),blue);

        draw(.6*dir(280)--(2*extension(.6*dir(280),.6*dir(230),.8*dir(280),.8*dir(320))-.6*dir(280)),lightred+dashed);

        label("$\ell$",(2*extension(.6*dir(280),.6*dir(230),.8*dir(280),.8*dir(320))-.6*dir(280)),S);
        label("$\ell_1$",(-1,0),NW);
        label("$\ell_2$",dir(280),SW);

        label("$P$",(0,0),NE);
    \end{asy}
    \end{center}
    By definition, the point nearest $P$ on $\ell_2$ is part of the bottom-left bruise and lies on a red line $\ell$; since it is closest to the intersection, it is part of the bottom-right bruise as well. Thus, $\ell$ is also part of the bottom-right bruise, so both bruises represent the same line.
\end{proof}

There are $k$ blue lines, so $\binom k2$ blue points. Each represents at most two of the $n-k$ red lines, so
\[2\binom k2\ge n-k\implies k\ge\sqrt n,\]
as desired.

\paragraph{Fourth solution, for $O(\sqrt{n\log n})$ lines (Po-Shen Loh)} A \emph{hypergraph} is a graph where edges can connect more than two vertices. For simplicity, any two vertices are joined by at most one edge.
\begin{lemma*}[Duke, Lefmann, R\"odl]
    Given a hypergraph $G$ with $N$ vertices, average degree $d$, and with edges all of size 3, such that any two vertices are joined by at most one 3-edge, there is an independent set of size at least
    \[O\left(N\sqrt{\frac{\log d}d}\right).\]
\end{lemma*}
We will use a similar strategy to the second solution, we will compute some expected value on general graphs and perform alterations to reduce the graph to a favorable one. Also in a similar manner to the second solution, we will pretend polygons with more than four edges have just four edges (since they are asymptotically irrelevant).

Consider each of the $n$ lines as a vertex and each of the finite regions as a hyperedge on the region's sides. So, we are pretending each edge has size 3 or 4. Again, color each vertex blue with probability $p$. Consider the following random variables:
\begin{itemize}
    \item Let $W$ be the number of blue vertices, so $\mathbb E[W]=pn$.
    \item Let $Y$ be the number of blue 4-edges (i.e.\ edges of size at least 4). There are at most $n^2/2$ such edges, so $\mathbb E[Y]\le p^4n^2/2<p^4n^2$.
    \item Let $Z$ be the number of pairs of blue vertices $(u,v)$ with two blue 3-edges containing both $u$, $v$. (Geometrically, there are at most two such edges.) Hence $\mathbb E[Z]\le p^4\binom n2<p^4n^2$.
\end{itemize}
The goal is to eliminate $W$, $Y$, $Z$, then apply the lemma.

Let $X$ be the number of blue edges remaining. Each edge has at least three vertices, and there are at most $n^2/2$ edges, so $\mathbb E[X]\le p^3n^2$.

By Markov's inequality
\[\mathbb P\left(Y>4p^4n^2\right)<\frac14,\quad\mathbb P\left(Z>4p^4n^2\right)<\frac14,\quad\text{and}\quad\mathbb P\left(X>4p^3n^2\right)<\frac14.\]
Moreover observe that $W$ is a binomial distribution, so
\[\mathbb P(W<0.99pn)\to0\quad\text{as}\quad n\to\infty.\]
The sum of these probabilities is less than 1, so by the union bound (Boole's inequality), it is possible for all four inequalities to fail, i.e.\ $Y\le4p^4n^2$, $Z\le4p^4n^2$, $X\le4p^3n^2$, and $W\ge0.99pn$.

Now, delete remove an edge from each of the bad cases in $Y$ and $Z$, so the number of remaining vertices is at least
\[N\ge W-Y-Z\ge0.99pn-8p^4n^2\sim pn\left(1-8p^3n\right).\]
Select $p=0.01n^{-1/3}$, so that $N\sim pn$.

The average degree is bounded by
\[d=\frac{3X}N\lesssim\frac{p^3n^2}{pn}=p^2n,\]
so applying the lemma, there is an independent set of size at least
\[\frac N{\sqrt d}\sqrt{\log d}\sim\frac{pn}{p\sqrt n}\sqrt{\log\sqrt{pn^2}}\sim\sqrt{n\log n},\]
as desired.

