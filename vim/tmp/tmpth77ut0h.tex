% Input your problem and solution below.
% Three dashes on a newline indicate the breaking points.

---

Circle $C_0$ has radius $1$, and the point $A_0$ is a point on the circle. Circle $C_1$ has radius $r<1$ and is internally tangent to $C_0$ at point $A_0$. Point $A_1$ lies on circle $C_1$ so that $A_1$ is located $90^{\circ}$ counterclockwise from $A_0$ on $C_1$. Circle $C_2$ has radius $r^2$ and is internally tangent to $C_1$ at point $A_1$. In this way a sequence of circles $C_1,C_2,C_3,...$ and a sequence of points on the circles $A_1,A_2,A_3,...$ are constructed, where circle $C_n$ has radius $r^n$ and is internally tangent to circle $C_{n-1}$ at point $A_{n-1}$, and point $A_n$ lies on $C_n$ $90^{\circ}$ counterclockwise from point $A_{n-1}$, as shown in the figure below. There is one point $B$ inside all of these circles. When $r=\tfrac{11}{60}$, the distance from the center of $C_0$ to $B$ is $\tfrac{m}{n}$, where $m$ and $n$ are relatively prime positive integers. Find $m+n$.
\begin{center}
    \begin{asy}
        size(6cm); defaultpen(fontsize(10pt));
        real r = 0.8;

        pair nthCircCent(int n){
            pair ans = (0, 0);
            for(int i = 1; i <= n; ++i)
            ans += rotate(90 * i - 90) * (r^(i - 1) - r^i, 0);
            return ans;
        }

        void dNthCirc(int n){
            draw(circle(nthCircCent(n), r^n));
        }

        dNthCirc(0);
        dNthCirc(1);
        dNthCirc(2);
        dNthCirc(3);

        dot("$A_0$", (1, 0), dir(0));
        dot("$A_1$", nthCircCent(1) + (0, r), dir(135));
        dot("$A_2$", nthCircCent(2) + (-r^2, 0), dir(0));
    \end{asy}
\end{center}

---

We use complex numbers. Suppose that $O_n$ is the center of circle $C_n$, and let $o_0=0$ and $a_0=1$. Then, since $C_n\subset C_{n-1}$, we seek $\lim_{n\to\infty} |o_n|$. However, \[\lim_{n\to\infty} o_n=\sum_{k=0}^\infty (1-r)r^ki^k=\frac{1-r}{1-ri}=\frac{49}{60}\div\left(1-\frac{11}{60}i\right),\]
whence \[\lim_{n\to\infty}|o_n|=\left|\frac{49}{60}\div\left(1-\frac{11}{60}i\right)\right|=\frac{49}{60}\div\left|1-\frac{11}{60}i\right|=\frac{49}{61},\]
and the required sum is $49+61=\boxed{110}$.

---

110
