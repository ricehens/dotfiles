% Input your problem and solution below.
% Three dashes on a newline indicate the breaking points.

---

We say that a finite set $\mathcal S$ of points in the plane is \emph{balanced} if, for any two different points $A$ and $B$ in $\mathcal S$, there is a point $C$ in $\mathcal S$ such that $AC=BC$. We say that $\mathcal S$ is \emph{center-free} if for any three different points $A$, $B$, $C$ in $\mathcal S$, there is no points $P$ in $\mathcal S$ such that $PA=PB=PC$.
\begin{enumerate}[label=(\alph*),itemsep=0em]
    \item Show that for all integers $n\ge 3$, there exists a balanced set consisting of $n$ points.
    \item Determine all integers $n\ge 3$ for which there exists a balanced center-free set consisting of $n$ points.
\end{enumerate}

---

\paragraph{Solution (a)} For odd $n$, consider a regular $n$-gon. Now we construct even $n$ inductively using complex coordinates, with $\omega=e^{\pi i/3}$. Begin by letting $\mathcal S$ contain 0, 1, $\omega$, $\seg\omega$ for $n=4$.

Then, to obtain $n+2$ from $n$, select an arbitrary point $z$ on the unit circle such that $z$ and $z\omega$ are not already in $\mathcal S$, and add them.

\paragraph{Solution (b)} The answer is odd $n$, with the construction provided in (a). Now we prove even $n$ don't work.

Note that each of the $\binom n2$ pairs of points has a corresponding point equidistant to both of them, so by Pigeonhole some point is equidistant from at least
\[\left\lceil\frac1n\binom n2\right\rceil=\frac n2\]
pairs of points. For $\mathcal S$ to be center-free, these $n/2$ pairs of points must cover a total of $n$ more points, which is absurd.

