% Input your problem and solution below.
% Three dashes on a newline indicate the breaking points.

---

Find all positive integers $n$ for which there is a permutation $(p_1,p_2,\ldots,p_n)$ of the numbers $(1,2,\ldots,n)$ such that the sets $\{p_1+1,p_2+2,\ldots,p_n+n\}$ and $\{p_1-1,p_2-2,\ldots,p_n-n\}$ are complete residue systems modulo $n$.

---

The answer is whenever $\gcd(n,6)=1$. 
\setcounter{iclaim}0
\begin{iclaim}
    $\gcd(n,2)=1$.
\end{iclaim}
\begin{proof}
    Take the sum of the elements of each set modulo $n$: \[0\equiv\sum_{i=1}^np_i-\sum_{i=1}^ni\equiv\sum_{i=1}^n(p_i-1)\equiv\sum_{i=1}^ni\equiv\frac{n(n+1)}2\pmod n,\]
    so $n$ divides $n(n+1)/2$, and $n$ is odd.
\end{proof}
\begin{iclaim}
    $\gcd(n,3)=1$.
\end{iclaim}
\begin{proof}
    Take the sum of squares of the elements of each set modulo $n$: \[\frac{n(n+1)(2n+1)}3\equiv\sum_{i=1}^n\Big[(p_i-i)^2+(p_i+i)^2\Big]\equiv2\sum_{i=1}^n\left(p_i^2+i^2\right)\equiv\frac{2n(n+1)(2n+1)}3\pmod n,\]
    so $n$ divides $n(n+1)(2n+1)/3$, and $n$ is not divisible by $3$.
\end{proof}

To see that all $n$ with $\gcd(n,6)=1$ work, let $p_i=2i$. 
