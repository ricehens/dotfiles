% Input your problem and solution below.
% Three dashes on a newline indicate the breaking points.

---

Let $ABC$ be a triangle with $AB=30$, $BC=51$, $CA=63$. Points $P$ and $Q$ lie on $\overline{BC}$, $R$ lies on $\overline{CA}$, and $S$ lies on $\overline{AB}$ such that $PQRS$ is a parallelogram, and the center of $PQRS$ coincides with the centroid of $\triangle ABC$. What is the area of parallelogram $PQRS$?

---

Let $G$ be the centroid and $M$ the midpoint of $\overline{BC}$. In general, the area of $PQRS$ is $\tfrac49$ of the area of $\triangle ABC$.
\begin{center}
    \begin{asy}
        size(4cm); defaultpen(fontsize(10pt));
        pair A,B,C,P,Q,R,SS,G,M;
        A=dir(110);
        B=dir(200);
        C=dir(340);
        P=(2B+C)/3;
        Q=(B+2C)/3;
        R=(2A+C)/3;
        SS=(2A+B)/3;
        G=(A+B+C)/3;
        M=(B+C)/2;

        draw(A--B--C--A);
        draw(Q--R--SS--P);
        draw(A--M,dashed);

        dot("$A$",A,N);
        dot("$B$",B,SW);
        dot("$C$",C,SE);
        dot("$P$",P,S);
        dot("$Q$",Q,S);
        dot("$R$",R,NE);
        dot("$S$",SS,NW);
        dot("$G$",G,W);
        dot("$M$",M,S);
    \end{asy}
\end{center}
Let $d(X,\ell)$ be the distance from $X$ to $\ell$ and $d(\ell_1,\ell_2)$ be the distance between parallel lines $\ell_1$, $\ell_2$. Since $d(A,\overline{BC})=3d(G,\overline{BC})$ and $G$ is the center of $PQRS$, we have $d(\overline{RS},\overline{BC})=\tfrac23d(A,\overline{BC})$. This implies $AR=\tfrac13AC$ and $AS=\tfrac12AB$.

Now $\overline{AG}$ bisects $\overline{RS}$ by homothety, so the midpoint of $\overline{PQ}$ is $M$. Then $PQ=RS=\tfrac13BC$, so $BP=PQ=QC$. Finally\[\frac{[ASR]}{[ABC]}=\frac19,\quad\frac{[BPS]}{[ABC]}=\frac29,\quad\frac{[CQR]}{[ABC]}=\frac29,\]and combining these results yields the desired conclusion. Alternatively the base of $PQRS$ is $\tfrac13$ that of $\triangle ABC$, and its height is $\tfrac23$ that of $\triangle ABC$, and the result readily follows as well.

From the given numbers, $[ABC]=756$, so the answer is $336$.

---

336
