% Input your problem and solution below.
% Three dashes on a newline indicate the breaking points.
% vim: tw=72

---

For each positive integer $n$, find the number of $n$-digit positive integers that satisfy both of the following conditions:
\begin{itemize}[itemsep=0em]
    \item no two consecutive digits are equal; and
    \item the last digit is a prime.
\end{itemize}

---

The answer is $\tfrac25\big(9^n-(-1)^n\big)$. Let $a_n$ denote the answer for $n$. Clearly $a_1=4$.
\begin{iclaim*}
    For all $n>1$, $a_n=4\cdot9^{n-1}-a_{n-1}$
\end{iclaim*}
\begin{proof}
    There are $4\cdot9^{n-1}$ numbers with at most $n$ digits obeying the properties: choose the units digit in $4$ ways and each remaining digit can be chosen in $10-1=9$ ways (excluding the previous chosen digit). But $a_{n-1}$ of these start with a $0$; this is because we cannot have more than $1$ leading zero due to consecutive digits not being equal. Thus $a_n=4\cdot9^{n-1}-a_{n-1}$, as desired.
\end{proof}

Now it is not hard to evaluate $a_n$: $$a_n=4\cdot9^{n-1}-4\cdot9^{n-2}+\cdots+(-1)^{n-1}\cdot4=4\cdot9^{n-1}\cdot\frac{1-(-9)^{-n}}{1-(-9)^{-1}}=\frac25\big(9^n-(-1)^n\big),$$
as desired.
