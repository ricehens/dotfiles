desc: Distance from harmonic conjugate
source: AIME II 2012/15
tags: [2020-01, answer, geo, chard, length]

---

Triangle $ABC$ is inscribed in circle $\omega$ with $AB=5$, $BC=7$, and $AC=3$. The bisector of angle $A$ meets side $BC$ at $D$ and circle $\omega$ at a second point $E$. Let $\gamma$ be the circle with diameter $DE$. Circles $\omega$ and $\gamma$ meet at $E$ and a second point $F$. Then $AF^2=\frac mn$, where $m$ and $n$ are relatively prime positive integers. Find $m+n$.

---

\begin{center}
    \begin{asy}
        size(7cm); defaultpen(fontsize(10pt));
        pair A,B,C,M,D,EE,F,K;
        A=(0,15*sqrt(3)/14);
        B=(-65/14,0);
        C=(33/14,0);
        M=(B+C)/2;
        D=extension(A,incenter(A,B,C),B,C);
        EE=2*foot(circumcenter(A,B,C),A,D)-A;
        F=reflect(circumcenter(A,B,C),(D+EE)/2)*EE;
        K=2*circumcenter(A,B,C)-EE;

        draw(B--F--C,gray);
        draw(K--F,dashed);
        draw(K--EE,Dotted);
        draw(circumcircle(A,B,C));
        draw(A--B--C--A--EE);
        draw(circle( (D+EE)/2,abs(D-EE)/2));

        dot("$A$",A,N);
        dot("$B$",B,W);
        dot("$C$",C,E);
        dot("$M$",M,NW);
        dot("$D$",D,NE);
        dot("$E$",EE,S);
        dot("$F$",F,SE);
        dot("$K$",K,N);
    \end{asy}
\end{center}
Note that $E$ is the midpoint of major arc $BC$ on $\omega$. Let $M$ be the midpoint of $\seg{BC}$ and let $\seg{DF}$ intersect $\omega$ again at $K$. Notice that $\angle EFK=\angle EFD=90\dg$, so $K$ is the midpoint of arc $BAC$ on $\omega$, so $\seg{FD}$ bisects $\angle BFC$. Since $\angle EMD=90\dg$, $M$ lies on $\gamma$.

Let $x=BF$ and $y=CF$. By the Law of Cosines on $\triangle ABC$, we have \[\cos A=\frac{5^2+3^2-7^2}{2\cdot5\cdot3}=-\frac12,\]
so $\angle BAC=120\dg$ and $\angle BFC=60\dg$. It follows that $49=x^2-xy+y^2$, again by the Law of Cosines. Furthermore by the Angle Bisector theorem, \[\frac xy=\frac{BD}{DC}=\frac53.\]
Thus we can denote $x=5t$ and $y=3t$, from which \[49=(5t)^2-(5t)(3t)+(3t)^2=19t^2\implies t=\frac7{\sqrt{19}},\]
so by Ptolemy's theorem on $ABFC$, \[7AF=5\cdot3t+3\cdot5t=30t\implies AF=\frac{30t}7=\frac{30}{\sqrt{19}}.\]
It follows that $AF^2=900/19$, and the requested sum is $900+19=919$.
