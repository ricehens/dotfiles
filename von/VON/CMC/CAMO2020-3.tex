author: Eric Shen
desc: (APQ) and (DXY) tangent
source: CAMO 2020/3
tags: [2019-12, oly, tricky, geo, tangent, projective, nice]

---

Let $ABC$ be a triangle with incircle $\omega$, and let $\omega$ touch $\overline{BC}$, $\overline{CA}$, $\overline{AB}$ at $D$, $E$, $F$, respectively. Point $M$ is the midpoint of $\overline{EF}$, and $T$ is the point on $\omega$ such that $\overline{DT}$ is a diameter. Line $MT$ meets the line through $A$ parallel to $\overline{BC}$ at $P$ and $\omega$ again at $Q$. Lines $DF$ and $DE$ intersect line $AP$ at $X$ and $Y$ respectively. Prove that the circumcircles of $\triangle APQ$ and $\triangle DXY$ are tangent.

---

\begin{center}
    \begin{asy}
        size(8cm);
        defaultpen(fontsize(10pt));

        pair A,B,C,I,D,EE,F,M,T,P,Q,L,NN,R,X,Y,K,G,Dp,SS;
        A=dir(125);
        B=dir(200);
        C=dir(340);
        I=incenter(A,B,C);
        D=foot(I,B,C);
        EE=foot(I,C,A);
        F=foot(I,A,B);
        M=(EE+F)/2;
        T=2I-D;
        P=extension(M,T,A,A+B-C);
        Q=2*foot(I,M,T)-T;
        L=2*foot(I,A,T)-T;
        NN=(B+C)/2;
        R=extension(D,T,A,P);
        X=extension(D,F,A,P);
        Y=extension(D,EE,A,P);
        K=extension(D,T,EE,F);
        G=circumcenter(L,T,2R-T);
        Dp=2*circumcenter(D,X,Y)-D;
        SS=reflect(X,Y)*T;

        draw(incircle(A,B,C));
        draw(circumcircle(D,X,Y),gray);
        draw(circumcircle(A,P,Q),gray);
        draw(A--B--C--A);
        draw(D--EE--F--D);
        draw(F--X--Y);
        draw(Y--EE);
        draw(P--Q,gray);
        draw(D--T--R,gray);
        draw(L--Dp,gray);

        dot("$A$",A,dir(80));
        dot("$B$",B,SW);
        dot("$C$",C,SE);
        dot("$D$",D,S);
        dot("$E$",EE,dir(10));
        dot("$F$",F,W);
        dot("$M$",M,NW);
        dot("$T$",T,NW);
        dot("$P$",P,dir(80));
        dot("$Q$",Q,SW);
        dot("$L$",L,SE);
        dot("$N$",NN,S);
        dot("$R$",R,N);
        dot("$X$",X,NW);
        dot("$Y$",Y,NE);
        dot("$D'$",Dp,NW);
    \end{asy}
\end{center}
Let $\overline{DT}$ intersect $\overline{AP}$ at $R$, and let $\overline{AT}$ intersect $\omega$ again at $L$.
\setcounter{claim}0
\begin{claim}
    $T$ lies on $\overline{EX}$ and $\overline{FY}$, $T$ is the orthocenter of $\triangle DXY$, and $A$ is the midpoint of $\overline{XY}$.
\end{claim}
\begin{proof}
    Redefine $X=\overline{DF}\cap\overline{TE}$ and $Y=\overline{DE}\cap\overline{TF}$. Since $\angle DET=\angle DFT=90^\circ$, $T$ is the orthocenter of $\triangle DXY$. Thus, $\overline{DT}\perp\overline{XY}$, so $\overline{XY}\parallel\overline{BC}$.

    By the Three Tangents lemma, the tangents to $\omega$ at $E$ and $F$ intersect at the midpoint of $\overline{XY}$; but this is $A$, thus recovering the original definitions of $X$ and $Y$.
\end{proof}
\begin{claim}
    $L$ lies on $(DXY)$ and $(APQ)$.
\end{claim}
\begin{proof}
    Since $A$ is the midpoint of $\overline{XY}$ and $T$ is the orthocenter of $\triangle DXY$, $\overline{AT}$ passes through $D'$, the antipode of $D$ on $(DXY)$. Note that $\angle DLD'=\angle DLT=90^\circ$, so $L$ lies on $(DXY)$.

    Now $L\in(APQ)$ follows from $TA\cdot TL=TR\cdot TD=TP\cdot TQ$, thus proving the claim.
\end{proof}

Finally, since $\seg{TM}$ is the $T$-symmedian of $\triangle TXY$ and $L$ is the Miquel point of $XYEF$, \[\frac{LX}{LY}=\frac{XF}{YE}=\frac{TX}{TY}=\left(\frac{PX}{PY}\right)^2.\]
It follows that $\seg{LP}$ is a symmedian of $\triangle LXY$. Since $\seg{LA}$ and $\seg{LP}$ are isogonal, $(LAP)$ and $(DXY)$ are tangent, and we are done.
