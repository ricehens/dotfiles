author: Joseph Zhang
desc: When does regular partitioning exist
source: CJMO 2019/5
tags: [2019-10, oly, medium, combo, graph-theory]

---

Let $S$ be a set of $mn+1$ points equally spaced around a circle. Exactly one line segment is drawn between every pair of points in $S$, and each line segment is colored one of $m$ colors. Call a coloring of line segments \emph{fair} if for any color $C$ of the $m$ colors and any point $P$ in $S$, $P$ is the endpoint of exactly $n$ line segments of color $C.$ Find all ordered pairs of positive integers $(m,n)$ such that a fair coloring exists.

---

The answer is all $(m,n)$ such that at least one of the following assertions is true:
\begin{itemize}[itemsep=0em]
    \item $m$ is odd;
    \item $n$ is even.
\end{itemize}

First, we show that this is impossible if $m$ is even and $n$ is odd. Note that $mn+1$ must then be odd. However, it follows that the edges of each color form a $n$-regular graph. Since the graph has an odd number of vertices, this is impossible by the Handshaking Lemma.

A fair coloring clearly exists if $n$ is even. Suppose the distance $d(P,Q)$ between two points $P$ and $Q$ on a circle is one more the number of points between them on the shortest path around the circle. In other words, the distance between two points is the smallest number of arcs one must pass through to get between the two points going around the circle. Label the colors $C_1$ to $C_m$. Then, for each pair of points $P$ and $Q$, color the segment connecting $P$ and $Q$ color $C_i$ if and only if \[\frac{n}{2}(i-1)<d(P,Q)\le\frac{n}{2}\cdot i.\]
It is easy to see that this coloring is fair.

Now, it suffices to show that if $m$ and $n$ are both odd, a fair coloring exists. We have two ways to finish.

\begin{itemize}
    \item \textit{First approach.} From here on, assume that $m$ and $n$ are odd. Label the $mn+1$ points \[V_0,\,V_{1,1},\,V_{1,2},\,\ldots,\,V_{1,m},\,V_{2,1},\,\ldots,\,V_{n,m}.\]
        Furthermore, for all $1\le i\le n$, let $G_i$ be the complete graph containing the points $V_{i,1},V_{i,2},\ldots,$ $V_{i,m}$. We color $G_i$ in a manner such that no two adjacent edges are the same color. Suppose that the vertices of $G_i$ are arranged in a regular $n$-gon in the order $V_{i,1},V_{i,2},\ldots,V_{i,m}$. Now, for every side of the polygon, if the opposite vertex is $V_{i,j}$, color the side $C_j$. Furthermore, color every diagonal the same color as the side parallel to it.

        Since no two parallel segments can be adjacent, no two adjacent edges share a color. Moreover, every vertex $V_{i,j}$ is incident to exactly one edge of each color except $C_j$. Now, for every $j$, color the edge $\overline{V_0V_{i,j}}$  the color $C_j$. Then, every vertex of $G_i$ is now incident to exactly one edge of each color.

        For all $1\le i,j\le n$, we will connect $G_i$ and $G_j$ in the following fashion: For all $1\le p,q\le m$, we color the edge $\overline{V_{i,p}V_{j,q}}$ the color $C_k$, where $1\le k\le m$ is the unique integer such that $p+k\equiv q\pmod{m}$. Then, every node in $G_i$ is incident to exactly one edge of every color that connects it to either $V_0$ or another node in $G_i$, and for every other graph $G_j$, exactly one edge of every color connecting it to some node in $G_j$. Therefore, every node is now incident to exactly $n$ edges of each color, and so we are done.

    \item \textit{Second approach.} Assume $m$ and $n$ are odd. Label the $mn+1$ points \[V_0,V_1,V_2,\ldots,V_{mn}.\]
        Then, for all $0<i,j\le mn$, color the segment $\overline{V_iV_j}$ the color $C_k$ such that $i+j\equiv k\pmod{m}$. Furthermore, for all $0<i\le mn$, color $\overline{V_0V_i}$ the color $C_k$ such that $2i\equiv k\pmod{m}$. It is easy to check that this coloring is fair, so we are done.
\end{itemize}
