author: Justin Lee
desc: s(9k)=9s(k)
source: CJMO 2020/4
tags: [2019-12, oly, easy, nt]

---

For all positive integers $k$, define $s(k)$ to be the result when the last digit of $k$ is moved to the front of $k$. For instance, $s(2020)=202$ and $s(1234)=4123$. For each positive integer $n$, find the number of positive integers $k<10^n$ that satisfy $s(9k)=9s(k)$.

---

First observe that all multiples of $10$ work: if $k=10t$, then it is easy to see that $s(9k)=9t=9s(k)$. Say a positive integer $k$ is $n$-\emph{good} if it not divisible by $10$, has $n$ digits, and satisfies $s(9k)=9s(k)$. The key is to characterize all $n$-good integers.
\begin{claim*}[Characterizing good integers]
    A positive integer $k$ is $n$-good if and only if
    \begin{itemize}[itemsep=0em]
        \item $10^{n-1}\le k<10^n/9$, and
        \item the units digit of $k$ is $1$.
    \end{itemize}
\end{claim*}
\begin{proof}
    The two bullet-points are equivalent to \[k=10^{n-1}+10\ell+1\]
    for some $\ell<10^{n-2}/9$. If $k$ is of this form, then $\ell$ and $9\ell$ have the same number of digits. Thus \[s(9k)=9\cdot10^{n-1}+9\cdot10^{n-2}+9\ell=9s(k),\]
    and $k$ is good.

    Conversely, assume that $k$ is $n$-good. Let the units digit of $k$ be $d$. There are two cases to consider.
    \begin{itemize}
        \item Suppose $d=1$. Then $9k$ has a units digit of $9$, so $9s(k)=s(9k)$ has a leading digit of $9$. It follows that $9s(k)=s(9k)$ also has $n$-digits, so $9k$ has $n$ digits, and $k<10^n/9$.

        \item Suppose $d\ne1$. Since $9k\equiv -k\pmod{10}$, the units digit of $9k$ is $10-d$, so the leading digit of $s(9k)$ is also $10-d$.

            The leading digit of $s(k)$ is $d\ne1$, so the leading digit of $9s(k)$ is either $d$ or $d-1$. Note that $d-1=10-d$ is impossible, so $d=10-d$, and $d=5$.

            Let $e$ be the tens digit of $k$. This becomes the units digit of $s(k)$, so the units digit of $9s(k)$ is $10-e$. However the units digit of $s(9k)$ is the tens digit of $9k$, which is $4+(10-e)\pmod{10}$, contradiction. Thus this case is impossible.
    \end{itemize}

    Combining the above arguments, the claim has been proven.
\end{proof}

The number of multiples of $10$ less than $10^n$ is $10^{n-1}-1$. Let $f(m)$ be the number of $m$-good integers for $m\ge2$. Note that if all $m$ digits are $1$, then the integer is good. Otherwise there must be a $0$. Suppose there are $k$ digits after the $0$. All the digits before the $0$ must be $1$, and we have $10^{k-1}$ choices for the digits after the $0$ but before the units digit.

Therefore we may compute \[f(m)=1+\sum_{i=0}^{m-3}10^{i-1}=\frac{10^{m-2}+8}9.\]
Clearly $f(1)=1=\frac{1+8}9$. Finally \[\sum_{m=1}^nf(m)=\frac{8n}9+\frac19\left(1+\sum_{i=0}^{n-2}10^i\right)=\frac{10^{n-1}+72n+8}{81},\]
and after adding back multiples of $10$, the answer is \[10^{n-1}-1+\frac{10^{n-1}+72n+8}{81}=\boxed{\frac{82\cdot10^{n-1}+72n-73}{81}},\]
and we are done.

