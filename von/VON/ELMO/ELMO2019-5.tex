author: Carl Schildkraut
desc: ab+1 in S
source: ELMO 2019/5
tags: [2020-04, oly, medium, nt, sums]

---

Let $S$ be a nonempty set of integers so that, for any (not necessarily distinct) integers $a$ and $b$ in $S$, $ab+1$ is also in $S$. Show that there are finitely many primes which do not divide any element of $S$.

---

Let $p$ be such a prime. I claim for $s\in S$, we have $p\mid s^2-s+1$. The required conclusion follows from here.

Work in $\mathbb F_p$, so $0\notin S$ and $|S|<p$. Let $s\in S$, and consider the map $S\to S$ by $x\mapsto sx+1$. Since $s\ne0$, this map is bijective. Summing over $x\in S$, where $\sigma$ denotes the sum of the elements in $S$, we have \[\sigma=s\sigma+|S|\implies \sigma=\frac{|S|}{1-s}.\]
This holds for all $s\in S$, so $|S|=1$.

It follows that $s=s^2-1$, so $s^2-s+1=0$, as claimed.
