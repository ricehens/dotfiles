desc: f(x+y)=max(f(x),y)+min(f(y),x)
source: ELMO SL 2010 A3
tags: [2019-10, oly, medium, alg, FE]

---

Solve over $\mathbb R$ the functional equation \[f(x+y)=\max(f(x),y)+\min(f(y),x).\]

---

The answer is $f(x)=x$, which clearly works. Note that by symmetry we have
\begin{align*}
    f(x+y)&=\max(f(x),y)+\min(f(y),x)\\
    &=\min(f(x),y)+\max(f(y),x).
\end{align*}
Hence $|y-f(x)|=|x-f(y)|$. Taking $y=f(x)$ yields $f(f(x))=x$, so $f$ is an involution and therefore bijective. Say that there are $x$ and $y$ for which $f(x)+x\ne f(y)+y$. Since either $f(x)+f(y)=x+y$ or $f(x)+x=f(y)+y$ by the given equation, the former must be true. However write
\begin{align*}
    f(x+f(y))&=\max(f(x),f(y))+\min(x,y)\\
    &=\min(f(x),f(y))+\max(x,y).
\end{align*}
Consequently either $f(x)+x=f(y)+y$ or $f(x)+y=f(y)+x$; the latter must be true. Hence $f(x)+f(y)=x+y$ and $f(x)-f(y)=x-y$, so $f(x)=x$ and $f(y)=y$. Take some fixed point $a$. Then $f(x+a)=\min(f(x),a)+\max(x,a)=f(f(x)+a)$, so $f(x)=x$ for all $x$.

Henceforth that $f(x)+x=f(y)+y$ for all $x$ and $y$. Then $f(x)=-x+b$ for some $b$, so take sufficiently big $Y$ such that $Y>f(x)$ but $x>f(Y)$. Then $f(x+Y)=Y+f(Y)$, but $f$ is bijective, contradiction.
