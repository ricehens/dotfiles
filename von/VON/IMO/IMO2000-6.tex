desc: Website favicon
source: IMO 2000/6
tags: [2019-10, oly, tricky, geo, angle-chasing, nice]

---

In acute triangle $ABC$, let $H_1$, $H_2$, $H_3$ be the feet of the altitudes from $A$, $B$, $C$, respectively, and let $T_1$, $T_2$, $T_3$ be the points where the incircle touches $\overline{BC}$, $\overline{CA}$, $\overline{AB}$, respectively. Prove that the reflections of $\overline{H_1H_2}$, $\overline{H_2H_3}$, $\overline{H_3H_1}$ over $\overline{T_1T_2}$, $\overline{T_2T_3}$, $\overline{T_3T_1}$, respectively, are the sides of a triangle that is inscribed in the incircle of $\triangle ABC$.

---

Let $Y$ lie on the incircle such that $\seg{T_2Y}\parallel\seg{T_3T_1}$. We will show that $Y$ lies on the reflection of $\seg{H_2H_3}$ over $\seg{T_2T_3}$, which is sufficient by symmetry.
\begin{center}
    \begin{asy}
        size(9cm);
        defaultpen(fontsize(10pt));

        pen pri=springgreen;
        pen sec=red;
        pen tri=blue;
        pen qua=purple;
        pen qui=orange;
        pen fil=pri+opacity(0.05);
        pen sfil=sec+opacity(0.05);
        pen tfil=tri+opacity(0.05);
        pen qfil=qua+opacity(0.05);
        pen qifil=qui+opacity(0.05);

        pair A, B, C, H1, H2, H3, I, T1, T2, T3, X, Y, Z, P, Yp;
        A=dir(110);
        B=dir(200);
        C=dir(340);
        H1=foot(A, B, C);
        H2=foot(B, C, A);
        H3=foot(C, A, B);
        I=incenter(A, B, C);
        T1=foot(I, B, C);
        T2=foot(I, C, A);
        T3=foot(I, A, B);
        X=2*foot(T2+T3-T1, T2, T3)-(T2+T3-T1);
        Y=2*foot(T3+T1-T2, T3, T1)-(T3+T1-T2);
        Z=2*foot(T1+T2-T3, T1, T2)-(T1+T2-T3);
        P=extension(T1, Y, T2, T3);
        Yp=2*foot(Y, T2, T3)-Y;

        filldraw(A -- B -- C -- A -- cycle, fil, pri);
        filldraw(incircle(A, B, C), fil, pri);
        filldraw(T1 -- T2 -- T3 -- cycle, sfil, sec); 
        filldraw(X -- Y -- Z -- cycle, tfil, tri);
        draw(T1 -- P -- T2, qua);
        draw(H1 -- H2 -- H3 -- H1, qui+dotted);
        draw(H2 -- Yp, qui+dotted);
        draw(Y -- Yp, tri+dashed);
        draw(H2 -- P, qui+dotted);
        draw(arc((B+C)/2, C, B), qui);
        fill(arc((B+C)/2, C, B) -- cycle, qifil);
        draw(B -- P, qui+dashed);

        dot("$X$", X, S);
        dot("$Y$", Y, SE);
        dot("$Z$", Z, dir(195));
        dot("$Y'$", Yp, N);
        dot("$A$", A, N);
        dot("$B$", B, SW);
        dot("$C$", C, SE);
        dot("$T_1$", T1, S);
        dot("$T_2$", T2, N/2);
        dot("$T_3$", T3, dir(150));
        dot("$H_1$", H1, S);
        dot("$H_2$", H2, N);
        dot("$H_3$", H3, dir(120));
        dot("$P$", P, E);
    \end{asy}
\end{center}
Let $P=\seg{T_1Y}\cap\seg{T_2T_3}$, which lies on $\seg{BI}$ by reflection, and let $Y'$ be the reflection of $Y$ over $\seg{T_2T_3}$. Then by the Iran Lemma, $P$ lies on $(BCH_2H_3)$. Since $\da H_2PB=\da H_2CB=\da T_2PY$, but $\seg{PB}$ bisects $\angle T_2PY$, $\seg{PH_2}$ bisects $\angle T_2PY'$, so we deduce by $PY'=PY=PT_2$ that $PT_2H_2Y'$ is a kite. Now \[\da PH_2Y'=\da T_2H_2P=\da CBP=\da PBH_3=\da PH_2H_3,\]
completing the proof.

