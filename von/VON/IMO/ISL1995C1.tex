desc: Dyck paths with fixed number valleys
source: ISL 1999 C1
tags: [2020-03, oly, hard, combo, rigid, nice, involved, waltz]

---

Let $n\ge1$ be an integer. A \emph{Dyck path} in the $xy$-plane is a chain of consecutive unit moves either to the right (denoted by $E$) or upwards (denoted by $N$) from $(0,0)$ to $(n,n)$, such that all points on the path lie in the half-plane $x\ge y$. A \emph{valley} in a Dyck path is the occurrence of two consecutive moves of the form $EN$.

Show that the number of Dyck paths from $(0,0)$ to $(n,n)$ that contain exactly $s$ valleys ($1\le s\le n$) is \[\frac1s\binom{n-1}{s-1}\binom n{s-1}.\]

---

Induction with enough computational fortitude works. Here is a nice solution from \S2.4.2 in \href{https://www.springer.com/cda/content/document/cda_downloaddocument/9781493930906-c1.pdf}{Kyle Petersen's \emph{Eulerian Numbers}}. We will exhibit \[\binom{n-1}{s-1}\binom n{s-1}\]
lattice paths, partition them into equivalence classes of $s$ elements, and finally show exactly one element from each equivalence class is a Dyck path with $s$ valleys.

Redefine a valley to be a point on a Dyck path such that the previous move is $E$ and the subsequent move is $N$. Analogously define a \emph{peak} to be a point such that the previous move is $N$ and the subsequent move is $E$. (In the figure, the colored dots are peaks.) Define the \emph{height} of a point $(x,y)$ as $x-y$, so a valley has smaller height than its neighbors and a peak has greater height than its neighbors. If some peak has positive height, say the \emph{absolute peak} is the point with greatest height and greatest $y$-coordinate; e.g.\ the green dot below. Otherwise (if the path is a Dyck path), place the absolute peak at the beginning.
\begin{center}
\begin{asy}
    size(7cm); defaultpen(fontsize(10pt));

    for (real x=-1; x<=8+1e-9; x+=1) draw( (x,0)--(x,8),gray);
    for (real y=0; y<=8+1e-9; y+=1) draw( (-1,y)--(8,y),gray);
    draw( (0,0)--(8,8));

    pair[] p = {(-1,0),(0,0),(1,0),(2,0),(2,1),(2,2),(2,3),(3,3),(4,3),(4,4),(5,4),(5,5),(5,6),(6,6),(7,6),(7,7),(8,7),(8,8)};
    dot(p[0]);
    for (int i=0; i<p.length-1; ++i) {
        dot(p[i+1]);
        draw(p[i]--p[i+1],linewidth(1));
    }

    dot( (2,3),red);
    dot( (4,4),red);
    dot( (5,6),green);
    dot( (7,7),red);
\end{asy}
\end{center}
Consider the set of lattice paths $P$ from $(-1,0)$ to $(n,n)$ that begin with $E$, and in $N$, and have exactly $s$ valleys. These paths are determined by its peaks $(x_i,y_i)$ for $1\le i<s$, such that $0\le x_1<x_2<\cdots<x_{s-1}\le n-1$ and $1\le y_1<y_2<\cdots<y_{s-1}\le n-1$. By Stars and Bars we may choose the $x$-coordinates in $\tbinom n{s-1}$ ways and the $y$-coordinates in $\tbinom{n-1}{s-1}$ ways. Thus \[|P|=\binom{n-1}{s-1}\binom n{s-1}.\]

We express each $p\in P$ has a sequence of moves, and we dissect each sequence into peak-less paths. Place a marker $(\bullet)$ at the absolute peak. For instance in the above figure, we would write \[p=(EEENNN)(ENN)(EEN)\bullet(EEN)(EN).\]
Each equivalence class $[p]$ consists of taking cyclic permutations of peak-less paths. For our example, \[ [p]=\begin{Bmatrix}(EEENNN)(ENN)(EEN)\bullet(EEN)(EN)\\ (EN)(EEENNN)(ENN)(EEN)\bullet(EEN)\\ \bullet(EEN)(EN)(EEENNN)(ENN)(EEN)\\ (EEN)\bullet(EEN)(EN)(EEENNN)(ENN)\\ (ENN)(EEN)\bullet(EEN)(EN)(EEENNN)\end{Bmatrix}.\]
Note that the marker is also cyclically permuted, and the marker uniquely identifies the cyclic permutation of $p$, since after the peak there must always be more $E$'s than $N$'s.

Hence each equivalence class contains $s$ paths, and only one path --- the path with the marker on the far left --- is a Dyck path. This concludes the proof.

