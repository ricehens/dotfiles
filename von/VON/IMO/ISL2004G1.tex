desc: Bisector BAC and MON intersect at R
source: ISL 2004 G1
tags: [2019-11, oly, trivial, geo, angle-chasing]

---

Let $ABC$ be an acute-angled triangle with $AB\ne AC$. The circle with diameter $\seg{BC}$ intersects sides $AB$ and $AC$ at $M$ and $N$ respectively. Denote by $O$ the midpoint of side $BC$. The bisectors of angles $\angle BAC$ and $\angle MON$ intersect at $R$. Prove that the circumcircles of the triangles $BMR$ and $CNR$ have a common point lying on the side $BC$.

---

By Miquel's Theorem we just need to show that $AMRN$ is cyclic. But since $R$ lies on the bisector of $\angle MON$, $RM=RN$, and since $R$ lies on the bisector of $\angle MAN$, $R$ must be the midpoint of minor arc $MN$ on the circumcircle of $\triangle AMN$. Since $R$ lies on arc $MN$, the intersection $(BMR)\cap(CNR)\setminus D$ lies on segment $BC$.
