desc: Evan's StarCraft tournament
source: ISL 2018 C5
tags: [2019-10, oly, brutal, combo, equality, optimization, global, nice, involved]

---

Let $k$ be a positive integer. The organizing commitee of a tennis tournament is to schedule the matches for $2k$ players so that every two players play once, each day exactly one match is played, and each player arrives to the tournament site the day of his first match, and departs the day of his last match. For every day a player is present on the tournament, the committee has to pay $1$ coin to the hotel. The organizers want to design the schedule so as to minimise the total cost of all players' stays. Determine this minimum cost.

---

The answer is $2k^3-\frac12k^2-\frac12k$. Label the players $A_1$, $\ldots$, $A_{2k}$, and say that they enter the hotel in that order.
\setcounter{claim}0
\begin{claim}
    There exists an equality case where the players leave the hotel in the same order they arrive.
\end{claim}
\begin{proof}
    We will show that for every permutation of possible games, there is a permutation in which the players leave in the order $A_1$, $\ldots$, $A_{2k}$, which is not more expensive. Perform the following operation until the players leave in sorted order:

    Take two players $A_p$ and $A_q$, $p<q$ (hence $A_p$ arrives earlier than $A_q$), so that $A_p$ leaves after $A_q$. For every other opponent $A_i$ that $A_q$ plays, if $A_p$ has not played $A_i$ by the time $A_q$ plays him, swap the day on which the game between $A_p$ and $A_i$ occurs and the day on which the game between $A_q$ and $A_i$ occurs.

    The set of days on which $A_i$ plays a game remains fixed, so the total number of days $A_i$ spends at the hotel remains unchanged. Furthermore $A_p$ enters the hotel at the same time, and leaves no later than when $A_q$ originally left, and $A_q$ enters no earlier than when he originally arrived, and leaves no later than when $A_p$ originally left. Hence, it is impossible for this arrangement to result in a higher cost.
\end{proof}

Henceforth assume WLOG the players leave the hotel in the same order they arrive. Let $C(i,j)$ be the number of players at the hotel when $A_i$ and $A_j$ play their match; so that we seek to minimize $\sum_{i,j}C(i,j)$.
\begin{claim}
    If $C(p,q)\le n$ for some $n$, then either $p,q\le n$ or $p,q>2k-n$.
\end{claim}
\begin{proof}
    We prove the contrapositive. Clearly players $A_i$, for $p\le i\le q$ must be at the hotel. Note that if any player $A_i$, $i\le p$, has left the hotel, then he has played all other players, so players $A_p$ through $A_{2k}$ must be at the hotel.

    Hence if players $A_1$, $\ldots$, $A_q$ are not all at the hotel, then players $A_p$, $\ldots$, $A_{2k}$ are at the hotel. Then if $q>n$, the first scenario yields $C(p,q)>n$, and if $p\le 2k-n$, the second scenario yields $C(p,q)>n$. Thus the claim has been proven.
\end{proof}

It is now clear that the minimum possible value of $C(p,q)$ is the smallest integer $n$ such that either $p,q\le n$ or $p,q>2k-n$. We can verify number of days with cost at most $n$ is given by \[f(n)=\begin{cases}2\binom n2&\text{if }n<k,\\ \binom{2k}2-(2k-n)^2&\text{otherwise.}\end{cases}\]
After some computation, we deduce that
\begin{align*}
    \sum_{i=1}^{2k-1}\sum_{j=i+1}^{2k}C(i,j)&\ge\sum_{n=1}^{2k}n(f(n)-f(n-1))\\
    &=2kf(2k)-\sum_{n=k}^{2k-1}f(n)-\sum_{n=1}^{k-1}f(n)\\
    &=k\binom{2k}2+\frac{k(k-1)(2k-1)}6-\frac{k(k-1)(k-2)}3\\
    &=2k^3-\frac{k^2}2-\frac k2.
\end{align*}
Finally, consider the following construction that preserves equality:
\begin{itemize}
    \item For the first $\frac12k(k-1)$ days, players $A_1$, $\ldots$, $A_k$ arrive at the hotel and play a round-robin. Assume that when player $A_i$ arrives, all the players currently at the hotel have already played a round-robin among themselves.
    \item For the next $\frac12k(k+1)$ days, for $i=1$ through $k$, player $A_{k+i}$ arrives at the hotel and plays players $A_1$ through $A_{k-i+1}$.
    \item For the next $\frac12k(k-1)$ days, from $i=1$ through $k$, player $A_i$ plays players $A_{2k-j}$ for $j=0$ through $i-2$, then departs from the hotel.
    \item For the final $\frac12k(k-1)$ days, players $A_{k+1}$, $\ldots$, $A_{2k}$ play a round robin in a similar manner to the round-robin among players $A_1$, $\ldots$, $A_k$.
\end{itemize}
It is not hard to see that equality holds for this construction, so we are done.
