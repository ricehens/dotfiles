desc: Angle chasing madness
source: Iran TST 2011/6a
tags: [2019-11, oly, hard, geo, angle-chasing, conditional, moving-points]

---

Let $BCB'C'$ be a rectangle inscribed in the circle $\omega$ with center $O$. Points $K$ and $H$ lie on the tangents to $\omega$ at $B$ and $C$, respectively, and points $K'$ and $H'$ lie on the angle bisectors of $\angle BCO$ and $\angle CBO$, respectively, such that $\overline{KK'}$ and $\overline{HH'}$ are perpendicular to $\overline{BC}$. Prove that $K$, $H'$, $B'$ are collinear if and only if $H$, $K'$, $C'$ are collinear.

---

\begin{center}
    \begin{asy}
        size(8cm);
        defaultpen(fontsize(9pt));

        pen pri=deepblue;
        pen sec=royalblue;
        pen tri=blue;
        pen qua=Cyan;
        pen qui=deepcyan;
        pen fil=pri+opacity(0.05);
        pen sfil=sec+opacity(0.05);
        pen tfil=tri+opacity(0.05);
        pen qfil=qua+opacity(0.05);
        pen qifil=qui+opacity(0.05);

        pair O, T, B, C, Bp, Cp, I, K, Kp, H, Hp, M, NN, X, Y, P, Q, O1, O2, J, Kq, Hq;
        O=(0, 0); T=(0, 1.7);
        B=intersectionpoints(circle(O, 1), circle(T/2, length(T)/2))[1];
        C=intersectionpoints(circle(O, 1), circle(T/2, length(T)/2))[0];
        Bp=-B; Cp=-C;
        I=incenter(O, B, C);
        K=(2B+7T)/9;
        Kp=extension(C, I, K, foot(K, B, C));
        H=extension(Kp, Cp, T, C);
        Hp=extension(B, I, H, foot(H, B, C));
        M=dir(0);
        NN=dir(180);
        X=foot(K, B, C);
        Y=foot(H, B, C);
        P=foot(B, Bp, K);
        Q=foot(C, Cp, H);
        O1=extension(O, P, K, K+O-B);
        O2=extension(O, Q, H, H+O-C);
        J=intersectionpoint(O1 -- O2, circle(O1, length(K-O1)));
        Kq=extension(K, J, H, Hp);
        Hq=extension(H, J, K, Kp);

        filldraw(circle(O, 1), fil, pri);
        filldraw(circumcircle(B, X, P), sfil, sec);
        filldraw(circumcircle(C, Y, Q), sfil, sec);
        filldraw(circumcircle(P, X, Y), sfil, sec);
        filldraw(circumcircle(B, Y, P), tfil, tri);
        filldraw(circumcircle(C, X, Q), tfil, tri);
        draw(K -- Bp, tri);
        draw(H -- Cp, tri);
        draw(B -- Bp, pri);
        draw(C -- Cp, pri);
        draw(B -- T -- C -- B, pri);
        draw(K -- Kp, pri);
        draw(H -- Hp, pri);
        draw(B -- M, sec);
        draw(C -- NN, sec);
        draw(M -- P -- Cp, qua);
        draw(NN -- Q -- Bp, qua);

        dot("$O$", O, S);
        dot("$T$", T, N);
        dot("$B$", B, dir(190));
        dot("$C$", C, dir(-10));
        dot("$B'$", Bp, Bp);
        dot("$C'$", Cp, Cp);
        dot("$I$", I, S);
        dot("$K$", K, dir(95));
        dot("$H$", H, dir(80));
        dot("$K'$", Kp, dir(155));
        dot("$H'$", Hp, dir(210));
        dot("$M$", M, M);
        dot("$N$", NN, NN);
        dot("$X$", X, NW);
        dot("$Y$", Y, NE);
        dot("$P$", P, dir(60));
        dot("$Q$", Q, dir(102));
    \end{asy}
\end{center}
\paragraph{First solution, by angle chasing}     Ignore the collinearities for now. Let $\overline{B'K}$ and $\overline{C'H}$ intersect $\omega$ again at $P$ and $Q$, and let $\overline{BC}$ intersect $\overline{KK'}$ at $X$ and $\overline{HH'}$ at $Y$. Denote by $M$ and $N$ the midpoints of arcs $CB'$ and $BC'$, so that $H'\in\overline{BM}$ and $K'\in\overline{CN}$.
\setcounter{claim}0
\begin{claim}
    Points $Q$, $Y$, $B'$ are collinear; points $P$, $X$, $C'$ are collinear; and $XPQY$ is cyclic.
\end{claim}
\begin{proof}
    Since $\measuredangle CQY=\measuredangle CHY=90^{\circ}-\measuredangle YCH=\measuredangle OCB=\measuredangle CBB'=\measuredangle CQB'$, points $Q$, $Y$, $B'$ are collinear. Analogously $P$, $X$, $C'$ are collinear. Finally the concyclicity follows from the converse of Reim's theorem. To spell it out, $\measuredangle PXY=\measuredangle PC'B'=\measuredangle PQB'=\measuredangle PQY$.
\end{proof}
\begin{claim}
    Points $H$, $K'$, $C'$ are collinear if and only if points $Q$, $X$, $N$ are collinear. Symmetrically, points $K$, $H'$, $B'$ are collinear if and only if points $P$, $Y$, $M$ are collinear.
\end{claim}
\begin{proof}
    Check that $\measuredangle K'QX=\measuredangle K'CX=\measuredangle NCB=\measuredangle C'QN$, so \[\overline{HK'C'}\iff\measuredangle C'QX=\measuredangle K'QX\iff\measuredangle C'QX=\measuredangle C'QN\iff\overline{QXN},\]
    where $\overline{UVW}$ is the assertion that $U$, $V$, $W$ are collinear.
\end{proof}
\begin{claim}
    Points $Q$, $X$, $N$ are collinear if and only if points $P$, $Y$, $M$ are collinear.
\end{claim}
\begin{proof}
    Assume that $Q$, $X$, $N$ are collinear. It follows that $\measuredangle C'PY=\measuredangle XPY=\measuredangle XQY=\measuredangle NQB'=\measuredangle C'QN$, proving the claim.
\end{proof}

Putting these claims together yields the desired conclusion.

\paragraph{Second solution, by moving points}     Let $M$ and $N$ be the midpoints of arcs $CB'$ and $BC'$ respectively. Choose a point $K$ on $\overline{BT}$; we construct $K'$ on $\overline{CN}$ with $\overline{KK'}\perp\overline{BC}$, let $H=\overline{CT}\cap\overline{C'K'}$, construct $H'$ on $\overline{BM}$ with $\overline{HH'}\perp\overline{BC}$, and let $K_0=\overline{BT}\cap\overline{B'H'}$. Our task is to prove that $K=K_0$.

Move $K$ along line $BT$ at a linear rate, and let $\infty_{\perp BC}$ be the point at infinity perpendicular to $\overline{BC}$.
\begin{itemize}[itemsep=0em]
    \item By projection through $\infty_{\perp BC}$, $K'$ moves along $\overline{CN}$ at a linear rate.
    \item By projection through $C'$, $H$ moves along $\overline{CT}$ at a linear rate.
    \item By projection through $\infty_{\perp BC}$, $H'$ moves along $\overline{BM}$ at a linear rate.
    \item By projection through $B'$, $K_0$ moves along $\overline{BT}$ at a linear rate.
\end{itemize}
Thus it suffices to verify the hypothesis for three values of $K$. We verify the following special cases:
\begin{itemize}[itemsep=0em]
    \item $K=B$: Then $K'$ lies on $\overline{BC}$, to $H$ is the reflection of $C$ over $T$. Then $H'=B$, so $K_0=B$.
    \item $K$ is the reflection of $B$ over $T$: Then $K'=C$, so $H=C$ and $H'$ is a point on $\overline{CB'}$. Thus $K_0=K$.
    \item $K$ is the point at infinity along $\overline{TB}$: Then $K'$ is the point at infinity along $\overline{CN}$, so $H=\overline{TC}\cap\overline{MC'}$. Then since $H'$ lies on $\overline{BM}$, it is the reflection of $H$ over $\overline{MN}$, so $\overline{B'H'}$ is tangent to $\omega$. It follows that $K_0=K$.
\end{itemize}
This completes the proof.


