desc: Rhombus in heptagon
source: MPfG 2019/16
tags: [2019-10, answer, cbrutal, geo, length, nice, waltz]

---

The figure shows a regular heptagon with sides of length 1.
\begin{center}
    \begin{asy}
        size(3cm);
        defaultpen(fontsize(9pt));
        pair A,B,C,D,EE,F,G,P,M;
        A=dir(90);
        B=dir(90+360/7*1);
        C=dir(90+360/7*2);
        D=dir(90+360/7*3);
        EE=dir(90+360/7*4);
        F=dir(90+360/7*5);
        G=dir(90+360/7*6);
        P=B+G-A;
        M=(D+EE)/2;
        draw(B--C--D--EE--F--G--A--B--P--G);
        draw(P--M);
        dot(A); dot(B); dot(C); dot(D); dot(EE); dot(F); dot(G);
        dot(P); dot(M);
        draw(rightanglemark(P,M,EE,4));
        label("$1$",P--B,rotate(90)*unit(B-P));
        label("$1$",P--G,rotate(90)*unit(P-G));
        label("$d$",P--M,W);
    \end{asy}
\end{center}
Determine the indicated length $d$. Express your answer in simplified radical form.

---

\paragraph{First solution, by median formula (David Altizio)}     Label the points as in the diagram shown.
\begin{center}
    \begin{asy}
        size(5cm);
        defaultpen(fontsize(10pt));
        pair A,B,C,D,EE,F,G,P,M,O;
        A=dir(90);
        B=dir(90+360/7*1);
        C=dir(90+360/7*2);
        D=dir(90+360/7*3);
        EE=dir(90+360/7*4);
        F=dir(90+360/7*5);
        G=dir(90+360/7*6);
        P=B+G-A;
        M=(D+EE)/2;
        O=(B+G)/2;
        draw(B--C--D--EE--F--G--A--B--P--G--B);
        draw(A--P);
        dot("$A$",A,A);
        dot("$B$",B,B);
        dot("$C$",C,C);
        dot("$D$",D,D);
        dot("$E$",EE,EE);
        dot("$F$",F,F);
        dot("$G$",G,G);
        dot("$P$",P,S);
        dot("$O$",O,NW);
    \end{asy}
\end{center}
Let $O$ be the midpoint of $\seg{AP}$ and $\seg{BG}$, and let $s_2=BG$ and $s_3=CF$. Since the heptagon is regular, by the median formula on $\triangle DBG$, \[4DO^2=2(DB^2+DG^2)-BG^2=s_2^2+2s_3^2,\]
whereas by the median formula on $\triangle DAP$, \[4DO^2=2(DA^2+DP^2)-AP^2=s_2^2+2s_3^2+2DP^2-4.\]
Setting these equal, we find that $DP^2=2$, so the distance from $P$ to $\seg{DE}$ is $\sqrt7/2$.

\paragraph{Second solution, by trigonometry (Tristan Shin)}     Refer to the above diagram. Since $\seg{AB}\parallel\seg{GP}$, $C$, $P$, $G$ are collinear, and similarly $F$, $P$, $B$ are collinear. It is not hard to see that $\angle CBP=3\pi/7$, so by the Law of Cosines, $CP=2\sin(3\pi/14)$. Since $\angle DCP=3\pi/7$,
\begin{align*}
    DP^2&=CD^2+DP^2-2\cdot CD\cdot DP\cdot\cos\angle DCP\\
    &=1+4\sin^2\frac{3\pi}{14}-4\sin\frac{3\pi}{14}\cos\frac{3\pi}7\\
    &=1+4\sin\frac{3\pi}{14}\left(\cos\frac{2\pi}7-\cos\frac{3\pi}7\right)\\
    &=1+8\cos\frac{2\pi}7\cos\frac{5\pi}{14}\cos\frac{\pi}{14}\\
    &=1-8\cos\frac\pi7\cos\frac{2\pi}7\cos\frac{4\pi}7\\
    &=1-\frac{8\sin\frac\pi7\cos\frac\pi7\cos\frac{2\pi}7\cos\frac{4\pi}7}{\sin\frac\pi7}\\
    &=1-\frac{\sin\frac{8\pi}7}{\sin\frac\pi7}\\
    &=2,
\end{align*}
and thus the distance from $P$ to $\seg{DE}$ is $\sqrt7/2$.

