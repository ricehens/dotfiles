desc: Evil equation with geometric interpretation
source: Mock AIME 2018/15
tags: [2019-12, answer, chard, geo, length, hidden-geo]
author: Eric Shen

---

Let $a$, $b$, $c$, and $d$ be positive real numbers such that \[195=a^2+b^2=c^2+d^2=\frac{13\left(ac+bd\right)^2}{13b^2-10bc+13c^2}=\frac{5\left(ad+bc\right)^2}{5a^2-8ac+5c^2}.\]
Find the greatest integer that does not exceed $a+b+c+d$.

---

Clearly, an Algebra Bash would be catastrophic. Instead, we present a geometric approach. \\
\begin{lemma*}
    In cyclic quadrilateral $ABCD$ with circumradius $R$, the area $K$ of $ABCD$ is \[K=2R^2\sin A\sin B\sin\theta,\]
    where $\theta$ is either one of the angles formed by the diagonals $AC$ and $BD$ of quadrilateral $ABCD$.
\end{lemma*}
\begin{proof}
    Suppose $\widehat{AB}=w$, $\widehat{BC}=x$, $\widehat{CD}=y$, and $\widehat{DA}=z$. By the Extended Law of Sines, $BD=2R\sin A$ and $AC=2R\sin B$. It follows that \[K=\frac{1}{2}\cdot BD\cdot AC\cdot\sin\theta=2R^2\sin A\sin B\sin\theta,\]
    and our lemma has been proven.
\end{proof}
\begin{center}
    \begin{asy}
        size(6cm); defaultpen(fontsize(10pt));

        real markscalefactor=0.03;
        path rightanglemark(pair A, pair B, pair C, real s=8)
        {
            pair P,Q,R;
            P=s*markscalefactor*unit(A-B)+B;
            R=s*markscalefactor*unit(C-B)+B;
            Q=P+R-B;
            return P--Q--R;
        }

        pair A, B, C, D;
        A=(0, 7);
        B=(0, 0);
        C=(4, 0);
        D=(4-40/13, 96/13);
        draw(A -- B -- C -- D -- A);
        draw(A -- C);
        draw(B -- D);
        draw(circle((2, 7/2), sqrt(65)/2));
        label("$A$", A, NW);
        label("$B$", B, SW);
        label("$C$", C, SE);
        label("$D$", D, N);
        draw(rightanglemark(A, B, C));
        draw(rightanglemark(C, D, A));
    \end{asy}
\end{center}
Consider a cyclic quadrilateral $ABCD$ with $\angle ABC=\angle CDA=90^\circ$ and $AB=a$, $BC=b$, $CD=c$, and $DA=d$. Then, the diameter of the circumcircle of $ABCD$ is $AC=\sqrt{195}$.\\

By Ptolemy's Theorem, $ac+bd=\sqrt{195}\cdot BD$. Substituting into the given equation gives \[195=\frac{13(ac+bd)^2}{13b^2-10bc+13c^2}=\frac{13\cdot 195\cdot BD^2}{13b^2-10bc+13c^2}=\frac{195\cdot BD^2}{b^2-\frac{10}{13}bc+c^2}.\]
We then have $BD^2=b^2+c^2-2bc\cdot\frac{5}{13}$. This is the Law of Cosines on $\triangle BCD$. It is easy to see that $\cos\angle BCD=\frac{5}{13}$. It follows that $\sin\angle BCD=\frac{12}{13}$. Then, by our Lemma, the area of $ABCD$ is \[K=2\cdot\left(\frac{\sqrt{195}}{2}\right)^2\cdot 1\cdot\frac{12}{13}\cdot\sin\theta=90\sin\theta.\]
By the Law of Sines on $\triangle BCD$, $BD=\frac{12\sqrt{195}}{13}$. By Ptolemy's, $ac+bd=180$. \\

Suppose we define the lengths of the arcs as in the proof of our lemma. Then, $\theta=\frac{w+y}{2}$.

\begin{center}
    \begin{asy}
        size(6cm); defaultpen(fontsize(10pt));

        real markscalefactor=0.03;
        path rightanglemark(pair A, pair B, pair C, real s=8)
        {
            pair P,Q,R;
            P=s*markscalefactor*unit(A-B)+B;
            R=s*markscalefactor*unit(C-B)+B;
            Q=P+R-B;
            return P--Q--R;
        }

        pair A, B, C, D;
        A=(0, 7);
        B=(0, 0);
        C=(4, 0);
        D=(24/5, 7-32/5);
        draw(A -- B -- C -- D -- A);
        draw(A -- C);
        draw(B -- D);
        draw(circle((2, 7/2), sqrt(65)/2));
        label("$A'$", A, NW);
        label("$B'$", B, SW);
        label("$D'$", C, SE);
        label("$C'$", D, E);
        draw(rightanglemark(A, B, C));
        draw(rightanglemark(C, D, A));
    \end{asy}
\end{center}

Now consider cyclic quadrilateral $A'B'D'C'$, with $\angle A'B'D'=\angle D'C'A'=90^\circ$ and $A'B'=a$, $B'D'=b$, $D'C'=d$, and $C'A'=c$. It follows that $\widehat{A'B'}=w$, $\widehat{B'D'}=x$, $\widehat{D'C'}=z$, and $\widehat{C'A'}=y$. Also note that $A'D'=\sqrt{195}$. Notice that \[\sin\theta=\sin\left(\frac{w+y}{2}\right)=\sin\left(\frac{x+z}{2}\right)=\sin\angle B'A'C'.\]
By Ptolemy's, $ad+bc=\sqrt{195}\cdot B'C'$. Substituting into the given equation gives \[195=\frac{5(ad+bc)^2}{5a^2-8ac+5c^2}=\frac{5\cdot 195\cdot B'C'^2}{5a^2-8ac+5c^2}=\frac{195\cdot B'C'^2}{a^2-\frac{8}{5}ac+c^2}.\]
We then have $B'C'^2=a^2+c^2-2ac\cdot\frac{4}{5}$. Then, $\cos\angle B'A'C'=\frac{4}{5}$, and $\sin\theta=\sin\angle B'A'C'=\frac{3}{5}$. It follows that \[K=90\sin\theta=54\implies ab+cd=108.\]
Note that $B'C'=\frac{3\sqrt{195}}{5}$ by the Law of Sines on $\triangle A'B'C'$. By Ptolemy's, $ad+bc=117$. It follows that
\begin{align*}
    a+b+c+d&=\sqrt{\left(a^2+b^2\right)+\left(c^2+d^2\right)+2(ac+bd)+2(ab+cd)+2(ad+bc)}\\
    &=\sqrt{195+195+2\cdot 180+2\cdot 108+2\cdot 117}=\sqrt{1200}=20\sqrt{3},
\end{align*}
and the answer is $\lfloor 20\sqrt3\rfloor=34$.

---

034
