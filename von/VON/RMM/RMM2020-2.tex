desc: Co-harmonic sequences
source: RMM 2020/2
tags: [2020-02, oly, medium, alg, sequence]

---

Let $N\ge2$ be an integer, and let $\mathbf a=(a_1,\ldots,a_N)$ and $\mathbf b=(b_1,\ldots,b_N)$ be sequences of nonnegative integers. Let $a_i=a_{i+N}$ and $b_i=b_{i+N}$ for all integers $i$. We say $\mathbf a$ is \emph{$\mathbf b$-harmonic} if each $a_i$ equals the following arithmetic mean: \[a_i=\frac1{2b_i+1}\sum_{s=-b_i}^{b_i}a_{i+s}.\]
Suppose that neither $\mathbf a$ nor $\mathbf b$ is a constant sequence, and that both $\mathbf a$ is $\mathbf b$-harmonic and $\mathbf b$ is $\mathbf a$-harmonic.

Prove that at least $N+1$ of the numbers $a_1$, $\ldots$, $a_N$, $b_1$, $\ldots$, $b_N$ are zero.

---

The pith is this claim:
\begin{claim*}
    For all $i$, we have $0\in\{a_i,b_i\}$.
\end{claim*}
\begin{proof}
    Assume for contradiction this does not hold. Say $i$ is a \emph{candidate} if $a_i$ is maximal over all $j$ with $0\notin\{a_j,b_j\}$, and select a candidate $i$ such that $i-1$ is not a candidate. Assume without loss of generality $a_i\ge b_j$ for all $j$ with $0\notin\{a_j,b_j\}$.

    Since $\mathbf a$ is $\mathbf b$-harmonic, there is an $s$ with $|s|\le b_i$ such that $a_{i+s}>a_i$ (the inequality is strict since $a_{i-1}<a_i$). By assumption that $a_i$ is maximal, we know $b_{i+s}=0$. Since $\mathbf b$ is $\mathbf a$-harmonic, $b_{i+s}=0$ is the average of $b_{i+s+t}$ over all $t$ with $|t|\le a_{i+s}$; thus all $b_{i+s+t}$ are zero. However $a_{i+s}>a_i\ge b_i\ge |s|$, so $b_i=0$, contradiction.
\end{proof}

To finish, there must exist $i$ with $a_i=0$ but $a_{i-1}\ne0$. Then $a_{i+s}=0$ for all $|s|\le b_i$, so $b_i=0$. Thus if $k$ elements of $\mathbf a$ are nonzero, then at least $k+1$ elements of $\mathbf b$ are zero, so the number of zeros among $a_1$, $\ldots$, $a_N$, $b_1$, $\ldots$, $b_N$ is at least $(N-k)+(k+1)=N+1$.

