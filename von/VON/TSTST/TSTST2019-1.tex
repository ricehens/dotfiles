desc: AMC -> TSTST FE
source: TSTST 2019/1
tags: [2019-10, oly, medium, alg, FE]

---

Find all binary operations $\mathbin{\diamondsuit}:\mathbb R_{>0}\times\mathbb R_{>0}\to\mathbb R_{>0}$ (meaning $\mathbin{\diamondsuit}$ takes pairs of positive real numbers) such that for any real numbers $a,b,c>0$,
\begin{itemize}[itemsep=0em]
    \item the equation $a\mathbin{\diamondsuit}(b\mathbin{\diamondsuit}c)=(a\mathbin{\diamondsuit}b)\cdot c$ holds; and
    \item if $a\ge 1$ then $a\mathbin{\diamondsuit}a\ge 1$.
\end{itemize}

---

The answer is $\times$ and $\div$. It is easy to check that these work. We now show these are the only solutions.
\setcounter{iclaim}0
\begin{iclaim}
    If $a\mathbin{\diamondsuit}p=a\mathbin{\diamondsuit}q$, then $p=q$.
\end{iclaim}
\begin{proof}
    Check that \[(a\mathbin{\diamondsuit}a)\cdot p=a\mathbin{\diamondsuit}(a\mathbin{\diamondsuit}p)=a\mathbin{\diamondsuit}(a\mathbin{\diamondsuit}q)=(a\mathbin{\diamondsuit}a)\cdot q,\]
    whence $p=q$.
\end{proof}
\begin{iclaim}
    If $p\mathbin{\diamondsuit}a=q\mathbin{\diamondsuit}a$, then $p=q$.
\end{iclaim}
\begin{proof}
    Check that \[(a\mathbin{\diamondsuit}p)\cdot a=a\mathbin{\diamondsuit}(p\mathbin{\diamondsuit}a)=a\mathbin{\diamondsuit}(q\mathbin{\diamondsuit}a)=(a\mathbin{\diamondsuit}q)\cdot a,\]
    whence $a\mathbin{\diamondsuit}p=a\mathbin{\diamondsuit}q$ and $p=q$.
\end{proof}
\begin{iclaim}
    $a\mathbin{\diamondsuit}1=a$ and $1\mathbin{\diamondsuit}(1\mathbin{\diamondsuit}a)=a$.
\end{iclaim}
\begin{proof}
    For the first part, note that $a\mathbin{\diamondsuit}(a\mathbin{\diamondsuit}1)=(a\mathbin{\diamondsuit}a)\cdot 1=a\mathbin{\diamondsuit}a$, so by Claim 1, $a\mathbin{\diamondsuit}1=a$, as desired. For the second part, check that $1\mathbin{\diamondsuit}(1\mathbin{\diamondsuit}a)=(1\mathbin{\diamondsuit}1)\cdot a=a$.
\end{proof}
\begin{iclaim}
    $a\mathbin{\diamondsuit}b=a\cdot(1\mathbin{\diamondsuit}b)$.
\end{iclaim}
\begin{proof}
    Check that \[a\mathbin{\diamondsuit}b=a\mathbin{\diamondsuit}\big(1\mathbin{\diamondsuit}(1\mathbin{\diamondsuit}b)\big)=(a\mathbin{\diamondsuit}1)\cdot(1\mathbin{\diamondsuit}b)=a\cdot(1\mathbin{\diamondsuit}b),\]
    as requested.
\end{proof}

Now, let $f(b)=1\mathbin{\diamondsuit}b$, so that $a\mathbin{\diamondsuit}b=af(b)$. Our given functional equation rewrites to \[af\big(bf(c)\big)=acf(b)\implies f\big(bf(c)\big)=cf(b).\]
However, by Claim 1, $f$ is injective, and by Claim 4, $f$ is an involution, so plugging in $f(c)$ as $c$ gives $f(bc)=f(b)f(c)$. Hence, $g(x)=\ln f(e^x)$ satisfies $g(b+c)=g(b)+g(c)$. The second condition implies that $g(x)\ge -x$ for all $x\ge 0$, so $g$ is bounded, and thus $g(x)=kx$ for some $k$.

It follows that $f(x)=x^k$, but substituting yields $k=\pm 1$. Hence, $a\mathbin{\diamondsuit}b=a\cdot b$ or $a\div b$, as desired.
