desc: Incircles IP=IQ
source: TSTST 2019/9
tags: [2019-10, oly, hard, geo, projective]

---

Let $ABC$ be a triangle with incenter $I$. Points $K$ and $L$ are chosen on segment $BC$ such that the incircles of $\triangle ABK$ and $\triangle ABL$ are tangent at $P$, and the incircles of $\triangle ACK$ and $\triangle ACL$ are tangent at $Q$. Prove that $IP=IQ$.

---

\begin{center}
    \begin{asy}
        size(12cm);
        defaultpen(fontsize(10pt));

        pen pri=royalblue;
        pen sec=deepcyan;
        pen tri=Cyan;
        pen fil=pri+opacity(0.05);
        pen sfil=sec+opacity(0.05);
        pen tfil=tri+opacity(0.05);

        real r=1.7353;
        pair A, B, C, I, IB, P,Q, IC, K, L, JB, JC, T;
        A=dir(120);
        B=dir(210);
        C=dir(330);
        I=incenter(A,B,C);
        IB=(7B+r*I)/(7+r);
        P=IB+unit(I-B)*length(IB-foot(IB,B,C));
        Q=I+unit(C-I)*length(P-I);
        IC=incenter(C,extension(A,C,Q,Q+A-foot(A,C,I)),extension(B,C,Q,Q+A-foot(A,C,I)));
        K=extension(B,C,A,2*foot(B,A,IB)-B);
        L=extension(B,C,A,2*foot(C,A,IC)-C);
        JB=incenter(A,B,L);
        JC=incenter(A,C,K);
        T=extension(IB,IC,B,C);

        draw(B--I--C,sec);
        draw(Q--T,sec);
        draw(JC--T--IC,sec);
        fill(A--B--C-- cycle,fil);
        draw(B--A--C--T,pri);
        draw(K--A--L,pri);
        filldraw(incircle(A,B,L),tfil,tri);
        filldraw(incircle(A,B,K),tfil,tri);
        filldraw(incircle(A,C,L),tfil,tri);
        filldraw(incircle(A,C,K),tfil,tri);

        dot("$A$",A,N);
        dot("$B$",B,S);
        dot("$C$",C,SE);
        dot("$T$",T,SW);
        dot("$I_B$",IB,dir(285));
        dot("$I_C$",IC,dir(255));
        dot("$J_B$",JB,NW);
        dot("$J_C$",JC,NE);
        dot("$P$",P,dir(85));
        dot("$Q$",Q,dir(120));
        dot("$K$",K,S);
        dot("$L$",L,S);
        dot("$I$",I,N);
    \end{asy}
\end{center}
Let $I_B$, $J_B$, $I_C$, and $J_C$ denote the incenters of $\triangle ABK$, $\triangle ABL$, $\triangle ACL$, and $\triangle ACK$, respectively. Also let $r_{\triangle XYZ}$ denote the inradius of $\triangle XYZ$ for all $X,Y,Z$.
\begin{lemma*}
    For any points $K$ and $L$ on $\overline{BC}$ of $\triangle ABC$, if $I_B$, $J_B$, $I_C$, and $J_C$ denote the incenters of $\triangle ABK$, $\triangle ABL$, $\triangle ACL$, and $\triangle ACK$, respectively, then $\overline{I_BI_C}$, $\overline{J_BJ_C}$, and $\overline{BC}$ concur at a point $T$.
\end{lemma*}
\begin{proof}
    Rotation by $\tfrac12\angle A$ about $A$ gives \[A(BI_B;IJ_B)=A(IJ_C;CI_C)=A(CI_C;IJ_C),\]
    and thus $(BI_B;IJ_B)=(CI_C;IJ_C)$ and the result follows.\footnote{Alternatively we can directly compute that \[A(BI_B;IJ_B)=\frac{\sin\tfrac12\angle BAC\cdot\sin\tfrac12\angle KAL}{\sin\tfrac12\angle KAC\cdot\sin\tfrac12\angle BAL}=A(CI_C;IJ_C).\]}
\end{proof}

Considering the homothety centered at $B$ sending $(I_B)$ to $(J_B)$, we can check that the scale factor is \[\frac{PJ_B}{PI_B}=\frac{r_{\triangle ABL}}{r_{\triangle ABK}}=\frac{BJ_B}{BI_B}\implies -1=(BP;I_BJ_B),\]
and similarly $-1=(CP;I_CJ_C)$. It is immediate that $\overline{PQ}$ also passes through $T$. By Menelaus on $\triangle II_BI_C$, \[-1=\frac{IP}{PI_B}\cdot\frac{I_BT}{TI_C}\cdot\frac{I_CQ}{QI}=\frac{IP}{QI}\cdot\frac{I_BT}{TI_C}\cdot\frac{I_CQ}{PI_B}=\frac{IP}{QI}\cdot\frac{r_{\triangle ABK}}{r_{\triangle ACL}}\cdot\frac{r_{\triangle ACL}}{r_{\triangle ABK}}=\frac{IP}{QI},\]
whence $IP=IQ$, as desired.
