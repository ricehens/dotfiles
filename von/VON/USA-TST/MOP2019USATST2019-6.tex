desc: Cyclic extouch points
source: MOP 2019 $+$ USA TST 2019/6
tags: [2019-10, oly, tricky, geo, projective, parallelogram, conditional, mini]

---

Let $ABC$ be a triangle with incenter $I$, and let $D$ be a point on line $BC$ satisfying $\angle AID=90^\circ$. Denote by $E$ and $F$ the feet of the altitudes from $B$ and $C$, respectively.

Let the excircle of triangle $ABC$ opposite the vertex $A$ be tangent to $\overline{BC}$ at point $A_1$. Define points $B_1$ on $\overline{CA}$ and $C_1$ on $\overline{AB}$ analogously, using the excircles opposite $B$ and $C$, respectively. Prove that:
\begin{enumerate}[label=(\alph*)]
    \item (MOP 2019) If line $EF$ is tangent to the incircle of $\triangle ABC$, then quadrilateral $AB_1A_1C_1$ is cyclic.
    \item (USA TST 2019/6) If quadrilateral $AB_1A_1C_1$ is cyclic, then $\overline{AD}$ is tangent to the circumcircle of $\triangle DB_1C_1$.
\end{enumerate}

---

\paragraph{Solution to part (a)}     It is easy to check that $\triangle ABC$ is acute, say by repeating the argument of JMO 2019/4.
\begin{center}
    \begin{asy}
        size(10cm);
        defaultpen(fontsize(10pt));

        pen pri=purple;
        pen sec=magenta;
        pen tri=fuchsia;
        pen qua=red;
        pen fil=pri+opacity(0.05);
        pen qfil=qua+opacity(0.05);

        pair B, C, L, I, A, A0, B0, C0, EE, F, A1, B1, C1, T, X, H;
        B=dir(190); C=dir(350); L=dir(270);
        I=intersectionpoint(arc(L, length(B-L), 90, 180, CCW), (-B) -- (-C));
        A=intersectionpoint(I -- (I+100*(I-L)), circle((0, 0), 1));
        A0=foot(I, B, C);
        B0=foot(I, C, A);
        C0=foot(I, A, B);
        EE=foot(B, C, A);
        F=foot(C, A, B);
        A1=B+C-A0;
        B1=foot(A1, A, C);
        C1=foot(A1, A, B);
        T=foot(I, EE, F);
        X=foot(A, B, C);
        H=orthocenter(A, B, C);

        filldraw(A -- B -- C -- cycle, fil, pri);
        filldraw(incircle(A, B, C), fil, pri);
        draw(B -- EE, sec+dashed); draw(C -- F, sec+dashed);
        draw(A -- X, sec);
        draw(B1 -- A1 -- C1 -- B1, sec);
        draw(A0 -- I -- T, tri); draw(B0 -- I -- C0 -- B0, tri);
        draw(EE -- F, pri);
        draw(T -- A0, qua);
        filldraw(circumcircle(A, B1, C1), qfil, qua);

        dot("$A$", A, N);
        dot("$B$", B, SW);
        dot("$C$", C, SE);
        dot("$A_0$", A0, S);
        dot("$B_0$", B0, unit(B0-I));
        dot("$C_0$", C0, dir(140));
        dot("$A_1$", A1, SE);
        dot("$B_1$", B1, dir(30));
        dot("$C_1$", C1, W);
        dot("$E$", EE, unit(EE-B));
        dot("$F$", F, dir(150));
        dot("$T$", T, dir(80));
        dot("$X$", X, SW);
        dot("$H$", H, dir(320));
        dot("$I$", I, N);
    \end{asy}
\end{center}
Denote the intouch points by $A_0$, $B_0$, $C_0$, the orthocenter by $H$, the $A$-excenter by $I_A$, the Bevan point\footnote{the circumcenter of the excentral triangle} by $V$, and the point where the incircle touches $\overline{EF}$ by $T$. By Brianchon's theorem on $BCB_0EFC_0$ and $BA_0CETF$, $\overline{B_0C_0}$ and $\overline{A_0T}$ intersect at $H$. Remark that $\overline{AH}\parallel\overline{IA_0}$. Since $BCEF$ is both cyclic and tangential, \[\angle A_0HB_0=\frac{\widehat{A_0B_0}+\widehat{TC_0}}2=\frac{(180^\circ-\angle B_0CA_0)+(180^\circ-\angle C_0FT)}2=90^\circ,\]
whence $\overline{HA_0}\perp\overline{B_0C_0}$. However, $\overline{AI}\perp\overline{B_0C_0}$, so $\overline{AI}\parallel\overline{HA_0}$, and $AHA_0I$ is a parallelogram.

Let $R$, $r$, $r_A$ denote the circumradius, the inradius, and the $A$-exradius, respectively. Note that $2R\cos A=AH=IA_0=r$. It is easy to see that $\triangle ABC\sim\triangle AEF$ with scale factor $\cos A$, and also $r$ is the $A$-exradius of $\triangle AEF$, so $2R=r_A$. This implies that $I_AV=2R=r_A=I_AA_1$. Since the excentral triangle is acute, $A_1=V$, whence $\angle AB_1A_1=\angle AC_1A_1=90^\circ$, and we are done. 

\paragraph{First solution to part (a), by harmonic bundles}     Let $V=\seg{I_BB_1}\cap\seg{I_CC_1}$ be the Bevan point of $\triangle ABC$. We know that $\seg{VA_1}\perp\seg{BC}$, $\seg{VB_1}\perp\seg{CA}$, $\seg{VC_1}\perp\seg{AB}$, so $V$ lies on $(AB_1C_1)$. If $A_1\ne V$, then since $\seg{AV}$ is a diameter of $(AB_1C_1)$, we have $\seg{AV}\parallel\seg{BC}$, which is absurd. Thus $V=A_1$, and $\seg{AA_1}$ is a diameter of $(AB_1C_1)$. 

By Pappus' theorem on $BACI_CA_1I_B$, we have $I\in\seg{B_1C_1}$. Denote by $X$ the foot of the altitude from $A$, so that it lies on $(AB_1C_1)$. Notice that \[-1=(A,\overline{BC}\cap\overline{I_BI_C};I_B,I_C)\stackrel{A_1}=(AX;B_1C_1).\]
Let the tangents to $(AB_1C_1)$ at $A$ and $X$ meet at $M$ on $\seg{B_1C_1}$. Since $I$ is the foot of the $A$-angle bisector of $\triangle AB_1C_1$ and $AB_1XC_1$ is harmonic, $(AIX)$ is the $A$-Apollonius circle of $\overline{B_1C_1}$.
\begin{center}
    \begin{asy}
        size(10cm);
        defaultpen(fontsize(10pt));

        pen pri=royalblue+linewidth(0.5);
        pen sec=Cyan+linewidth(0.5);
        pen tri=springgreen+linewidth(0.5);
        pen qua=chartreuse+linewidth(0.5);
        pen fil=pri+opacity(0.05);
        pen sfil=sec+opacity(0.05);
        pen tfil=tri+opacity(0.05);
        pen qfil=qua+opacity(0.05);

        pair B,C,L,I,A,A1,B1,C1,EE,F,J,D,M,K,IB,IC,X;
        B=dir(190); C=dir(350); L=dir(270);
        I=intersectionpoint(arc(L,length(B-L),90,180,CCW),(-B)--(-C));
        A=intersectionpoint(I--(I+100*(I-L)),circle((0,0),1));
        A1=B+C-foot(I,B,C);
        B1=foot(A1,A,C);
        C1=foot(A1,A,B);
        EE=foot(I,A,C);
        F=foot(I,A,B);
        J=reflect(circumcenter(A,B,C),circumcenter(A,EE,F))*A;
        D=extension(A,J,B,C);
        M=(A+D)/2;
        K=dir(90);
        IB=extension(B,I,A,K);
        IC=extension(C,I,A,K);
        X=foot(A,B,C);

        filldraw(circumcircle(A,X,D),qfil,qua);
        draw(B--IB,tri); draw(C--IC,tri);
        filldraw(circumcircle(A,B1,C1),tfil,tri);
        //filldraw(circumcircle(A,EE,F),sfil,sec);
        draw(IB--A1--IC,sec+dashed);
        draw(IB--IC,sec);
        draw(D--I--A,sec);
        draw(M--X,sec);
        filldraw(circumcircle(A,B,C),fil,pri);
        draw(B--D--A,pri);
        draw(M--B1,pri);
        draw(A--B--C--A,pri);

        //clip((L+(100,-1/8))--(L+(-100,-1/8))--(-100,100)--(100,100)--cycle);

        dot("$A$",A,NW);
        dot("$B$",B,SW);
        dot("$C$",C,SE);
        //dot("$L$",L,S);
        dot("$I$",I,dir(60));
        dot("$A_1$",A1,dir(300));
        dot("$B_1$",B1,dir(20));
        dot("$C_1$",C1,dir(305));
        //dot("$E$",EE,NE);
        //dot("$F$",F,NW);
        //dot("$J$",J,dir(150));
        dot("$D$",D,SW);
        dot("$M$",M,dir(195));
        //dot("$K$",K,N);
        dot("$I_B$",IB,NE);
        dot("$I_C$",IC,S);
        dot("$X$",X,dir(255));
    \end{asy}
\end{center}
Since $\angle DIA=90^\circ=\angle DXA$, $AIXD$ is cyclic. However $M$ is the circumcenter of $\triangle AIX$, so $M$ is the midpoint of $\overline{AD}$, and $MD^2=MA^2=MB_1\cdot MC_1$. This completes the proof.

\paragraph{Second (official) solution to part (b), by angle chasing}     Let $V$ be the antipode of $A$ on $(AB_1C_1)$. Let the incircle of $\triangle ABC$ touch $\overline{CA}$ and $\overline{AB}$ at $E$ and $F$, respectively, and let $J$ be the Miquel point of $BCEF$. Furthermore, let $M$ be the midpoint of $\overline{AD}$.

We claim that $\overline{AA_1}$ is a diamter of $(AB_1A_1C_1)$. Note that $V$ is the Bevan point of $\triangle ABC$, so $\overline{VA_1}\perp\overline{BC}$. Furthermore, if $V\ne A_1$, then $\overline{VA_1}\perp\overline{AA_1}$, which would require that $A\in\overline{BC}$, which is absurd.

By Pappus' theorem on $\overline{BA_1C}$ and $\overline{I_CAI_B}$, $I$ lies on $\overline{B_1C_1}$, and by the Radical Axis theorem on $(AI)$, $(ABC)$, and $(BIC)$, $J$ lies on $\overline{AD}$. Since $\triangle JBF\sim\triangle JCE$, \[\frac{AC_1}{AB_1}=\frac{JF}{JE}=\frac{JB}{JC},\]
and also $\measuredangle C_1AB_1=\measuredangle BAC=\measuredangle BJC$, so $\triangle AC_1B_1\sim\triangle JBC$. 

This implies that \[\measuredangle DAB=\measuredangle JAF=\measuredangle JEF=\measuredangle JCB=\measuredangle AB_1C_1,\]
so $\overline{AD}$ is tangent to $(AB_1C_1)$. Moreover, \[\measuredangle MIA=\measuredangle IAM=\measuredangle IAC_1+\measuredangle C_1AM=\measuredangle B_1AI+\measuredangle C_1B_1A=\measuredangle B_1IA,\]
whence $M$ lies on $\overline{B_1IC_1}$. Hence, $MD^2=MA^2=MB_1\cdot MC_1$, and the desired result follows. 


