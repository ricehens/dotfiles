author: Linus Hamilton
desc: Usamons and diodes
source: USA TST 2015/3
tags: [2020-04, oly, medium, combo, anti, trash, waltz]

---

A physicist encounters $2015$ atoms called usamons. Each usamon either has one electron or zero electrons, and the physicist can't tell the difference. The physicist's only tool is a diode. The physicist may connect the diode from any usamon $A$ to any other usamon $B$. (This connection is directed.) When she does so, if usamon $A$ has an electron and usamon $B$ does not, then the electron jumps from $A$ to $B$. In any other case, nothing happens. In addition, the physicist cannot tell whether an electron jumps during any given step. The physicist's goal is to isolate two usamons that she is $100\%$ sure are currently in the same state. Is there any sequence of diode usage that makes this possible?

---

\paragraph{First solution, by invariants} The answer is no. Place the usamons in a single-file line. Let the signature be the binary string $\{0,1\}^{2015}$ such that bit $i$ is $1$ if and only if the $i$th usamon from the left has an electron.

Consider the $2016$ signatures $(0,0,\ldots,0)$, $(1,0,\ldots,0)$, $(1,1,\ldots,0)$, $\ldots$, $(1,1,\ldots,1)$. It is useless to connect the diode from usamon $A$ to usamon $B$, with $A>B$. Assume that if we connect the diode from usamon $A$ to usamon $B$, with $A<B$, we also swap $A$ and $B$ in the line. Then the signatures are invariant, and repeat.

\paragraph{Second, official solution (Linus Hamilton)} The answer is no. Allow the electrons to each carry any real number. If the diode points from $x$ to $y$, swap them iff $x>y$. Then if we place the usamons in a line, our procedure is basically a sorting machine.

Give each usamon a distinct real number, and let the physicist perform a series of moves. Then she selects two usamons labeled $a$, $b$ with $a\ne b$. Assume without loss of generality $a<b$.

Now go back and say the usamons with labels $\ge b$ had a usamon, and the others didn't. Contradiction.
