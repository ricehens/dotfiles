desc: g(x)+100x all bijective
source: USA TST 2019/2
tags: [2019-12, oly, brutal, nt, vp, sums, nice, involved, favorite]

---

Let $\mathbb Z/n\mathbb Z$ denote the set of integers considered modulo $n$ (hence $\mathbb Z/n\mathbb Z$ has $n$ elements). Find all positive integers $n$ for which there exists a bijective function $g:\mathbb Z/n\mathbb Z\to\mathbb Z/n\mathbb Z$, such that the 101 functions \[g(x),\quad g(x)+x,\quad g(x)+2x,\quad\ldots,\quad g(x)+100x\]
are all bijections on $\mathbb Z/n\mathbb Z$.

---

The answer is all $n$ with no prime factors less than or equal to $101$. To see that $g$ exists for such $n$, just take $g(x)=x$.

Let $p$ be the smallest prime factor of $n$, and assume that $p\le101$. We will show \[g(x),\quad g(x)+x,\quad g(x)+2x,\quad\ldots,\quad g(x)+(p-1)x\]
cannot all be bijections on $\mathbb Z/n\mathbb Z$.
\begin{claim*}
    We can eliminate $n$ even; i.e.\ henceforth assume $n$ is odd and $p\ge3$.
\end{claim*}
\begin{proof}
    Assume for contradiction $g(x)$ and $g(x)+x$ are bijective. Note that \[\sum_{x=0}^{n-1}g(x)\equiv\sum_{x=0}^{n-1}\Big(g(x)+x\Big)\implies0\equiv\sum_{x=0}^{n-1}x\equiv\frac{n(n-1)}2\pmod n,\]
    thus $n$ is odd.
\end{proof}
\setcounter{lemma}0
\begin{lemma}
    If $e=\nu_p(n)$, then \[\nu_p\left(\sum_{x=0}^{n-1}x^{p-1}\right)=e-1.\]
\end{lemma}
\begin{proof}
    Note that \[\sum_{x=0}^{n-1}x^{p-1}\equiv\frac n{p^e}\sum_{x=0}^{p^e-1}x^{p-1}\pmod{p^e},\]
    so we prove $\nu_p\left(\sum_{x=0}^{p^e-1}x^{p-1}\right)=e-1$. Define \[T_k:=\sum_{\nu_p(x)=k}x^{p-1},\quad\text{so that}\quad\sum_{x=0}^{p^e-1}x^{p-1}\equiv\sum_{k=0}^eT_k\pmod{p^e}.\]
    If $g$ denotes a primitive root mod $p^{e-k}$, then \[T_k\equiv p^{k(p-1)}\sum_{i=0}^{\vphi(p^{e-k})-1}g^{i(p-1)}\equiv\frac{g^{(p-1)\vphi(p^{e-k})}-1}{g^{p-1}-1}p^{k(p-1)}\pmod{p^e}.\]
    By Lifting the Exponent, \[\nu_p\left(g^{(p-1)\vphi(p^{e-k})}-1\right)=\nu_p\left(g^{p-1}-1\right)+e-k-1,\]
    so $\nu_p(T_k)=e-k-1+k(p-1)=e-1+k(p-2)$, which equals $e-1$ when $k=0$ and is at least $e$ otherwise. Thus \[\nu_p\left(\sum_{k=0}^eT_k\right)=\nu_p(T_0)=e-1,\]
    as desired.
\end{proof}
\begin{lemma}
    If such a function $g$ exists, then \[\sum_{x=0}^{n-1}x^{p-1}\equiv0\pmod{p^e}.\]
\end{lemma}
\begin{proof}
    Consider the polynomial in $k$ \[f(k):=\sum_{k=0}^{n-1}\Big(g(x)+kx\Big)^{p-1}-\sum_{x=0}^{n-1}x^{p-1}.\]
    Note that $f$ has degree $p-1$, but $p^e\mid f(x)$ for $x=0,\ldots,p-1$.

    I claim that if $f$ is a polynomial with degree at most $p-1$ and $p^e\mid f(x)$ for $x=0,\ldots,p-1$, then all coefficients of $f$ are divisible by $p^e$. This proves the lemma by looking at the leading coefficient.

    We proceed by induction on $e$. The base case $e=1$ is as follows: in $\mathbb F_p$, $f$ has $p$ roots but degree at most $p-1$, so it is the zero polynomial.

    For the inductive step, if $p^e\mid f(x)$ for all $x=0,\ldots,p-1$, by the inductive hypothesis, the coefficients of $f(x)$ are divisible by $p^{e-1}$. But $g(x):=f(x)/p$ is still always divisible by $p^{e-1}$, so by the inductive step, the coefficients of $g$ are all divisible by $p^{e-1}$. Thus the coefficients of $f$ are all divisible by $p^e$, as desired.
\end{proof}

We conclude by combining Lemma 1 and Lemma 2.

