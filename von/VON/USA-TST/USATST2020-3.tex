author: Nikolai Beluhov
desc: Greek god flooding
source: USA TST 2020/3
tags: [2019-12, oly, brutal, combo, construction, optimization, nice, involved, favorite]

---

Let $\alpha\ge1$ be a real number. Hephaestus and Poseidon play a turn-based game on an infinite grid of unit squares. Before the game starts, Poseidon chooses a finite number of cells to be \emph{flooded}. Hephaestus is building a \emph{levee}, which is a subset of unit edges of the grid (called \emph{walls}) forming a connected, non-self-intersecting path or loop.%\footnote{More formally, there must exist lattice points $A_0$, $A_1$, $\ldots$, $A_k$, pairwise distinct except possibly $A_0=A_k$, such that the set of walls is exactly $\{A_0A_1,A_1A_2,\ldots,A_{k-1}A_k\}$. Once a wall is built it cannot be destroyed; in particular, if the levee is a closed loop (i.e.\ $A_0=A_k$) then Hephaestus cannot add more walls. Since each wall has length $1$, the length of the levee is $k$.}.

The game begins with Hephaestus moving first. On each of Hephaestus's turns, he adds one or more walls to the levee, as long as the total length of the levee is at most $\alpha n$ after his $n$th turn. On each of Poseidon's turns, every cell which is adjacent to an already flooded cell and with no wall between them becomes flooded as well.

Hephaestus wins if the levee forms a closed loop such that all flooded cells are contained in the interior of the loop --- hence stopping the flood and saving the world. For which $\alpha$ can Hephaestus guarantee victory in a finite number of turns no matter how Poseidon chooses the initial cells to flood?

---

The answer is $\alpha>2$.

\bigskip

\textbf{Proof of sufficiency:}     Take some $\alpha>2$. We show it is possible to contain the flood. Our strategy is as follows. Here the blue circle is a large region (that grows in both directions at a rate of $1$ cell per move) that contains all the flooded cells.
\begin{center}
    \begin{asy}
        size(7cm); defaultpen(fontsize(12pt));
        pen pri=black+linewidth(2);
        pen sec=black+linewidth(1);
        pen fil=royalblue;
        pair lbl=2*S;
        real r=1/2;
        pair A,B,C,D,O,del;

        del=(0,0);
        A=(0,0)+del; B=(0,4)+del; C=(6,4)+del; D=(6,0)+del; O=(A+C)/2;
        draw(A--B--C--D--cycle,pri);
        filldraw(circle(O,r),fil);
        draw( (A+(1,1))--(B+(1,-1)),sec);
        label("Step I: Build",D--A,lbl);

        del=(8,0);
        A=(0,0)+del; B=(0,4)+del; C=(6,4)+del; D=(6,0)+del; O=(A+C)/2;
        draw(A--B--C--D--cycle,pri);
        filldraw(circle(O,r),fil);
        draw( ( (A+D)/2+(r,1))--(A+(1,1))--(B+(1,-1))--( (B+C)/2+(r,-1)),sec);
        label("Step II: Engulf",D--A,lbl);

        del=(0,-6);
        A=(0,0)+del; B=(0,4)+del; C=(6,4)+del; D=(6,0)+del; O=(A+C)/2;
        draw(A--B--C--D--cycle,pri);
        filldraw(circle(O,r),fil);
        draw( (D+(-1,1))--(A+(1,1))--(B+(1,-1))--(C+(-1,-1)),sec);
        label("Step III: Zoom",D--A,lbl);

        del=(8,-6);
        A=(0,0)+del; B=(0,4)+del; C=(6,4)+del; D=(6,0)+del; O=(A+C)/2;
        draw(A--B--C--D--cycle,pri);
        filldraw(circle(O,r),fil);
        draw( (D+(-1,1))--(A+(1,1))--(B+(1,-1))--(C+(-1,-1))--cycle,sec);
        label("Step IV: Eat",D--A,lbl);
    \end{asy}
\end{center}
\begin{enumerate}[label=\Roman*.]
    \item \textbf{Build a giant wall.} The total vertical height of the flood changes by at most $2$ a move. Start by building a wall sufficiently far away of arbitrary height. Since $\alpha>2$, the wall can be arbitrarily tall compared to the flood, while remaining a constant distance away from the center of the flood (since the wall can stop the flood from spreading to the other side).
    \item \textbf{Engulf the flood.} After the wall is sufficiently large, begin constructing walls rightward until the rightmost point on our walls is to the right of the rightmost point of the flood.

        The flood moves rightward at a rate of at most $1$ cell per move, while we can alternate between extending the top wall and the bottom wall, each increasing at a rate of $\alpha/2>1$ cells per move. If the original wall was large enough, the wall can extend past the flood without colliding into it, as the distance from the rightmost point of the wall and the rightmost point of the flood decreases by $\alpha/2-1$ cells each move.
    \item \textbf{Zoom past the flood.} Now, we essentially repeat the above process. The wall can be built rightward at a rate of $\alpha/2>1$, so we may extend an arbitrarily large distance past the rightmost point of the flood.
    \item \textbf{Eat the flood.} Finally, build the eastern wall. If we have ``zoomed'' sufficiently far past the flood, we can contain the entire flood, thus completing the process.
\end{enumerate}
Thus if $\alpha>2$, Hephaestus can stop the flood and save the world.

\bigskip

\textbf{Proof of necessity:}     Given two cells $A$ and $B$ in the flood, let $L$ be the shortest path between $A$ and $B$ (including the endpoints) that does not intersect the levee, say it has length $|L|$, and furthermore say the levee has perimeter $P$.
\begin{claim*}
    For any shortest path $L$ between two cells $A$ and $B$, we have $2(|L|+1)\le P$.
\end{claim*}
\begin{proof}
    From every cell in $L$, draw rays emanating from the center of that cell that do not lie along $L$; thus two rays are drawn from every cell except the endpoints, from which three are drawn. Stop these rays once they hit the levee (so that they are now segments), and call them \emph{beams}.

    I claim that no point on the levee lies on two or more beams. Beams are either parallel or perpendicular, but perpendicular beams intersect at the center of a cell, and thus not on the levee, so for two beams to intersect on the levee, the lines containing them must coincide and the corresponding rays must be in the same direction.

    Let these two rays emanate from $X$ and $Y$. Then by definition of these beams, $L$ does not contain segment $XY$. But $\seg{XY}$ does not intersect the levee, so we may instead go directly from $X$ to $Y$, contradicting the assumption that $L$ is the shortest path.

    With this, the claim readily follows.
\end{proof}

Suppose Poseidon starts with a flooded square $A$ surrounded by four flooded squares, and assume Hephaestus can stop the flood. Call the final state of the levee when the flood is sealed off the \emph{final levee}. Let $B$ be a point in the contained flood, and let $L$ be the shortest path between $A$ and $B$ that does not intersect the final levee. Suppose that $|L|$ is maximal among all $B$ in the contained flood. Again let $P$ be the perimeter of the final levee.

Note that if there is a wall at any point, it must be a part of the final levee. The flood will grow along $L$ until it reaches $B$. Since it already has a $1$ cell head-start (since $A$ is surrounded by four flooded squares) and we have assumed that $|L|$ is maximal, at most $|L|-1$ moves have passed. It follows that Hephaestus has built at most $\alpha(|L|-1)$ walls, so \[\alpha(|L|-1)\ge P\ge2(|L|+1)\implies\alpha>2,\]
and we are done.


