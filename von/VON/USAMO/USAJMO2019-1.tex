desc: Apples and pears invariant
source: USAJMO 2019/1
tags: [2019-10, oly, trivial, combo, invariant, process]

---

There are $a+b$ bowls arranged in a row, numbered $1$ through $a+b$, where $a$ and $b$ are given positive integers. Initially, each of the first $a$ bowls contains an apple, and each of the last $b$ bowls contains a pear.

A legal move consists of moving an apple from bowl $i$ to bowl $i+1$ and a pear from bowl $j$ to bowl $j-1$, provided that the difference $i-j$ is even. We permit multiple fruits in the same bowl at the same time. The goal is to end up with the first $b$ bowls each containing a pear and the last $a$ bowls each containing an apple. Show that this is possible if and only if the product $ab$ is even.


---

The problem consists of two parts:

\bigskip

\textbf{Proof of sufficiency:} Assume $ab$ is even, and since the problem is symmetric under reflection, without loss of generality let $a$ be even. I will show by induction on $b$ that the goal is always possible.

Consider $b=0$ as the base case --- there is nothing to show. Take the leftmost pear (in bowl $a+1$), and move it all the way to bowl $1$; when the pear moves from bowl $n$ to bowl $n-1$, also move the apple in bowl $a+2-n$.

Each apple moves exactly once, and so after this process, the leftmost bowl contains a pear, the next $a$ bowls each contain an apple, and the rightmost $b-1$ bowls contain a pear. Then apply the induction hypothesis on $b-1$ to sort the remaining $a+b-1$ apples.

\bigskip

\textbf{Proof of necessity:} Assume $ab$ is odd. We can check that in total, $ab$ moves occur. There are two types of moves:
\begin{itemize}[itemsep=0em]
    \item Moving two fruit from odd-numbered bowls to even-numbered bowls.
    \item Moving two fruit from even-numbered bowls to odd-numbered bowls.
\end{itemize}
Since in both the beginning and the end of the process, each bowl contains a fruit, the two types of moves occur equally often. This is absurd, since $ab$ is odd.


