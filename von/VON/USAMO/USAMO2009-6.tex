desc: (s_i-s_j)r and (t_i-t_j)/r integers
source: USAMO 2009/6
tags: [2019-11, oly, hard, nt, vp, nice]

---

Let $s_1$, $s_2$, $s_3$, $\ldots$ be an infinite, nonconstant sequence of rational numbers, meaining it is not the case that $s_1=s_2=s_3=\cdots$. Suppose that $t_1$, $t_2$, $t_3$, $\ldots$ is also an infinite, nonconstant sequence of rational numbers with the property that $(s_i-s_j)(t_i-t_j)$ is an integer for all $i$ and $j$. Prove that there exists a rational number $r$ such that $(s_i-s_j)r$ and $(t_i-t_j)/r$ are integers for all $i$ and $j$.

---

It suffices to focus on a single prime $p$. The given condition tells us that \[\nu_p(s_i-s_j)+\nu_p(t_i-t_j)\in\mathbb Z\quad\text{for all }i,j.\]
\setcounter{claim}0
\begin{claim}
    For all $i$, $j$, $k$, we have $\nu_p(s_i-s_j)+\nu_p(t_i-t_k)\ge0$.
\end{claim}
\begin{proof}
    Assume for contradiction otherwise. Note that \[\nu_p(s_i-s_j)<-\nu_p(t_i-t_k)\le\nu_p(s_i-s_k),\]
    from which $\nu_p(s_j-s_k)=\nu_p(s_i-s_j)$. Similarly $\nu_p(t_j-t_k)=\nu_p(t_i-t_k)$. Summing, \[\nu_p(s_j-s_k)+\nu_p(t_j-t_k)=\nu_p(s_i-s_j)+\nu_p(t_i-t_k)<0,\]
    contradiction.
\end{proof}
\begin{claim}
    For all $i$, $j$, $k$, $l$, we have $\nu_p(s_i-s_j)+\nu_p(t_k-t_\ell)\ge0$.
\end{claim}
\begin{proof}
    This proof is similar to that of Claim 1. Assume for contradiction otherwise. Note that \[\nu_p(s_i-s_j)<-\nu_p(t_k-t_\ell)\le\nu_p(s_i-s_k),\]
    where the second inequality is by Claim 1. It follows that $\nu_p(s_i-s_k)=\nu_p(s_i-s_j)$. Similarly $\nu_p(t_i-t_k)=\nu_p(t_k-t_\ell)$, and summing, \[\nu_p(s_i-s_k)+\nu_p(t_i-t_k)=\nu_p(s_i-s_j)+\nu_p(t_k-t_\ell)<0,\]
    contradiction.
\end{proof}

With these claims, it is clear that \[\min_{i,j}\nu_p(s_i-s_j)+\min_{i,j}\nu_p(t_i-t_j)\ge0,\]
so it is possible to choose $\nu_p(r)$ so that $\nu_p(s_i-s_j)+\nu_p(r)\ge0$ and $\nu_p(t_i-t_j)-\nu_p(r)\ge0$ for all $i$, $j$. This completes the proof.
