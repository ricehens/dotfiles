desc: a_ia_j <= i+j
source: USAMO 2010/3
tags: [2020-03, oly, hard, alg, sequence, ineq, involved]

---

The $2010$ positive real numbers $a_1$, $a_2$, $\ldots$, $a_{2010}$ satisfy the identity $a_ia_j\le i+j$ for all $1\le i<j\le2010$. Determine, with proof, the largest possible value of the product $a_1a_2\cdots a_{2010}$.

---

It is the obvious bound $3\cdot7\cdot11\cdots4019$, with upper bound obtained by \[a_1a_2\cdots a_{2010}=(a_1a_2)\cdots(a_{2019}a_{2020})\le3\cdots4019.\]
I am aware of two constructions, shown below. In both, $a_{2n}a_{2n-1}=2n-1$, so equality holds.
\begin{remark}
    Some preliminary remarks on motivation: for constructions, it turns out the most concerning bounds are those where $|i-j|=1$, and overall the bounds are more concerning when $i$, $j$ increase (hence why we fix $a_{2008}a_{2010}$ in the second construction), so we should push for equality in these cases.

    This gives the desired upper bound, and to achieve this, heuristically we want $a_n\approx\sqrt{2n}$. Both the below solutions build upon this estimate.
\end{remark}
\paragraph{First, explicit construction}     Let $a_{2n}=2\sqrt n$ and $a_{2n-1}=2\sqrt n-\frac1{2\sqrt n}$. We manually verify the required inequality:
\begin{itemize}
    \item Let $i=2a$, $j=2b$, where $a<b$. Then \[a_ia_j=\left(2\sqrt a\right)\left(2\sqrt b\right)=4\sqrt{ab}<2(a+b).\]
    \item Let $i=2a-1$, $j=2b-1$, where $a<b$. Note the estimate
        \[\frac1{4\sqrt{ab}}\le\frac1{4\sqrt{a(a+1)}}<\frac2{\left(\sqrt{a+1}-\sqrt a\right)^2}=2\left(\sqrt{a+1}-\sqrt a\right)^2\le2\left(\sqrt b-\sqrt a\right)^2,\]
        From this, we have \[a_ia_j=\left(2\sqrt a-\frac1{2\sqrt a}\right)\left(2\sqrt b-\frac1{2\sqrt b}\right)<4\sqrt{ab}+\frac1{4\sqrt{ab}}-2\le2(a+b)-2.\]
    \item Let $i=2a-1$, $j=2b$, where $a\le b$. Then \[a_ia_j=\left(2\sqrt a-\frac1{2\sqrt a}\right)\left(2\sqrt b\right)=4\sqrt{ab}-\sqrt{\frac ba}\le2(a+b)-1.\]
    \item Let $i=2a$, $j=2b-1$, where $a<b$. Note the estimate \[\sqrt{\frac1b}=\frac2{2\sqrt b}\le\frac{2(b-a)}{\sqrt b+\sqrt a}=2\left(\sqrt b-\sqrt a\right)\implies 1-\sqrt{\frac ab}=2\left(\sqrt b-\sqrt a\right)^2.\]
        From this, we have \[a_ia_j=\left(2\sqrt a\right)\left(2\sqrt b-\frac1{2\sqrt b}\right)=4\sqrt{ab}-\sqrt{\frac ab}\le2(a+b)-1.\]
\end{itemize}
Hence $a_ia_j\le i+j$ for all $i$, $j$.

\paragraph{Second, more motivated construction}     Take the sequence with $a_na_{n+1}=2n+1$ for each $n$ and $a_{2008}a_{2010}=4028$. Note that the inequality holds when $j-1=1$ by definition. We first verify the required inequality for $j-i=2$. Backwards induct on $i$, with the base case $i=2008$ given and the case $i=2007$ clear by noting \[a_{2007}a_{2009}=\frac{4015}{a_{2018}}\cdot\frac{4017}{a_{2010}}=\frac{4015\cdot4017}{4018}\le4017.\]
If the inequality holds for integers greater than $i$, then \[a_ia_{i+2}=\frac{(a_ia_{i+1})(a_{i+2}a_{i+3})(a_{i+2}a_{i+4})}{(a_{i+1}a_{i+2})(a_{i+3}a_{i+4})}\le\frac{(2i+1)(2i+5)(2i+6)}{(2i+3)(2i+7)}\le2i+2,\]
thus settling $j-i=2$. We now prove the inequality for all $i$, $j$, by induction on $j-i$ with increment $2$. The base cases $j-i\in\{1,2\}$ have already been proven.

If $a_ia_j\le i+j$, then \[a_ia_{j+2}\le\frac{a_{j+2}}{a_j}(i+j)=\frac{a_{j+1}a_{j+2}}{a_ja_{j+1}}(i+j)=\frac{2j+3}{2j+1}(i+j)\le i+j+2,\]
where the last inequality holds since \[\frac{2j+3}{2j+1}(i+j)\le i+j+2\iff\frac2{2j+1}(i+j)\le 2\iff i<j.\]
This completes the proof.

