desc: MAN is KID
source: USAMO 2017/3
tags: [2019-10, oly, medium, geo, reference, queue-point]

---

Let $ABC$ be a scalene triangle with circumcircle $\Omega$ and incenter $I$. Ray $AI$ meets $\overline{BC}$ at $D$ and meets $\Omega$ again at $M$; the circle with diameter $\overline{DM}$ cuts $\Omega$ again at $K$. Lines $MK$ and $BC$ meet at $S$, and $N$ is the midpoint of $\overline{IS}$. The circumcircles of $\triangle KID$ and $\triangle MAN$ intersect at points $L_1$ and $L_2$. Prove that $\Omega$ passes through the midpoint of either $\overline{IL_1}$ or $\overline{IL_2}$.

---

The obvious first step: let $W$ be the midpoint of arc $BAC$, so $W$, $D$, $K$ collinear. By $AB/AC=DB/DC=KB/KC$, we have $-1=(AK;BC)$, and thus $\seg{AS}$ is the external bisector of $\angle BAC$.
\begin{center}
\begin{asy}
    size(12cm); defaultpen(fontsize(10pt));
    pen pri=heavyred;
    pen sec=orange;
    pen tri=olive;
    pen qua=purple+pink;
    pen fil=lightred+opacity(0.05);
    pen sfil=orange+opacity(0.05);
    pen tfil=yellow+opacity(0.05);
    pen qfil=purple+pink+opacity(0.05);

    pair O,A,B,C,I,D,M,WW,K,SS,NN,IA,IB,IC,T,L;
    O=(0,0);
    A=dir(140);
    B=dir(210);
    C=dir(330);
    I=incenter(A,B,C);
    D=extension(A,I,B,C);
    M=dir(270);
    WW=dir(90);
    K=2*foot(O,WW,D)-WW;
    SS=extension(M,K,B,C);
    NN=(I+SS)/2;
    IA=2M-I;
    IB=extension(B,I,IA,C);
    IC=extension(C,I,IA,B);
    T=2*foot(O,I,WW)-WW;
    L=2T-I;

    filldraw(circumcircle(IA,IB,IC),qfil,qua+dashed);
    draw(IA--A,qua+Dotted);
    filldraw(IA--IB--IC--cycle,qfil,qua);
    draw(IC--SS,qua);
    filldraw(circumcircle(M,A,NN),tfil,tri);
    filldraw(circumcircle(K,I,D),tfil,tri);
    draw(I--SS,tri+dashed);
    draw(WW--L,tri);
    filldraw(circumcircle(B,I,C),sfil,sec);
    draw(WW--K,sec+dashed);
    draw(IA--SS,sec);
    draw(M--SS,sec);
    filldraw(circle(O,1),fil,pri);
    filldraw(A--B--C--cycle,fil,pri);
    draw(B--SS,pri);

    dot("$A$",A,dir(110));
    dot("$B$",B,dir(215));
    dot("$C$",C,dir(-10));
    dot("$I$",I,N);
    dot("$D$",D,SE);
    dot("$M$",M,SE);
    dot("$W$",WW,N);
    dot("$K$",K,dir(285));
    dot("$S$",SS,W);
    dot("$N$",NN,NW);
    dot("$I_A$",IA,S);
    dot("$I_B$",IB,NE);
    dot("$I_C$",IC,NW);
    dot("$T$",T,dir(195));
    dot("$L$",L,dir(240));
\end{asy}
\end{center}
To prove the problem, we will describe a point $L$ with the three required properties: (i) the midpoint of $\seg{IL}$ lies on $(ABC)$, (ii) $L$ lies on $(MAN)$, and (iii) $L$ lies on $(KID)$.

Let $\seg{WI}$ intersect $(ABC)$ again at $T$, and let $L$ be the reflection of $I$ over $T$. By design $L$ obeys condition (i).

To prove condition (ii), I contend $M$, $A$, $N$, $L$ lie on the nine-point circle of $\triangle I_ASW$. Note that $\angle ILI_A=\angle ITM=90\dg$, so $\seg{I_AA}\perp\seg{WS}$ and $\seg{WL}\perp\seg{I_AS}$. It follows that $I$ is the orthocenter of $\triangle I_ASW$. Then the hypothesis in (ii) becomes clear.

Finally to verify (iii), recall that $\seg{WB}$ is tangent to $(BIC)$, thus $WI\cdot WL=WB^2=WD\cdot WK$ by Shooting lemma. This completes the proof.
\begin{remark}
    This is really an orthocenter problem in terms of $\triangle I_AI_BI_C$, with orthic triangle $\triangle ABC$. The desired point $L$ is the so-called ``Queue point'' of $\triangle I_AI_BI_C$.
\end{remark}
%\begin{center}
%    \begin{asy}
%        size(10cm);
%        defaultpen(fontsize(10pt));
%
%        pen pri=red+linewidth(0.5);
%        pen sec=orange+linewidth(0.5);
%        pen tri=fuchsia+linewidth(0.5);
%        pen fil=red+opacity(0.05);
%        pen sfil=orange+opacity(0.05);
%        pen tfil=fuchsia+opacity(0.05);
%
%        pair IA, IB, IC, A, B, C, I, M, D, P, Mp, SS, L, K, NN, IAp;
%        IA=dir(110);
%        IB=dir(230);
%        IC=dir(310);
%        A=foot(IA, IB, IC);
%        B=foot(IB, IC, IA);
%        C=foot(IC, IA, IB);
%        I=orthocenter(IA, IB, IC);
%        M=(IA+I)/2;
%        D=extension(A, IA, B, C);
%        P=(B+C)/2;
%        Mp=(IB+IC)/2;
%        SS=extension(B, C, IB, IC);
%        L=intersectionpoint(SS -- (IA+(SS-IA)*0.01), circumcircle(IA, IB, IC));
%        K=intersectionpoint(SS -- (M+(SS-M)*0.01), circumcircle(A, B, C));
%        NN=(SS+I)/2;
%        IAp=-IA;
%
%        filldraw(circumcircle(IA, IB, IC), fil, pri);
%        filldraw(circumcircle(I, SS, A), fil, pri);
%        filldraw(circumcircle(A, B, C), sfil, sec);
%        filldraw(circumcircle(D, P, M), sfil, sec);
%        filldraw(circumcircle(M, A, NN), tfil, tri);
%        filldraw(circumcircle(K, I, D), tfil, tri);
%
%        draw(IA -- IB -- IC -- IA, pri);
%        draw(IA -- SS -- IB, pri); draw(B -- SS -- M, pri);
%        draw(IA -- A, sec+dashed); draw(IB -- B, sec+dashed); draw(IC -- C, sec+dashed);
%        draw(K -- Mp, tri); draw(L -- IAp, tri); draw(M -- Mp, tri);
%        draw(SS -- I, tri+dashed);
%
%        dot("$I_A$", IA, N);
%        dot("$I_B$", IB, IB);
%        dot("$I_C$", IC, IC);
%        dot("$A$", A, dir(285));
%        dot("$B$", B, E);
%        dot("$C$", C, NW);
%        dot("$I$", I, NE);
%        dot("$M$", M, dir(75));
%        dot("$D$", D, dir(60));
%        dot("$P$", P, dir(75));
%        dot("$X$", Mp, S);
%        dot("$S$", SS, SW);
%        dot("$L$", L, W);
%        dot("$K$", K, NW);
%        dot("$N$", NN, S);
%        dot("$I_A'$", IAp, IAp);
%    \end{asy}
%\end{center}
