desc: MAN is KID
source: USAMO 2017/3
tags: [2019-10, oly, medium, geo, reference, queue-point]

---

Let $ABC$ be a scalene triangle with circumcircle $\Omega$ and incenter $I$. Ray $AI$ meets $\overline{BC}$ at $D$ and meets $\Omega$ again at $M$; the circle with diameter $\overline{DM}$ cuts $\Omega$ again at $K$. Lines $MK$ and $BC$ meet at $S$, and $N$ is the midpoint of $\overline{IS}$. The circumcircles of $\triangle KID$ and $\triangle MAN$ intersect at points $L_1$ and $L_2$. Prove that $\Omega$ passes through the midpoint of either $\overline{IL_1}$ or $\overline{IL_2}$.

---

Let $I_A$, $I_B$, and $I_C$ be the $A$-, $B$-, and $C$-excenters, respectively. The key is to consider $\triangle I_AI_BI_C$ as our reference triangle.
\begin{center}
    \begin{asy}
        size(12cm);
        defaultpen(fontsize(10pt));

        pen pri=red+linewidth(0.5);
        pen sec=orange+linewidth(0.5);
        pen tri=fuchsia+linewidth(0.5);
        pen fil=red+opacity(0.05);
        pen sfil=orange+opacity(0.05);
        pen tfil=fuchsia+opacity(0.05);

        pair IA, IB, IC, A, B, C, I, M, D, P, Mp, SS, L, K, NN, IAp;
        IA=dir(110);
        IB=dir(230);
        IC=dir(310);
        A=foot(IA, IB, IC);
        B=foot(IB, IC, IA);
        C=foot(IC, IA, IB);
        I=orthocenter(IA, IB, IC);
        M=(IA+I)/2;
        D=extension(A, IA, B, C);
        P=(B+C)/2;
        Mp=(IB+IC)/2;
        SS=extension(B, C, IB, IC);
        L=intersectionpoint(SS -- (IA+(SS-IA)*0.01), circumcircle(IA, IB, IC));
        K=intersectionpoint(SS -- (M+(SS-M)*0.01), circumcircle(A, B, C));
        NN=(SS+I)/2;
        IAp=-IA;

        filldraw(circumcircle(IA, IB, IC), fil, pri);
        filldraw(circumcircle(I, SS, A), fil, pri);
        filldraw(circumcircle(A, B, C), sfil, sec);
        filldraw(circumcircle(D, P, M), sfil, sec);
        filldraw(circumcircle(M, A, NN), tfil, tri);
        filldraw(circumcircle(K, I, D), tfil, tri);

        draw(IA -- IB -- IC -- IA, pri);
        draw(IA -- SS -- IB, pri); draw(B -- SS -- M, pri);
        draw(IA -- A, sec+dashed); draw(IB -- B, sec+dashed); draw(IC -- C, sec+dashed);
        draw(K -- Mp, tri); draw(L -- IAp, tri); draw(M -- Mp, tri);
        draw(SS -- I, tri+dashed);

        dot("$I_A$", IA, N);
        dot("$I_B$", IB, IB);
        dot("$I_C$", IC, IC);
        dot("$A$", A, dir(285));
        dot("$B$", B, E);
        dot("$C$", C, NW);
        dot("$I$", I, NE);
        dot("$M$", M, dir(75));
        dot("$D$", D, dir(60));
        dot("$P$", P, dir(75));
        dot("$X$", Mp, S);
        dot("$S$", SS, SW);
        dot("$L$", L, W);
        dot("$K$", K, NW);
        dot("$N$", NN, S);
        dot("$I_A'$", IAp, IAp);
    \end{asy}
\end{center}
Clearly $I$ is the orthocenter of $\triangle I_AI_BI_C$. Now, let $X$ be the midpoint of $\overline{I_BI_C}$ and $I_A'$ be the reflection of $I$ over $X$, so that $I_A'$ is the antipode of $I_A$ on $(I_AI_BI_C)$. Also denote by $P$ the midpoint of $\overline{BC}$. Since $\Omega$ is the nine-point circle of $\triangle I_AI_BI_C$, $M$ is the midpoint of $\overline{I_AI}$ and $X$ is the antipode of $M$ on $\Omega$. Because $\measuredangle MXD=90^\circ=\measuredangle MKX$, $K,D,X$ are collinear. Furthermore, $\overline{MK},\overline{BC},\overline{XA}$ concur at the orthocenter $S$ of $\triangle DMX$.

Let $L$ be the foot from $X$ to $\overline{AS}$. Notice that $L$ is the Miquel Point of $I_BI_CBC$, so $L,I,X,I_A'$ are collinear and $\measuredangle I_ALI_A'=90^\circ$. It follows that $L\in(I_AI_BI_C)$, whence the midpoint of $\overline{IL}$ lies on $\Omega$.

Obviously $(LMAN)$ is the nine-point circle of $\triangle AIS$. Notice that \[XD\cdot XK=XP\cdot XM=XA\cdot XS=XI\cdot XL,\]
whence $L\in(KID)$, and we win.
