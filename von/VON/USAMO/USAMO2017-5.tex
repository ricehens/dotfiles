desc: Labeling lattice points
source: USAMO 2017/5
tags: [2019-10, oly, hard, combo, optimization, nice]

---

Find all real numbers $c>0$ such that there exists a labeling of the lattice points $(x,y)\in\mathbb Z^2$ with positive integers for which:
\begin{itemize}[itemsep=0em]
    \item only finitely many distinct labels occur, and
    \item for each label $i$, the distance between any two points labeled $i$ is at least $c^i$.
\end{itemize}

---

The answer is $c<\sqrt2$. We will describe a labeling for each $c<\sqrt2$, and then show $c=\sqrt2$ (and hence $c\ge\sqrt2$) does not work.

\bigskip

\textbf{Construction for $c<\sqrt2$:} Suppose that $k$ is the smallest integer such that $c^k<(\sqrt2)^{k-1}$. Toss on the complex plane, and denote \[S_1=\{z:\text{Re}(z)+\text{Im}(z)\equiv 1\hspace{-0.75em}\pmod2\}.\]
Now, let \[S_t=\{z(1+i):z\in S_{t-1}\}\]
for all $1<t<k$. Label all points in $S_t$ the label $t$, and label the rest of the points the label $k$. It is easy to see that each point is in exactly one of $S_1$, $S_2$, $\ldots$, $S_k$, so this labeling works.

\bigskip

\textbf{Proof for $c=\sqrt2$:} We will show that no labeling exists. First we prove a lemma.
\begin{lemma*}
    It is impossible to place four points in the interior of a unit square such that any two points are a distance of at least $1$ apart.
\end{lemma*}
\begin{proof}
    Consider any four points in the interior of the unit square, and let $O$ be the center of the unit square. By the Pigeonhole Principle there are two points $P$ and $Q$ such that $\angle POQ\le90^\circ$. Then $PQ^2\le OP^2+OQ^2<1$, as desired.
\end{proof}

Next we claim the following, which obviously implies the desired result.
\begin{claim*}
    Any square of size $2^n\times2^n$ must contain a cell with label greater than $2n$.
\end{claim*}
We induct on $n$, with base case $n=1$ obvious. Now suppose the claim is true for $n-1$, and assume for contradiction that it is possible to cells the points of a $2^n\times2^n$ square such that no label exceeds $2n$.

By the lemma, there are at most three cells labeled $2n$; hence without loss of generality the top-left $2^{n-1}\times2^{n-1}$ grid does not contain a cell labeled $2n$. By the inductive hypothesis, it must contain a cell labeled $2n-1$.
\begin{center}
\begin{asy}
    size(5cm); defaultpen(fontsize(10pt));
    for (real i=1; i<=3+1e-9; i+=1) {
        draw( (-i,-4)--(-i,4),gray);
        draw( (i,-4)--(i,4),gray);
        draw( (-4,-i)--(4,-i),gray);
        draw( (-4,i)--(4,i),gray);
    }
    fill( (-4,0)--(0,0)--(0,4)--(-4,4)--cycle,lightgreen+opacity(0.15));
    fill( (-2,0)--(2,0)--(2,4)--(-2,4)--cycle,cyan+opacity(0.15));
    fill( (-3,-3)--(-3,1)--(1,1)--(1,-3)--cycle,lightred+opacity(0.15));
    draw( (-4,-4)--(4,-4)--(4,4)--(-4,4)--cycle);
    draw( (-4,0)--(4,0)); draw( (0,-4)--(0,4));
    label("$5$",(-2.5,1.5));
    label("$6$",(1.5,1.5));
\end{asy}
\end{center}
Consider the grid of dimensions $2^{n-1}\times2^{n-1}$ directly to the right of the point labeled $2n-1$, with top row coinciding with the top row of the $2^n\times2^n$ grid. Clearly this grid cannot contain a $2n-1$, so it contains a $2n$; furthermore this $2n$ is in the top-right $2^{n-1}\times2^{n-1}$ grid. Then it is easy to construct a grid of dimensions $2^{n-1}\times2^{n-1}$ orthogonally adjacent to the cell labeled $2n-1$ and either orthogonally or diagonally adjacent to the square labeled $2n$. This grid contains neither a $2n-1$ nor a $2n$, contradiction.


