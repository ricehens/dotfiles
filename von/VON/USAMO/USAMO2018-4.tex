desc: Pigeonhole on remainders
source: USAMO 2018/4
tags: [2019-10, oly, easy, nt, global, pigeonhole]

---

Let $p$ be a prime, and let $a_1,\ldots,a_p$ be integers. Show that there exists an integer $k$ such that the numbers \[a_1+k,\;a_2+2k,\;\ldots,\;a_p+pk\]
produce at least $\frac12p$ distinct remainders upon division by $p$.

---

For any two $i<j$, we have \[a_i+ik\equiv a_j+jk\pmod p\iff k\equiv(a_i-a_j)(j-i)^{-1}\pmod p.\]
Hence, the number of $(i,j,k)$ with $i<j$ and $a_i+ik\equiv a_j+jk\pmod p$ is precisely $\tbinom p2$.

By Pigeonhole, for some $k$ there are at most $\tfrac{p-1}2$ pairs $i<j$ with $a_i+ik\equiv a_j+jk\pmod p$, thus there are at least $\tfrac{p+1}2$ distinct residues among $a_1+k$, $\ldots$, $a_p+pk$.

