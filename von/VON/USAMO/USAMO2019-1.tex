desc: Arrow-based FE
source: USAMO 2019/1
tags: [2019-10, oly, easy, alg, FE, arrows]

---

Let $\mathbb N$ be the set of positive integers. A function $f:\mathbb N\to\mathbb N$ satisfies the equation \[\underbrace{f(f(\ldots f}_{f(n)\text{ times}}(n)\ldots))=\frac{n^2}{f(f(n))}\]
for all positive integers $n$. Given this information, determine all possible values of $f(1000)$.

---

The answer is all evens. It is not hard to check that any $f$ that fixes the odds and is an involution on the evens works. To prove $f(1000)$ must be even, we present two solutions.

\paragraph{First solution, by induction} In what follows, $f^k(n)$ means $f$ iterated $k$ times. Here we will prove all solutions to the functional equation are of the above form: $f$ fixes odds and is an involution on the evens. Note that to prove only the original problem statement, Claims 1 and 2 suffice.
\setcounter{claim}0
\begin{claim}
    $f$ is injective.
\end{claim}
\begin{proof}
    If $f(a)=f(b)$, then \[a^2=f^2(a)f^{f(a)}(a)=f^2(b)f^{f(b)}(b)=b^2,\]
    so $a=b$ follows.
\end{proof}
\begin{claim}
    If $n$ is odd, then $f(n)=n$.
\end{claim}
\begin{proof}
    We use strong induction on $n$, with no base case. Assume the claim holds for all odd positive integers less than $n$.

    Consider the equation \[f^2(n)f^{f(n)}(n)=n^2.\]
    I contend both terms on the left equal $n$. Let $m=f^2(n)$. If $m<n$, then $f^2(n)=f^2(m)$ by inductive hypothesis, so $n=m$ by injectivity, absurd. Thus $f^2(n)\ge n$. Analogously if $m=f^{f(n)}(n)$ and $m<n$, then $f^{f(n)}(n)=f^{f(n)}(m)$, so $n=m$ by injectivity, absurd. Thus we have $f^2(n)=f^{f(n)}(n)=n$.

    Now the sequence $n$, $f(n)$, $f^2(n)$, $\ldots$ repeats with period $2$, so if $f(n)$ is odd, then $f(n)=f^{f(n)}(n)=n$. Otherwise suppose $f(n)$ is even, and let $m=f(n)$. Then $f(m)=n$, so $m^2=f^2(m)f^{f(m)}(m)=n^2$, contradiction.
\end{proof}
\begin{claim}
    If $n$ is even, then $f^2(n)=n$.
\end{claim}
\begin{proof}
    The proof is similar to that of Claim 2. We use strong induction on $n$, with no base case. Assume the claim holds for all even positive integers less than $n$; also recall that $f$ fixes odds and is injective, so $f(n)$ is even for all even $n$.

    Again consider the equation \[f^2(n)f^{f(n)}(n)=n^2.\]
    It will suffice to show, once more, both terms on the left equal $n$. Let $m=f^2(n)$. If $m<n$, then $f^2(n)=f^2(m)$, so $n=m$ by injectivity, absurd, so $f^2(n)\ge n$. Similarly if $m=f^{f(n)}(n)$ and $m<n$, then $f^{f(n)}(n)=m=f^{f(n)}(m)$, so $n=m$ by injectivity, absurd.

    Hence $f^2(n)=n$, as needed.
\end{proof}

\paragraph{Second solution, by arrows (Espen Slettnes)} Just as we did above, we first show $f$ injective. Indeed, if $f(a)=f(b)$, then $a^2=f^2(a)f^{f(a)}(a)=f^2(b)=f^{f(b)}(b)=b^2$, so $f$ is injective.

Consider the sequence defined by $x_0=1000$ and $x_i=f(x_{i-1})$ for all $i\ge1$. Letting $n=x_i$ in the functional equation gives \[a_{i+2}a_{i+a_{i+1}}=a_i^2.\tag{$*$}\]
Take $i$ so that $a_i$ is minimal. Then $a_{i+2}\ge a_i$ and $a_{i+a_{i+1}}\ge a_i$, so by $(*)$ we must have $a_{i+2}=a_i$. By injectivity this reduces to $a_2=a_0$, so $(a_i)_{i\ge0}$ repeats with period $2$.

Take $i=0$ in $(*)$ to obtain $a_{a_1}=1000$. Then $a_1=1000$ or $a_1$ is even. End proof.


