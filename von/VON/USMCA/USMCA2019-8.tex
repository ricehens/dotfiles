desc: When is (2^m-1)(2^n-1) a square?
source: USMCA 2019/8
tags: [2020-03, oly, brutal, nt, expnt, involved]

---

Find all pairs of positive integers $(m,n)$ such that $(2^m-1)(2^n-1)$ is a perfect square.

---

The answer is $m=n$ or $\{m,n\}=\{3,6\}$, which clearly work.
\begin{remark}
    To complete the inductions, there are two prominent ideas:
    \begin{itemize}
        \item If $x$, $y$ are relatively prime and $xy$ is a perfect square, then $x$ and $y$ are themselves perfect squares.
        \item If $p\mid2^a-1$, then by LTE \[\nu_p\left(\frac{2^{ab}-1}{2^a-1}\right)=\nu_p(b).\]
    \end{itemize}
\end{remark}
\setcounter{claim}0
\setcounter{lemma}0
\begin{lemma}[Reduction to $m\mid n$]
    It suffices to show \[\dfrac{2^{ab}-1}{2^a-1}\]
    is a perfect square if and only if $b=1$ or $(a,b)=(3,2)$.
\end{lemma}
\begin{proof}
    Let $k=\gcd(m,n)$. Then $\gcd(2^m-1,2^n-1)=2^k-1$, so \[\frac{2^m-1}{2^k-1}\cdot\frac{2^n-1}{2^k-1}\]
    is a perfect square, but both terms are relatively prime, so each is a square.

    If it holds that either $m=k$ or $(m,k)=(6,3)$ and either $n=k$ or $(n,k)=(6,3)$, then
    \begin{itemize}[itemsep=0em]
        \item if either $m=k$ or $n=k$, then either $m\mid n$ or $n\mid m$;
        \item the case $(m,k)=(n,k)=(6,3)$ is impossible, since $\gcd(m,n)=6$.
    \end{itemize}
    Thus this suffices.
\end{proof}
\begin{lemma}[Reduction to $b$ odd]
    If $b$ is even, then $(a,b)=(3,2)$.
\end{lemma}
\begin{proof}
    If $b=2c$, then \[\frac{2^{2ac}-1}{2^a-1}=\left(2^{ac}+1\right)\frac{2^{ac}-1}{2^a-1}.\]
    If $p$ divides $\tfrac{2^{ac}-1}{2^a-1}$, then it divides $2^{ac}-1$ and since $p\ne2$, then $p$ does not divide $2^{ac}+1$. Thus the two terms are relatively prime, so $2^{ac}+1$ is a square $x^2$. From this $2^{ac}=x^2-1=(x-1)(x+1)$, so $ac=3$, and the rest is a finite case check.
\end{proof}

Henceforth $b$ is odd, and it suffices to prove no such $(a,b)$ work.
\begin{lemma}[Problem for $a=q^\ell\cdot d$, $b=q$]
    Let $q$ be an odd prime and $\ell\ge1$. Then \[\frac{2^{q^{\ell+1}\cdot d}-1}{2^{q^\ell\cdot d}-1}\]
    is not a square.
\end{lemma}
\begin{proof}
    Write the expression as \[\frac{2^{q^{\ell+1}\cdot d}-1}{2^{q^\ell\cdot d}-1}=1+2^{q^\ell\cdot d}+\cdots+2^{q^\ell(q-1)\cdot d}.\]
    We will bound this between two squares.

    Let $c_n:=\frac1{4^n}\binom{2n}n$, so that $\sum_{k=0}^nc_kc_{n-k}=1$ for all $n$. This follows from squaring the generating function \[\frac1{\sqrt{1-x}}=\sum_{n\ge0}c_nx^n.\]
    Let $r=(q-1)/2$, and consider the quantity \[P:=\sum_{k=1}^r2^{q^\ell\cdot dk}c_{r-k}.\]
    It is not hard to check that $P$ is an integer.

    I claim $P^2<1+2^{q^\ell\cdot d}+\cdots+2^{q^\ell(q-1)\cdot d}<(P+1)^2$. The lower bound follows from \[P^2=\sum_{k=1}^{2r}\left(2^{q^{\ell}\cdot dk}\sum_{j=1}^{k-1}c_{r-j}c_{r-k+j}\right)<\sum_{k=1}^{2r}2^{q^\ell\cdot dk}.\]
    The lower bound follows from
    \begin{align*}
        (P+1)^2&=1+2\sum_{k=1}^r2^{q^\ell\cdot dk}c_{r-k}+\sum_{k=1}^{2r}\left(2^{q^\ell\cdot dk}\sum_{j=1}^{k-1}c_{r-j}c_{r-k+j}\right)\\
        &\ge1+2\cdot2^{q^\ell\cdot dr}+\sum_{k=r+1}^{2r}2^{q^\ell\cdot dk}\\
        &\ge\sum_{k=1}^{2r}2^{q^\ell\cdot dk},
    \end{align*}
    thus the lemma is proven.
\end{proof}
\begin{lemma}[Problem for $a=q^kd$, $b=q^{\ell-k}$]
    Let $q$ be the minimal prime divisor of $a$, and let $a=q^kd$, where $q\nmid d$. Suppose $q$ is odd and $b=q^{\ell-k}$ for $\ell>k>0$. Then \[\frac{2^{q^{\ell}\cdot d}-1}{2^{q^k\cdot d}-1}\]
    is not a square.
\end{lemma}
\begin{proof}
    Write \[\frac{2^{q^{\ell}\cdot d}-1}{2^{q^k\cdot d}-1}=\frac{2^{q^{\ell-1}\cdot d}-1}{2^{q^k\cdot d}-1}\cdot\frac{2^{q^{\ell}\cdot d}-1}{2^{q^{\ell-1}\cdot d}-1}.\]
    If $p$ divides the left term, then $p$ divides $2^{q^{\ell-1}\cdot d}-1$, so \[\nu_p\left(\frac{2^{q^\ell\cdot d}-1}{2^{q^{\ell-1}\cdot d}-1}\right)=\nu_p(q)\]
    by LTE. However $p\ne q$, since otherwise $q\mid\gcd\left(2^{q-1}-1,2^{q^{\ell-1}\cdot d}-1\right)=1$, contradiction.

    Hence the two terms are relatively prime, but the right term is not a square by Lemma 3.
\end{proof}
\begin{lemma}[Problem for $b$ a prime power]
    If $b>1$ is an odd prime power, then \[\frac{2^{ab}-1}{2^a-1}\]
    is not a square.
\end{lemma}
\begin{proof}
    We induct on the number of distinct prime factors of $a$. Let $q$ be the smallest prime divisor of $a$, and let $a=q^kd$, where $q\nmid d$. Note that if $b$ is a prime power of $q$, then we are done by Lemma 4. Henceforth $q\nmid b$.

    Write \[\frac{2^{q^kd\cdot b}-1}{2^{q^kd}-1}=\left(\frac{2^{q^kdb}-1}{2^{q^kd}-1}\cdot\frac{2^d-1}{2^{db}-1}\right)\cdot\frac{2^{db}-1}{2^d-1}.\]
    (Here, the left term is the product of the two fractions in parentheses.) Note that the left term can be written as \[\frac{2^{q^kdb}-1}{2^{q^kd}-1}\cdot\frac{2^d-1}{2^{db}-1}=\prod_{\substack{i\mid q^kb\\ i\nmid q^k}}\Phi_i(2^d)\bigg/\prod_{i\mid b}\Phi_i(2^d)=\prod_{i\mid b}\frac{\Phi_{q^ki}(2^d)}{\Phi_i(2^d)},\]
    which is an integer. I claim the left and right terms are relatively prime.

    Let $p$ divide the right term. If $p\mid2^d-1$, then $p$ does not divide the left term by a simple application of LTE: \[\nu_p\left(\frac{2^{q^kdb}-1}{2^{q^kd}-1}\right)=\nu_p\left(q^kb\right)-\nu_p\left(q^k\right)=\nu_p(b)=\nu_p\left(\frac{2^{db}-1}{2^d-1}\right).\]
    Otherwise assume $p\nmid2^d-1$ but $p\mid2^{db}-1$. If $p\mid2^{q^kd}-1$, then we have a contradiction by \[p\mid\gcd\left(2^{q^kd}-1,2^{db}-1\right)=2^d-1,\]
    where the last step is since $q\nmid b$.

    Finally note that \[\nu_p\left(\frac{2^{q^kdb}-1}{2^{db}-1}\right)=\nu_p\left(q^k\right)=0\]
    by LTE, so $p$ does not divide the left term.

    Hence the two terms are relatively prime, and so the right term is a square. This completes the inductive step. The base case is $a=1$, for which $2^b-1$. But if $b\ge2$, then $2^b-1\equiv3\pmod4$, whence $b=1$, and the lemma is proven.
\end{proof}
\begin{lemma}[Final induction]
    If $b>1$ is odd then \[\frac{2^{ab}-1}{2^a-1}\]
    is not a square.
\end{lemma}
\begin{proof}
    We induct on the number of distinct prime divisors of $b$. The base case is Lemma 5. Let $q$ be the smallest prime divisor of $b$, and let $b=q^kc$ where $q\nmid c$. Write \[\frac{2^{q^kac}-1}{2^a-1}=\frac{2^{q^kac}-1}{2^{ac}-1}\cdot\frac{2^{ac}-1}{2^a-1}.\]
    First I claim $q$ does not divide the right term.

    We know $\ord_q(2)\mid ac$, but $\gcd(\ord_q(2),c)=1$ since $\ord_q(2)\le q-1$ and $q$ is the smallest prime divisor of $a$. Thus $\ord_q(2)\mid a$, so $q\mid2^a-1$. Then it follows from LTE that \[\nu_q\left(\frac{2^{ac}-1}{2^a-1}\right)=\nu_q(c)=0.\]

    I claim the left and right terms are relatively prime. Let $p$ divide the right term, then $p\mid2^{ac}-1$, so by LTE \[\nu_p\left(\frac{2^{q^kac}-1}{2^{ac}-1}\right)=\nu_p\left(q^k\right)=0\]
    since $q\ne p$.

    Thus the right term is a square, completing the inductive step.
\end{proof}

The statement in Lemma 1 is verified: if $\tfrac{2^{ab}-1}{2^a-1}$ is a square, then either $b=1$ or $(a,b)=(3,2)$. The conclusion follows.
\begin{remark}
    It is worth noting that the solution is written backwards. After the reductions presented in Lemma 1 and Lemma 2, the rest of the lemmas are discovered in reverse order.
\end{remark}
\begin{remark}
    Some problems similar in appearance are \href{https://artofproblemsolving.com/community/c6h254153p1389586}{China TST 2002/7/2} and \href{https://artofproblemsolving.com/community/c6h355801p1932947}{ISL 2009 N7}.
\end{remark}
